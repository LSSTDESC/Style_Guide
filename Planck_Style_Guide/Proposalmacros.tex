% MidexFORMAT - TeX macros for Midex Proposal
% 20-APR-1995 (CRL)
%-----------------------------------------------------------------------
%-----------------------------------------------------------------------
% Set page size.
\hsize=6.5in
\vsize=9in
\hoffset=0in
\voffset=0in
\bigskipamount=12pt plus4pt minus4pt
\parskip=0pt
\parindent=2em
%-----------------------------------------------------------------------
\def\raggedcenter{\leftskip=0pt plus6em \rightskip=\leftskip
   \parfillskip=0pt \spaceskip=.3333em \xspaceskip=.5em
   \pretolerance=9999 \tolerance=9999
   \hyphenpenalty=9999 \exhyphenpenalty=9999} 
%-----------------------------------------------------------------------
\def\red{\special{color rgb 1.0 0 0}}
\def\green{\special{color cmyk 1.0 0 1.0 .5}}
\def\blue{\special{color rgb 0 0 1.0}}
\def\magenta{\special{color cmyk 0 1.0 0 0}}
\def\maroon{\special{color cmyk 0 0.87 0.68 0.32}}
\def\black{\special{color cmyk 0 0 0 1.0}}
%-----------------------------------------------------------------------
% Font definitions 
%
\font\twentytwobf=cmbx12 at 22pt
\font\twentytwobi=cmmib10 at 22pt
\font\twentytwobsy=cmbsy10 at 22pt
%
\font\fourteenrm=cmr12 at 14.4pt 
\font\fourteeni=cmmi12 at 14.4pt 
\font\fourteenit=cmti12 at 14.4pt
\font\fourteensy=cmsy10 at 14.4pt
\font\fourteensl=cmsl12 at 14.4pt
\font\fourteenbf=cmbx12 at 14.4pt
\font\fourteenbi=cmmib10 at 14.4pt
\font\fourteenbsy=cmbsy10 at 14.4pt
\font\fourteencsc=cmcsc10 at 14.4pt
%
\font\twelverm=cmr12
\font\twelvei=cmmi12
\font\twelveit=cmti12
\font\twelvesy=cmsy10 at 12pt
\font\twelvesl=cmsl12  
\font\twelvebf=cmbx12
\font\twelvebi=cmmib10 at 12pt 
\font\twelvebsy=cmbsy10 at 12pt 
\font\twelvecsc=cmcsc10 at 12pt 
%
\font\elevenrm=cmr10 at 11pt
\font\eleveni=cmmi10 at 11pt
\font\elevenit=cmti10 at 11pt
\font\elevensy=cmsy10 at 11pt
\font\elevensl=cmsl10 at 11pt
\font\elevenbf=cmbx10 at 11pt
\font\elevenbi=cmmib10 at 11pt
\font\elevenbsy=cmbsy10 at 11pt
\font\elevencsc=cmcsc10 at 11pt
%
\font\elevenbfi=cmbxsl10 at 11pt  % what is this for? --- 10/99
%
\font\tenrm=cmr10   
\font\teni=cmmi10   
\font\tenit=cmti10  
\font\tensy=cmsy10  
\font\tensl=cmsl10  
\font\tenbf=cmbx10  
\font\tenbi=cmmib10 
\font\tenbsy=cmbsy10 
\font\tencsc=cmcsc10
%
\font\ninerm=cmr10 at 9pt
\font\ninei=cmmi10 at 9pt
\font\nineit=cmti10 at 9pt
\font\ninesy=cmsy10 at 9pt
\font\ninesl=cmsl10 at 9pt
\font\ninebf=cmbx10 at 9pt
\font\ninecsc=cmcsc10 at 9pt
%
\font\eightrm=cmr8
\font\eighti=cmmi8
\font\eightit=cmti8
\font\eightsy=cmsy8
\font\eightsl=cmsl8
\font\eightbf=cmbx8
\font\eightbi=cmmib10 at 8pt
\font\eightbsy=cmbsy10 at 8pt
\font\eightcsc=cmcsc10 at 8pt
%
\font\sevenrm=cmr7
\font\seveni=cmmi7
\font\sevenit=cmti7
\font\sevensy=cmsy7
\font\sevenbf=cmbx7
\font\sevenbi=cmmib10 at 7pt
\font\sevenbsy=cmbsy10 at 7pt
%
\font\sixrm=cmr6
\font\sixi=cmmi6
\font\sixit=cmti7 at 6pt
\font\sixsy=cmsy6
\font\sixbf=cmbx6
\font\sixbi=cmmib10 at 6pt
\font\sixbsy=cmbsy10 at 6pt
%
\font\fiverm=cmr5
\font\fivei=cmmi5
\font\fiveit=cmti7 at 5pt
\font\fivesy=cmsy5
\font\fivebf=cmbx5
\font\fivebi=cmmib10 at 5pt
\font\fivebsy=cmbsy10 at 5pt
%-----------------------------------------------------------------------
\def\fourteenpoint{\def\rm{\fam0\fourteenrm}%
    \textfont0=\fourteenrm \scriptfont0=\twelverm \scriptscriptfont0=\tenrm
    \textfont1=\fourteeni  \scriptfont1=\twelvei  \scriptscriptfont1=\teni
    \textfont2=\fourteensy \scriptfont2=\twelvesy \scriptscriptfont2=\tensy
    \textfont\itfam=\fourteenit  \def\it{\fam\itfam\fourteenit}%
    \textfont\slfam=\fourteensl  \def\sl{\fam\slfam\fourteensl}%
    \textfont\bffam=\fourteenbf  \scriptfont\bffam=\twelvebf
      \scriptscriptfont\bffam=\tenbf   \def\bf{\fam\bffam\fourteenbf}%
    \normalbaselineskip=20pt
    \setbox\strutbox=\hbox{\vrule height15pt depth5pt width0pt}%
    \let\sc=\twelverm  
    \normalbaselines\rm}
%-----------------------------------------------------------------------
\def\twelvepoint{\def\rm{\fam0\twelverm}%
    \textfont0=\twelverm \scriptfont0=\tenrm\scriptscriptfont0=\sevenrm
    \textfont1=\twelvei  \scriptfont1=\teni  \scriptscriptfont1=\seveni
    \textfont2=\twelvesy \scriptfont2=\tensy \scriptscriptfont2=\sevensy
    \textfont\itfam=\twelveit  \def\it{\fam\itfam\twelveit}%
    \textfont\slfam=\twelvesl  \def\sl{\fam\slfam\twelvesl}%
    \textfont\bffam=\twelvebf  \scriptfont\bffam=\tenbf
      \scriptscriptfont\bffam=\tenbf   \def\bf{\fam\bffam\twelvebf}%
    \normalbaselineskip=16pt
    \setbox\strutbox=\hbox{\vrule height12pt depth4pt width0pt}%
    \let\sc=\tenrm  \let\csc=\twelvecsc  \let\er=\tenrm
    \normalbaselines\rm}
%-----------------------------------------------------------------------
\def\elevenpoint{\def\rm{\fam0\elevenrm}%
    \textfont0=\elevenrm \scriptfont0=\eightrm \scriptscriptfont0=\sixrm
    \textfont1=\eleveni  \scriptfont1=\eighti  \scriptscriptfont1=\sixi
    \textfont2=\elevensy \scriptfont2=\eightsy \scriptscriptfont2=\sixsy
    \textfont\itfam=\elevenit  \def\it{\fam\itfam\elevenit}%
    \textfont\slfam=\elevensl  \def\sl{\fam\slfam\elevensl}%
    \textfont\bffam=\elevenbf  \scriptfont\bffam=\eightbf
      \scriptscriptfont\bffam=\sixbf   \def\bf{\fam\bffam\elevenbf}%
    \normalbaselineskip=13pt
    \setbox\strutbox=\hbox{\vrule height8.5pt depth3.5pt width0pt}%
    \let\sc=\eightrm  \let\csc=\elevencsc  \let\er=\eightrm
    \normalbaselines\rm}
%-----------------------------------------------------------------------
\def\tenpoint{\def\rm{\fam0\tenrm}%
    \textfont0=\tenrm \scriptfont0=\sevenrm \scriptscriptfont0=\fiverm
    \textfont1=\teni  \scriptfont1=\seveni  \scriptscriptfont1=\fivei
    \textfont2=\tensy \scriptfont2=\sevensy \scriptscriptfont2=\fivesy
    \textfont\itfam=\tenit  \def\it{\fam\itfam\tenit}%
    \textfont\slfam=\tensl  \def\sl{\fam\slfam\tensl}%
    \textfont\bffam=\tenbf  \scriptfont\bffam=\sevenbf
      \scriptscriptfont\bffam=\fivebf   \def\bf{\fam\bffam\tenbf}%
    \normalbaselineskip=12pt
    \setbox\strutbox=\hbox{\vrule height8.5pt depth3.5pt width0pt}%
    \let\sc=\sevenrm  \let\csc=\tencsc  \let\er=\eightrm
    \normalbaselines\rm}
%-----------------------------------------------------------------------
\def\ninepoint{\def\rm{\fam0\ninerm}%
    \textfont0=\ninerm \scriptfont0=\sevenrm \scriptscriptfont0=\fiverm
    \textfont1=\ninei  \scriptfont1=\seveni  \scriptscriptfont1=\fivei
    \textfont2=\ninesy \scriptfont2=\sevensy \scriptscriptfont2=\fivesy
    \textfont\itfam=\nineit  \def\it{\fam\itfam\nineit}%
    \textfont\slfam=\ninesl  \def\sl{\fam\slfam\ninesl}%
    \textfont\bffam=\ninebf  \scriptfont\bffam=\sevenbf
      \scriptscriptfont\bffam=\fivebf   \def\bf{\fam\bffam\ninebf}%
    \normalbaselineskip=10.5pt
    \setbox\strutbox=\hbox{\vrule height7.75pt depth2.75pt width0pt}%
    \let\sc=\sevenrm  \let\csc=\ninecsc  \let\er=\sevenrm
    \normalbaselines\rm}
%-----------------------------------------------------------------------
\def\eightpoint{\def\rm{\fam0\eightrm}%
    \textfont0=\eightrm \scriptfont0=\sixrm \scriptscriptfont0=\fiverm
    \textfont1=\eighti  \scriptfont1=\sixi  \scriptscriptfont1=\fivei
    \textfont2=\eightsy \scriptfont2=\sixsy \scriptscriptfont2=\fivesy
    \textfont\itfam=\eightit  \def\it{\fam\itfam\eightit}%
    \textfont\slfam=\eightsl  \def\sl{\fam\slfam\eightsl}%
    \textfont\bffam=\eightbf  \scriptfont\bffam=\sixbf
      \scriptscriptfont\bffam=\fivebf  \def\bf{\fam\bffam\eightbf}%
    \normalbaselineskip=9pt
    \setbox\strutbox=\hbox{\vrule height7pt depth2pt width0pt}%
    \let\sc=\sixrm  \let\er=\sixrm
    \normalbaselines\rm}
%-----------------------------------------------------------------------
\def\twentytwoboldmath{\def\rm{\fam0\twentytworm}
    \textfont0=\twentytwobf \scriptfont0=\fourteenbf
    \textfont1=\twentytwobi  \scriptfont1=\fourteenbi
    \textfont2=\twentytwobsy \scriptfont2=\fourteenbsy
     }
%
\def\elevenboldmath{\def\rm{\fam0\elevenrm}
    \textfont0=\elevenbf \scriptfont0=\eightbf \scriptscriptfont0=\sixbf
    \textfont1=\elevenbi  \scriptfont1=\eightbi  \scriptscriptfont1=\sixbi
    \textfont2=\elevenbsy \scriptfont2=\eightbsy \scriptscriptfont2=\sixbsy
     }
%
\def\tenboldmath{\def\rm{\fam0\tenrm}
    \textfont0=\tenbf \scriptfont0=\sevenbf \scriptscriptfont0=\fivebf
    \textfont1=\tenbi  \scriptfont1=\sevenbi  \scriptscriptfont1=\fivebi
    \textfont2=\tenbsy \scriptfont2=\sevenbsy \scriptscriptfont2=\fivebsy
     }
%
\def\nineboldmath{\def\rm{\fam0\ninerm}
    \textfont0=\ninebf \scriptfont0=\sevenbf \scriptscriptfont0=\fivebf
    \textfont1=\ninebi  \scriptfont1=\sevenbi  \scriptscriptfont1=\fivebi
    \textfont2=\ninebsy \scriptfont2=\sevenbsy \scriptscriptfont2=\fivebsy
     }
%
\def\eightboldmath{\def\rm{\fam0\eightrm}
    \textfont0=\eightbf \scriptfont0=\sixbf \scriptscriptfont0=\fivebf
    \textfont1=\eightbi  \scriptfont1=\sixbi  \scriptscriptfont1=\fivebi
    \textfont2=\eightbsy \scriptfont2=\sixbsy \scriptscriptfont2=\fivebsy
     }
%
\font\title=cmb10 at 24pt 
\font\subtitle=cmbx10 at 22pt 
\font\ssq=cmssq8 at 16pt
%
\def\bL2{$\elevenboldmath L_2$}
%-----------------------------------------------------------------------
% Some variables.
%-----------------------------------------------------------------------
\newcount\cpnumb % 0 on first page of each chapter
\newcount\sectnumber\sectnumber=0
\newcount\subsectnumber
\newcount\subsubsectnumber
\newcount\footnotenumber
\newcount\bullnumber
\newcount\figureno  \newcount\tableno
\newcount\nx 
\newcount\versionA
\newcount\versionB
\newcount\versionC
\def\nn{\advance\nx by 1\smallskip\noindent\number\nx.\ }
%
\newbox\tablebox    \newdimen\tablewidth
%
\newtoks\toksAP
\newtoks\toksA
\newtoks\toksB
\newtoks\toksC
%-----------------------------------------------------------------------
% Headers and footers. 
%-----------------------------------------------------------------------
\footline={\hfil}
\headline={\ifnum\pageno=0\hfil
           \else%\rightheadline\fi}
              \ifodd\pageno\rightheadline \else\leftheadline\fi
          \fi}
\def\rightheadline{%
     \rlap{\eightpoint Version~\number\versionA.\number\versionB\number\versionC, 
           \today}
     \elevenpoint\it\hfil\botmark\hfil\rm\folio}
\def\leftheadline{\elevenpoint\rm\folio\hfil\it\botmark\hfil
     \llap{\eightpoint Version~\number\versionA.\number\versionB\number\versionC, 
           \today}}
%
\def\today{\number\year\space\ifcase\month\or
    January\or February\or March\or April\or May\or June\or
    July\or August\or September\or October\or November\or December\fi
    \space\number\day }
%-----------------------------------------------------------------------
% Table of Contents
\newwrite\toc
%\openout\toc=toc
\newdimen\tocwidth
\def\setuptoc{\parindent=0pt\elevenpoint\baselineskip=16pt
     \setbox0=\hbox{10\quad}\tocwidth=\wd0}
%-----------------------------------------------------------------------
%
\widowpenalty=800
%
\def\verso{}
\def\recto{}
%
%
\def\TCAP#1|#2.{\vskip 3pt\line{\hbox to \tocwidth{#1\hfil}#2\hfil}}
\def\TCA#1|#2|#3.{\vskip 3pt\line{\hbox to \tocwidth{#1\hfil}#2\hfil#3}}
\def\TCB#1|#2|#3.{\line{\hglue 2em#1\hskip 10pt#2\hfil#3}}
\def\TCC#1|#2|#3.{\line{\hglue 4em#1\hskip 10pt{\it #2}\hfil#3}}
%
\toksAP={\TCAP}
\outer\def\appendix#1;#2\par{
   \def\rightheadline{\elevenpoint\it\hfil\botmark\hfil\rm#1--\folio}
   \def\leftheadline{\elevenpoint\rm#1--\folio\hfil\it\botmark\hfil}
   \pageno=1\mark{APPENDIX~#1~#2}
   \bigskip\leftline{\twelvebf APPENDIX~#1~#2}
  {\let\the=0\edef\next{\write\toc{\the\toksAP |APPENDIX~#1\quad#2.}}\next}%
     \nobreak\medskip}
%
\toksA={\TCA}
\outer\def\section#1\par{\advance\sectnumber by 1\footnotenumber=0\subsectnumber=0%
%     \def\verso{\number\sectnumber~#1}
%     \def\recto{\number\sectnumber~#1}
     \vskip 17pt plus 4pt minus 3pt\goodbreak
     \leftline{{\twelvebf\number\sectnumber\enspace#1}}
     \mark{\number\sectnumber~#1}
  {\let\the=0\edef\next{\write\toc{\the\toksA
     \number\sectnumber|#1|\the\count0.}}\next}%
     \nobreak\medskip\noindent}
%
\toksB={\TCB}
\outer\def\subsection#1\par{\subsubsectnumber=0\advance\subsectnumber by1%
    \ifdim\lastskip=\medskipamount\nobreak\vskip 8pt plus 2pt minus 2pt
      \else \vskip 14pt plus 3pt minus 3pt\goodbreak\fi
     \leftline{{\bf\number\sectnumber.\number\subsectnumber\enspace#1}}%
  {\let\the=0\edef\next{\write\toc{\the\toksB
     \number\sectnumber.\number\subsectnumber|#1|\the\count0.}}\next}%
     \nobreak\medskip\noindent}
%
\toksC={\TCC}
\outer\def\subsubsection#1\par{\advance\subsubsectnumber by1%
    \ifdim\lastskip=\medskipamount\nobreak\vskip 8pt plus 2pt minus 2pt
      \else \vskip 14pt plus 3pt minus 3pt\goodbreak\fi
     \leftline{{\elevenpoint
     \it\number\sectnumber.\number\subsectnumber.\number\subsubsectnumber\enspace
     #1}}%
  {\let\the=0\edef\next{\write\toc{\the\toksC \number\sectnumber.%
            \number\subsectnumber.\number\subsubsectnumber|#1|\the\count0.}}\next}%
     \nobreak\medskip\noindent}
%
%\outer\def\appendix#1\par{\subsubsectnumber=0%
%\bigskip\bigskip\bigskip\goodbreak\centerline{\bf Appendix.\enspace#1}%
%\nobreak\medskip}
%
\outer\def\beginreferences{\bigskip\goodbreak%
\centerline{\bf References}\nobreak\medskip\vskip-\parskip\noindent%
\begingroup\tolerance=1000\eightpoint\frenchspacing}
%
\outer\def\endreferences{\nonfrenchspacing\elevenpoint\endgroup}
%-----------------------------------------------------------------------
% Figures.
%-----------------------------------------------------------------------
\def\figure #1 {\advance\figureno by 1
        \ninepoint
           \ifdim#1>7.995in \pageinsert\vskip 0pt plus 16000fill
               F{\sc IG} \number\figureno---\readtoblankline
           \fi
           \ifdim#1<7.995in \midinsert\vskip #1
               F{\sc IG} \number\figureno---\readtoblankline
           \fi}
%
\def\psfigure #1#2#3#4#5{\advance\figureno by 1
           \ninepoint%
           \midinsert{\boxmaxdepth=0pt
              \moveleft #1\vbox to #2{
              \hbox{\lower #3\hbox{\special{postscriptfile #5 scaled #4}}}}}
              F{\sc IG} \number\figureno .---\readtoblankline
            }
%
\def\epsfigure #1#2{\advance\figureno by 1
           \ninepoint%
           \midinsert{\centerline{\epsfxsize=#1 \epsfbox{#2}}}
              F{\sc IG} \number\figureno .---\readtoblankline
            }
%
\def\epssfigure #1#2#3#4{\advance\figureno by 1
           \ninepoint%
           \midinsert{\centerline{\epsfysize=#1 \epsfbox{#2}}
                      \vskip 5pt
                      \centerline{\epsfysize=#3 \epsfbox{#4}}}
              F{\sc IG} \number\figureno .---\readtoblankline
            }
%
\def\epsssfigure #1#2#3#4#5#6{\advance\figureno by 1
           \ninepoint%
           \midinsert{\centerline{\epsfysize=#1 \epsfbox{#2}}
                      \vskip 5pt
                      \centerline{\epsfysize=#3 \epsfbox{#4}}
                      \vskip 5pt
                      \centerline{\epsfysize=#5 \epsfbox{#6}}}
              F{\sc IG} \number\figureno .---\readtoblankline
            }
%
\def\epxxfigure #1#2#3#4{\advance\figureno by 1
           \ninepoint%
           \midinsert{\centerline{\hglue -8pt\epsfxsize=#1 \epsfbox{#2} 
                                   \epsfxsize=#3 \epsfbox{#4}}}
              F{\sc IG} \number\figureno.---\readtoblankline
            }
%
\def\epyyfigure #1#2#3#4{\advance\figureno by 1
           \ninepoint%
           \midinsert{\centerline{\hglue -8pt\epsfysize=#1 \epsfbox{#2} 
                                   \epsfysize=#3 \epsfbox{#4}}}
              F{\sc IG} \number\figureno.---\readtoblankline
            }
%
\def\readtoblankline{\obeylines\readcap}
{\obeylines  \gdef\readcap#1 
  {\def\next{#1}%
   \ifx\next\empty\let\next=\endread %
   \else\next\space \let\next=\readcap\fi\next}}
\def\endread{\vfil\endinsert\elevenpoint}
%
\def\copytoblankline{\begingroup\setupcopy\copycap}
{\obeylines  \gdef\copycap#1
  {\def\next{#1}%
   \ifx\next\empty\let\next=\endcopy %
   \else\immediate\write\caps{\next} \let\next=\copycap\fi\next}}
\def\endcopy{\endgroup\immediate\write\caps{\string\par}}
%
\chardef\other=12
\def\setupcopy{\def\do##1{\catcode`##1=\other}\dospecials
               \catcode`\|=\other \obeylines}
\gdef\copyit#1{\immediate\write\caps{#1\string\par}}

%-----------------------------------------------------------------------
% Tables. 
%-----------------------------------------------------------------------
%
% To leave spaces in drafts for PostScript tables
%
\def\pstable #1#2#3#4#5#6{\advance\tableno by 1
           \begingroup
           \midinsert \boxmaxdepth=0pt
           \centerline{TABLE \number\tableno}
           \vskip 7pt
           \ninepoint
           \centerline{#6}\nointerlineskip\noindent\vglue 0.01pt
              \moveleft #1\vbox to #2{%
              \hbox{\lower #3\hbox{\special{postscriptfile #5 scaled #4}}}}%
              \endinsert
           \endgroup
            }
%
% Regular tables
%
\def\tabl #1\par{\centerline{TABLE \number\tableno}
                 \vskip 7pt
                 \centerline{#1}\nointerlineskip\noindent}
\def\table #1\par{\advance\tableno by 1
            \goodbreak\midinsert{\ninepoint\input #1 }\endinsert}
\def\endtable{\tablewidth=\wd\tablebox 
    $$\hss\copy\tablebox\hss$$
    \vskip-\lastskip\vskip -2pt}
\def\tablenote#1 #2\par{\begingroup \parindent=0.8em
    \abovedisplayshortskip=0pt\belowdisplayshortskip=0pt
    \noindent
    $$\hss\vbox{\hsize\tablewidth \hangindent=\parindent \hangafter=1 \noindent
    \hbox to \parindent{\sup{\rm #1}\hss}\strut#2\strut\par}\hss$$
    \endgroup}
\def\doubleline{\vskip 3pt\hrule \vskip 1.5pt \hrule \vskip 5pt}
%-----------------------------------------------------------------------
% Footnotes.
%-----------------------------------------------------------------------
\catcode`@=11 % borrow the private macros of PLAIN (with care)
%\newinsert\footins
\def\footnote#1{\edef\@sf{\spacefactor\the\spacefactor}#1\@sf
      \insert\footins\bgroup\ninepoint\parindent=2.0em
      \interlinepenalty100 \let\par=\endgraf
        \leftskip=\z@skip \rightskip=\z@skip
        \splittopskip=10pt plus 1pt minus 1pt \floatingpenalty=20000
        \smallskip\noindent#1\quad\bgroup\strut\aftergroup\@foot\let\next}
\skip\footins=12pt plus 2pt minus 4pt % space added when footnote is present
\dimen\footins=30pc % maximum footnotes per page
\def\footnoterule{}
\def\footref{\advance\footnotenumber by 1\footnote{$^{\number\footnotenumber}$}}
\def\R#1{{\bf(#1)}}
%-----------------------------------------------------------------------
\def\ruleskip{\vskip 1.5pt}
%-----------------------------------------------------------------------
%-----------------------------------------------------------------------
% macros for marginal stuff
%-----------------------------------------------------------------------
%
\newbox\marginbox
\def\strutdepth{\dp\strutbox}
\def\strutheight{\ht\strutbox}
%
\def\marginstuff#1{\setbox\marginbox=\vtop{
    \hsize=60pt\tolerance=10000\hbadness=10000
%    \hsize=70pt\tolerance=10000\hbadness=10000
    \parindent=0pt\eightpoint#1\par}\strut}
\def\setmarginal{\strut\vadjust{\kern-\strutdepth\kern-\ht\marginbox\vtop{%
    \moveleft 70pt\hbox{\copy\marginbox}}\kern-\dp\marginbox\kern\strutdepth}}
%    \moveleft 80pt\hbox{\copy\marginbox}}\kern-\dp\marginbox\kern\strutdepth}}
%-----------------------------------------------------------------------
% useful macros
%-----------------------------------------------------------------------
%
\def\PSI{{\it PSI\/}}
\def\L2{\ifmmode L_2\else $L_2$\fi}
%
\def\item#1#2\par{\smallskip\noindent{\bf #1}---#2\par}
%
\def\deg{\ifmmode^\circ\else$^\circ$\fi}
\def\solar{\ifmmode{\rm M}_{\mathord\odot}\else${\rm M}_{\mathord\odot}$\fi}
\def\sol{\solar}
\def\inv{\ifmmode^{-1}\else$^{-1}$\fi}
\def\eg{{e.g.}}
\def\ie{{i.e.}}
\def\cf{{cf.}}
\def\etal{{\it et al.}}
\def\etc{etc.}
\def\sup#1{$^{\rm #1}$}
\def\mag{\sup{m}}
\def\,{\thinspace}
\def\lfil{\leaders\hbox to 0.4em{\hss.\hss}\hfil}
\def\lea{\mathrel{\raise .4ex\hbox{\rlap{$<$}\lower 1.2ex\hbox{$\sim$}}}}
\def\gea{\mathrel{\raise .4ex\hbox{\rlap{$>$}\lower 1.2ex\hbox{$\sim$}}}}
\let\lsim=\lea
\let\gsim=\gea
\def\simprop{\mathrel{\raise .4ex\hbox{\rlap{$\propto$}\lower 1.2ex\hbox{$\sim$}}}}
\def\subapp{_{\rm app}}
\def\submin{_{\rm min}}
\def\submax{_{\rm max}}
\def\arccot{\mathop{\rm arccot}\nolimits}
\def\hub{$H_0=100$ km s$^{-1}$ Mpc$^{-1}$, $q_0 = 0.5$}
\def\hubh{$H_0=100 h$ km s$^{-1}$ Mpc$^{-1}$, $q_0 = 0.5$}
\def\ra[#1 #2 #3.#4]{#1\sup{h}#2\sup{m}#3\sup{s}\llap.#4}
\def\dec[#1 #2 #3.#4]{#1\deg$#2'#3''$\llap.#4}
%
\def\fline#1/#2/{[#1\,{\sc #2}]}
%
\def\mas{\,mas}
\def\pa{P.A.}
\def\pdeg{\ifmmode $\setbox0=\hbox{$^{\circ}$}\rlap{\hskip.11\wd0 .}$^{\circ}
          \else \setbox0=\hbox{$^{\circ}$}\rlap{\hskip.11\wd0 .}$^{\circ}$\fi}
\def\arcs{\ifmmode {^{\scriptscriptstyle\prime\prime}}
          \else $^{\scriptscriptstyle\prime\prime}$\fi}
\def\arcm{\ifmmode {^{\scriptscriptstyle\prime}}
          \else $^{\scriptscriptstyle\prime}$\fi}
\newdimen\sa  \newdimen\sb
\def\parcs{\sa=.07em \sb=.03em
     \ifmmode \hbox{\rlap{.}}^{\scriptscriptstyle\prime\kern -\sb\prime}\hbox{\kern -\sa}
     \else \rlap{.}$^{\scriptscriptstyle\prime\kern -\sb\prime}$\kern -\sa\fi}
\def\parcm{\sa=.08em \sb=.03em
     \ifmmode \hbox{\rlap{.}\kern\sa}^{\scriptscriptstyle\prime}\hbox{\kern-\sb}
     \else \rlap{.}\kern\sa$^{\scriptscriptstyle\prime}$\kern-\sb\fi}
%
\def\s#1{{\sc #1}}
\def\et{\hbox{\it et al.}}
\def\ET{{\it ET AL.}}
\def\expo#1{\ifmmode \times 10^{#1}\else $\times 10^{#1}$\fi}
\def\page{\vfill\eject}
\def\leaderfil{\leaders\hbox to 5pt{\hss.\hss}\hfil}
%
\def\mo{\ifmmode ^{-1}\else $^{-1}$\fi}
\def\sup#1{$^{\rm #1}$}
\def\mag{\sup{m}}
\def\,{\thinspace}
\def\tc#1{3C\,#1}
\def\dots{\relax\ifmmode \ldots\else $\ldots$\fi}
\def\WHzsr{\ifmmode $W\,Hz\mo\,sr\mo$\else W\,Hz\mo\,sr\mo\fi}
\def\mKs{\ifmmode $\,mK\,s$^{-1/2}\else \,mK\,s$^{-1/2}$\fi}
\def\microns{\ifmmode \,\mu$m$\else \,$\mu$m\fi}
\def\micron{\microns}
\def\muK{\ifmmode \,\mu$K$\else \,$\mu$\hbox{K}\fi}
\def\microK{\ifmmode \,\mu$K$\else \,$\mu$\hbox{K}\fi}
\def\muW{\ifmmode \,\mu$W$\else \,$\mu$\hbox{W}\fi}
\def\dtt{\Delta T/T}
\def\rra[#1 #2]{#1\sup{h}#2\sup{m}}
\def\kms{\ifmmode $\,km\,s$^{-1}\else \,km\,s$^{-1}$\fi}
\def\kmsMpc{\ifmmode $\,\kms\,Mpc\mo$\else \,\kms\,Mpc\mo\fi}
\def\DeltaT{\ifmmode \Delta T\else $\Delta T$\fi}
\def\deltat{\ifmmode \Delta t\else $\Delta t$\fi}
\def\fknee{\ifmmode f_{\rm knee}\else $f_{\rm knee}$\fi}
\def\Fmax{\ifmmode F_{\rm max}\else $F_{\rm max}$\fi}
%
\def\Bullet{\item{$\bullet$}}
\newdimen\bul \newdimen\dashn \newdimen\buln
\setbox0=\hbox{$\bullet$~}
\bul=\wd0 \hangindent=3em \dashn=\bul  \advance\dashn by \wd0
\def\bull#1\par{\smallskip\noindent{\hangindent=\bul\hangafter=1
    {$\bullet$~}#1\par}}
\def\dash#1\par{\smallskip\noindent{\hangindent=\dashn\hangafter=1
    \hglue\bul{--~}#1\par}}
\buln=\wd0
\def\bulln#1\par{\advance\bullnumber by 1 \hangindent=\buln \hangafter=0
            \smallskip\noindent\llap{\number\bullnumber)~}#1\par}
%
\def\gsim{\lower.6ex\hbox{$\buildrel>\over\sim$}}
\def\lsim{\lower.6ex\hbox{$\buildrel<\over\sim$}}
\def\ra[#1 #2 #3.#4]{#1\sup{h}#2\sup{m}#3\sup{s}\llap.#4}
\def\dec[#1 #2 #3.#4]{#1\deg#2\arcm#3\arcs\llap.#4}
\def\deco[#1 #2 #3]{#1\deg#2\arcm#3\arcs}
%----------------------------------------------------------------------
\def\refindent{\hangindent=1em \hangafter=1 \noindent}
\def\ref #1;#2;#3;#4\par{\refindent #1,{\it #2},{\bf #3},#4.\par}
\def\ajref #1;#2;#3;#4\par{\refindent #1. #2 {\bf #3},#4.\par}
\def\vref #1\par{\refindent #1.\par}
\def\cref #1;#2;#3;#4;#5\par{\refindent #1, in{\it #2}, 
                             \hbox{ed.}#3 #4, \hbox{p.} #5.\par}
\def\bref #1;#2;#3\par{\refindent #1, {\it #2} (#3).\par} 
%
\def\tjpref#1{{\par\noindent \hangindent=1em \hangafter=1 #1\par}}
\def\T#1{}
%\def\T#1{``#1'',} % for use in bibliography
% 
% Regular journals, IAU style.  Usage: e.g., \MN185,207.
% 
\def\AandA#1,#2.{{\it Astron. Astrophys.}, {\bf #1}, #2.}
\def\AandAS#1,#2.{{\it Astron. Astrophys. Suppl.}, {\bf #1}, #2.}
\def\AJ#1,#2.{{\it Astron. J.}, {\bf #1}, #2.}
\def\AN#1,#2.{{\it Astron. Nachr.}, {\bf #1}, #2.}
\def\APJ#1,#2.{{\it Astrophys. J.}, {\bf #1}, #2.}
\def\APJS#1,#2.{{\it Astrophys. J. Suppl.}, {\bf #1}, #2.}
\def\APJL#1,#2.{{\it Astrophys. J. (Letters)}, {\bf #1}, #2.}
\def\APLETT#1,#2.{{\it Astrophys. Letters}, {\bf #1}, #2.}
\def\APPLOPT#1,#2.{{\it Appl. Optics}, {\bf #1}, #2.}
\def\ARAA#1,#2.{{\it Ann. Rev. Astron. Astrophys.}, {\bf #1}, #2.}
\def\ASS#1,#2.{{\it Astrophys. Space Sci.}, {\bf #1}, #2.}
\def\BAAS#1,#2.{{\it Bull. Am. Astron. Soc.}, {\bf #1}, #2.}
\def\CJP#1,#2.{{\it Can. J. Phys.}, {\bf #1}, #2.}
\def\IEEE#1,#2.{{\it Proc. Inst. Elec. Electron. Engrs.}, {\bf #1}, #2.}
\def\JAA#1,#2.{{\it J. Astrophys. Astron.}, {\bf #1}, #2.}
\def\JGRES#1,#2.{{\it J. Geophys. Res.}, {\bf #1}, #2.}
\def\JHA#1,#2.{{\it J. Hist. Astron.}, {\bf #1}, #2.}
\def\JOSA#1,#2.{{\it J. Opt. Soc. Am.}, {\bf #1}, #2.}
\def\JGR#1,#2.{{\it J. Geophys. Res.}, {\bf #1}, #2.}
\def\MAG#1,#2.{{\it Mitt. Astron. Ges.}, {\bf #1}, #2.}
\def\MEMRAS#1,#2.{{\it Mem. Roy. Astron. Soc.}, {\bf #1}, #2.}
\def\MN#1,#2.{{\it Monthly Notices Roy. Astron. Soc.}, {\bf #1}, #2.}
\def\NAT#1,#2.{{\it Nature}, {\bf #1}, #2.}
\def\OBS#1,#2.{{\it Observatory}, {\bf #1}, #2.}
\def\PASJ#1,#2.{{\it Publ. Astron. Soc. Japan}, {\bf #1}, #2.}
\def\PASP#1,#2.{{\it Publ. Astron. Soc. Pacific}, {\bf #1}, #2.}
\def\PhysRevA#1,#2.{{\it Phys. Rev.~A}, {\bf #1}, #2.}
\def\PHILMAG#1,#2.{{\it Phil. Mag.}, {\bf #1}, #2.}
\def\QJ#1,#2.{{\it Quart. J. Roy. Astron. Soc.}, {\bf #1}, #2.}
\def\RADIOSCI#1,#2.{{\it Radio Sci.}, {\bf #1}, #2.}
\def\SCI#1,#2.{{\it Science}, {\bf #1}, #2.}
\def\SCIAMER#1,#2.{{\it Sci. Am.}, {\bf #1}, #2.}
\def\SPIE#1,#2.{{\it Proc. Soc. Photo-Opt. Instrum. Engrs.}, {\bf #1}, #2.}
\def\RevModPhys#1,#2.{{\it Rev. Mod. Phys.}, {\bf #1}, #2.}
\def\CelMech#1,#2.{{\it Celes. Mech.}, {\bf #1}, #2.}
\def\SovAst#1,#2.{{\it Soviet Astron.}, {\bf #1}, #2.}
\def\SovAstL#1,#2.{{\it Soviet Astron. Lett.}, {\bf #1}, #2.}
\def\AstZh#1,#2.{{\it Astron. Zh.}, {\bf #1}, #2.}
\def\Pisma#1,#2.{{\it Pis'ma Astron. Zh.}, {\bf #1}, #2.}
%
%-----------------------------------------------------------------
% Encapsulated PostScript figures (EPSF.tex)
%-----------------------------------------------------------------
\newread\epsffilein    % file to \read
\newif\ifepsffileok    % continue looking for the bounding box?
\newif\ifepsfbbfound   % success?
\newif\ifepsfverbose   % report what you're making?
\newdimen\epsfxsize    % horizontal size after scaling
\newdimen\epsfysize    % vertical size after scaling
\newdimen\epsftsize    % horizontal size before scaling
\newdimen\epsfrsize    % vertical size before scaling
\newdimen\epsftmp      % register for arithmetic manipulation
\newdimen\pspoints     % conversion factor
%
\pspoints=1bp          % Adobe points are `big'
\epsfxsize=0pt         % Default value, means `use natural size'
\epsfysize=0pt         % ditto
%
\def\epsfbox#1{\global\def\epsfllx{72}\global\def\epsflly{72}%
   \global\def\epsfurx{540}\global\def\epsfury{720}%
   \def\lbracket{[}\def\testit{#1}\ifx\testit\lbracket
   \let\next=\epsfgetlitbb\else\let\next=\epsfnormal\fi\next{#1}}%
%
\def\epsfgetlitbb#1#2 #3 #4 #5]#6{\epsfgrab #2 #3 #4 #5 .\\%
   \epsfsetgraph{#6}}%
%
\def\epsfnormal#1{\epsfgetbb{#1}\epsfsetgraph{#1}}%
%
\def\epsfgetbb#1{%
\openin\epsffilein=#1
\ifeof\epsffilein\errmessage{I couldn't open #1, will ignore it}\else
   {\epsffileoktrue \chardef\other=12
    \def\do##1{\catcode`##1=\other}\dospecials \catcode`\ =10
    \loop
       \read\epsffilein to \epsffileline
       \ifeof\epsffilein\epsffileokfalse\else
          \expandafter\epsfaux\epsffileline:. \\%
       \fi
   \ifepsffileok\repeat
   \ifepsfbbfound\else
    \ifepsfverbose\message{No bounding box comment in #1; using defaults}\fi\fi
   }\closein\epsffilein\fi}%
%
\def\epsfsetgraph#1{%
   \epsfrsize=\epsfury\pspoints
   \advance\epsfrsize by-\epsflly\pspoints
   \epsftsize=\epsfurx\pspoints
   \advance\epsftsize by-\epsfllx\pspoints
%
   \epsfsize\epsftsize\epsfrsize
   \ifnum\epsfxsize=0 \ifnum\epsfysize=0
      \epsfxsize=\epsftsize \epsfysize=\epsfrsize
%
     \else\epsftmp=\epsftsize \divide\epsftmp\epsfrsize
       \epsfxsize=\epsfysize \multiply\epsfxsize\epsftmp
       \multiply\epsftmp\epsfrsize \advance\epsftsize-\epsftmp
       \epsftmp=\epsfysize
       \loop \advance\epsftsize\epsftsize \divide\epsftmp 2
       \ifnum\epsftmp>0
          \ifnum\epsftsize<\epsfrsize\else
             \advance\epsftsize-\epsfrsize \advance\epsfxsize\epsftmp \fi
       \repeat
     \fi
   \else\epsftmp=\epsfrsize \divide\epsftmp\epsftsize
     \epsfysize=\epsfxsize \multiply\epsfysize\epsftmp   
     \multiply\epsftmp\epsftsize \advance\epsfrsize-\epsftmp
     \epsftmp=\epsfxsize
     \loop \advance\epsfrsize\epsfrsize \divide\epsftmp 2
     \ifnum\epsftmp>0
        \ifnum\epsfrsize<\epsftsize\else
           \advance\epsfrsize-\epsftsize \advance\epsfysize\epsftmp \fi
     \repeat     
   \fi
%
   \ifepsfverbose\message{#1: width=\the\epsfxsize, height=\the\epsfysize}\fi
   \epsftmp=10\epsfxsize \divide\epsftmp\pspoints
   \vbox to\epsfysize{\vfil\hbox to\epsfxsize{%
      \special{illustration #1 scaled \number\epsfscale}%% Textures
      \hfil}}%
\epsfxsize=0pt\epsfysize=0pt\epsfscale=1000 }%

%
{\catcode`\%=12 \global\let\epsfpercent=%\global\def\epsfbblit{%BoundingBox}}%
%
\long\def\epsfaux#1#2:#3\\{\ifx#1\epsfpercent
   \def\testit{#2}\ifx\testit\epsfbblit
      \epsfgrab #3 . . . \\%
      \epsffileokfalse
      \global\epsfbbfoundtrue
   \fi\else\ifx#1\par\else\epsffileokfalse\fi\fi}%
%
\def\epsfgrab #1 #2 #3 #4 #5\\{%
   \global\def\epsfllx{#1}\ifx\epsfllx\empty
      \epsfgrab #2 #3 #4 #5 .\\\else
   \global\def\epsflly{#2}%
   \global\def\epsfurx{#3}\global\def\epsfury{#4}\fi}%
%
\let\epsffile=\epsfbox
%%
\newcount\epsfscale    % computed scaling factor
\newdimen\epsftmpp     % register for arithmetic manipulation
\newdimen\epsftmppp    % register for arithmetic manipulation
\newdimen\epsfM        % scale=1000 means natural size (cf. TeX's \mag)
\newdimen\sppoints     % compute in units of \sppoints
%
\epsfscale=1000        % default value
\sppoints=1000sp       % TeX's unit is a scaled point
\epsfM=1000\sppoints
%
\def\computescale#1#2{%
  \epsftmpp=#1 \epsftmppp=#2
  \epsftmp=\epsftmpp \divide\epsftmp\epsftmppp  % p=[a/b]
  \epsfscale=\epsfM \multiply\epsfscale\epsftmp % s=c p
  \multiply\epsftmp\epsftmppp                   % p b
  \advance\epsftmpp-\epsftmp                    % q=a-p b
  \epsftmp=\epsfM                               % c=1000
  \loop \advance\epsftmpp\epsftmpp              % q'=2q
    \divide\epsftmp 2                           % c'=c/2
    \ifnum\epsftmp>0
      \ifnum\epsftmpp<\epsftmppp\else           % q' >= b  =>
        \advance\epsftmpp-\epsftmppp            %   q'=2q-b
        \advance\epsfscale\epsftmp \fi          %   s=s+[c/2]
  \repeat
  \divide\epsfscale\sppoints}
%
\def\epsfsize#1#2{%
  \ifnum\epsfscale=1000
    \ifnum\epsfxsize=0
      \ifnum\epsfysize=0
      \else \computescale{\epsfysize}{#2}% returns \epsfscale
      \fi
    \else \computescale{\epsfxsize}{#1}% returns \epsfscale
    \fi
  \else
    \epsfxsize=#1
    \divide\epsfxsize by 1000 \multiply\epsfxsize by \epsfscale
  \fi}

