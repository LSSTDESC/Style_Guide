\documentclass[letterpaper,11pt]{article}

\usepackage{multirow}
\usepackage{datetime2}
\newcommand*{\version}{Version \today}

%%% NB: Because hyperref can cause a fatal and obscure crash when a link ends up spanning a page break, we will use the "draft" option until the last possible minute.

\usepackage{color}
\definecolor{citecol}{rgb}{0.0,0.0,0.5}
\usepackage[draft,breaklinks=true,colorlinks=true,citecolor=citecol,linkcolor=citecol]{hyperref}
\usepackage{graphicx}
\usepackage{natbib}
\usepackage{multicol}

%%%%%%%%%%%%%%%%%%%%%%%%%%%
%%%%% Page dimensions %%%%%
%%%%%%%%%%%%%%%%%%%%%%%%%%%

\setlength{\textwidth}{7in} 
\setlength{\textheight}{9.75in}
\setlength{\topmargin}{-0.5in}%{-0.0625in} 
\setlength{\oddsidemargin}{-0.25in}
\setlength{\evensidemargin}{-0.25in} 
\setlength{\headheight}{0in}
\setlength{\headsep}{0in} 
\setlength{\hoffset}{0in}
\setlength{\voffset}{0in}

% customize captions
\usepackage[margin=10pt,font=small,labelfont=bf]{caption}
\setlength{\abovecaptionskip}{5pt}
\setlength{\belowcaptionskip}{5pt}

% % customize headers/footers
% \usepackage{fancyhdr}
% \fancyhead{}                % clear all header fields
% \fancyhead[RO]{\rightmark}  % section
% \fancyhead[LE]{\leftmark}   % chapter
% \fancyfoot{}                % clear all footer fields
% \fancyfoot[CE,CO]{\thepage} % page number
% \renewcommand{\headrulewidth}{0.4pt}

% single column (non-float) figures
\newenvironment{Figure}
  {\par\medskip\noindent\minipage{\linewidth}}
  {\endminipage\par\medskip}

% single column (non-float) tables (identical to Figure)
\newenvironment{Table}
  {\par\medskip\noindent\minipage{\linewidth}}
  {\endminipage\par\medskip}

% for compatibility in both report and article versions
\newcommand*{\mytitle}[2][]{\title{#2}}

%\input{user_macros}

\hypersetup{draft=false}
\setlength{\textheight}{9.25in}
\setlength{\headsep}{0.25in} 
\setlength{\topmargin}{-0.75in}

\usepackage{fancyhdr}
\renewcommand{\headrulewidth}{0pt}
\fancyhead[L]{}
\fancyhead[R]{
  \centering{\includegraphics[width=5.25in,angle=0]{figures/header}}
}
\fancyfoot{}
\pagestyle{fancy}
\renewcommand{\sectionmark}[1]{\markright{\thesection\ #1}}

%Planck style file, to be used with A&A style to produce Planck papers for publication.
%
% version 28 September 2010 --- useful macros --- CRL
% version 17 October 2010   --- first cut at important instrument values, from Daniele Mennella and
%                               Francois Bouchet, 13 October 2010 --- CRL
% version 18 October 2010   --- LFI FWHM changed to one value per feed, rather than M & S separately
%                               LFI FWHM uncertainties added for individual feeds.  Corrections made
%                               to LFI values. --- Andrea Zacchei
% version 24 October 2010   --- added to and corrected definitions.  No changes made to instrument
%                               quantities. --- CRL 
% version 31 October 2010   --- added definition of \muKHz. --- CRL
%
% version 15 November 2010  --- fixed conflict with aa.cls in definition of \endtable
%                               by naming the command below "\endPlancktable".  See section
%                               13.16 of the Style Guide.
%
% version 06 December 2010  --- Set up names with and without units.
%                               Add \allearlypapers command to ensure that all early papers are
%                               included in the reference list.
%                               Define macro for the name of the 4He JT cooler.
%
% version 07 December 2010  --- removed extraneous "planck2011-1.2" entry in \allearlypapers
%
% version 12 December 2010  --- added \endPlancktablewide command to set tablenotes to the full
%                               page width in the \begin{table*}...\end{table*} environment when
%                               the ``twocolumn'' option is specified in the \documentclass command.
%                               (It would be more elegant to extract the appropriate width from the
%                               aa.cls system at the time of execution, but that is buried more
%                               deeply in the system than I investigated.)
%
% version 05 January 2011   --- added unit \MJysr.  HFI performance values updated per FRB email
%                               01/05/2011 02:38-0800, and Brendan Crill email 01/05/2011 18:08 -0800.
%
% version 06 January 2011   --- changed \scriptscriptstyle primes to \scriptstyle, to better match the
%                               tx fonts used by A&A.
%
% version 07 January 2011   --- modified \allearlypapers to correspond with final early paper list.  
%                               Fixed 545 GHz center frequency.
%
% version 07 January 2011b  --- changed LFI white-noise sensitivity numbers to correct problem with units
%
% version 05 July 2011      --- added \Msol and \Lsol to get the symbols for solar mass and luminosity.
%                               Deleted previous definitions of \solar and \sol, which were equivalent
%                               to the new \Msol.
%
% version 16 August 2011    --- changed comments on \endPlancktable and \endPlancktablewide for clarity
%
% version 11 September 2011 --- changed definition of \tablenote to make footnote labels italic, as per A\&A
%
% version 26 April 2011     --- changed definition of \Planck to agree with what is said in the Style Guide (!)
%
% version 04 Dec 2013       --- included 2013 results references
%
% version 17 Jan 2014       --- included fix to bibtex file v4.3, i.e. \providecommand{\sorthelp}[1]{}
%
% version 26 Jul 2014       --- fixed incompatibility problem with aa.cls v8.0 and v8.2.  v8.2 should now be used
%                               for all Planck papers.
%                           --- fixed problem in definition of "\all2013resultspapers" that introduced a blanck
%                               into the reference to p06b.
%                           --- removed all the parameter definition stuff at the end.  We weren't using it, and
%                               it took up a lot of space.
%
% version 28 Jan 2015       --- added "\alltwentyfiftennresultspapers" and corrected "\all2013resultspapers" to
%                               "\all20thirteenresultspapers",
%
% Usage:  after the \documentclass[traditabstract]{aa} command in the La\TeX\ input file,
%         add this command:      \input Planck.tex


\def\setsymbol#1#2{\expandafter\def\csname #1\endcsname{#2}}
\def\getsymbol#1{\csname #1\endcsname}

%-----------------------------------------------------------------------
% Planck
%-----------------------------------------------------------------------
\def\Planck{\textit{Planck}}

%-----------------------------------------------------------------------
% The Planck Helium-4 JT cooler
%-----------------------------------------------------------------------
\def\HeJT{$^4$He-JT}

%-----------------------------------------------------------------------
% To include all Planck Early Results papers in the reference lists
%-----------------------------------------------------------------------
\def\allearlypapers{\nocite{planck2011-1.1, planck2011-1.3, planck2011-1.4, planck2011-1.5, planck2011-1.6, planck2011-1.7, planck2011-1.10, planck2011-1.10sup, planck2011-5.1a, planck2011-5.1b, planck2011-5.2a, planck2011-5.2b, planck2011-5.2c, planck2011-6.1, planck2011-6.2, planck2011-6.3a, planck2011-6.4a, planck2011-6.4b, planck2011-6.6, planck2011-7.0, planck2011-7.2, planck2011-7.3, planck2011-7.7a, planck2011-7.7b, planck2011-7.12, planck2011-7.13}}

%-----------------------------------------------------------------------
% To include all Planck 2013 Results papers in the reference lists
%-----------------------------------------------------------------------
\def\alltwentythirteenresultspapers{\nocite{planck2013-p01, planck2013-p02, planck2013-p02a, planck2013-p02d, planck2013-p02b, planck2013-p03, planck2013-p03c, planck2013-p03f, planck2013-p03d, planck2013-p03e, planck2013-p01a, planck2013-p06, planck2013-p03a, planck2013-pip88, planck2013-p08, planck2013-p11, planck2013-p12, planck2013-p13, planck2013-p14, planck2013-p15, planck2013-p05b, planck2013-p17, planck2013-p09, planck2013-p09a, planck2013-p20, planck2013-p19, planck2013-pipaberration, planck2013-p05, planck2013-p05a, planck2013-pip56, planck2013-p06b, planck2013-p01a}}

%-----------------------------------------------------------------------
% To include all Planck 2015 Results papers in the reference lists
%-----------------------------------------------------------------------
\def\alltwentyfifteenresultspapers{\nocite{planck2014-a01, planck2014-a03, planck2014-a04, planck2014-a05, planck2014-a06, planck2014-a07, planck2014-a08, planck2014-a09, planck2014-a11, planck2014-a12, planck2014-a13, planck2014-a14, planck2014-a15, planck2014-a16, planck2014-a17, planck2014-a18, planck2014-a19, planck2014-a20, planck2014-a22, planck2014-a24, planck2014-a26, planck2014-a28, planck2014-a29, planck2014-a30, planck2014-a31, planck2014-a35, planck2014-a36, planck2014-a37, planck2014-ES}}

%-----------------------------------------------------------------------
% Tables
%-----------------------------------------------------------------------
\newbox\tablebox    \newdimen\tablewidth
\def\leaderfil{\leaders\hbox to 5pt{\hss.\hss}\hfil}
%
% use the following definition of \endPlancktable for ApJ style notes to tables, set to the 
%         width of the table
% \def\endPlancktable{\tablewidth=\wd\tablebox 
%
% use the following definitions of \endPlancktable and \endPlancktablewide for A&A style notes 
% set to one-column  or full-page width, respectively
\def\endPlancktable{\tablewidth=\columnwidth 
    $$\hss\copy\tablebox\hss$$
    \vskip-\lastskip\vskip -2pt}
\def\endPlancktablewide{\tablewidth=\textwidth 
    $$\hss\copy\tablebox\hss$$
    \vskip-\lastskip\vskip -2pt}
\def\tablenote#1 #2\par{\begingroup \parindent=0.8em
    \abovedisplayshortskip=0pt\belowdisplayshortskip=0pt
    \noindent
    $$\hss\vbox{\hsize\tablewidth \hangindent=\parindent \hangafter=1 \noindent
    \hbox to \parindent{$^#1$\hss}\strut#2\strut\par}\hss$$
    \endgroup}
\def\doubleline{\vskip 3pt\hrule \vskip 1.5pt \hrule \vskip 5pt}

%-----------------------------------------------------------------------
% useful macros
%-----------------------------------------------------------------------
%
\def\L2{\ifmmode L_2\else $L_2$\fi}
%
\def\dtt{\Delta T/T}
\def\DeltaT{\ifmmode \Delta T\else $\Delta T$\fi}
\def\deltat{\ifmmode \Delta t\else $\Delta t$\fi}
\def\fknee{\ifmmode f_{\rm knee}\else $f_{\rm knee}$\fi}
\def\Fmax{\ifmmode F_{\rm max}\else $F_{\rm max}$\fi}
%
\def\solar{\ifmmode{\rm M}_{\mathord\odot}\else${\rm M}_{\mathord\odot}$\fi}
\def\Msolar{\ifmmode{\rm M}_{\mathord\odot}\else${\rm M}_{\mathord\odot}$\fi}
\def\Lsolar{\ifmmode{\rm L}_{\mathord\odot}\else${\rm L}_{\mathord\odot}$\fi}
%
\def\inv{\ifmmode^{-1}\else$^{-1}$\fi}
\def\mo{\ifmmode^{-1}\else$^{-1}$\fi}
\def\sup#1{\ifmmode ^{\rm #1}\else $^{\rm #1}$\fi}
\def\expo#1{\ifmmode \times 10^{#1}\else $\times 10^{#1}$\fi}
%
\def\,{\thinspace}
\def\lsim{\mathrel{\raise .4ex\hbox{\rlap{$<$}\lower 1.2ex\hbox{$\sim$}}}}
\def\gsim{\mathrel{\raise .4ex\hbox{\rlap{$>$}\lower 1.2ex\hbox{$\sim$}}}}
\let\lea=\lsim
\let\gea=\gsim
\def\simprop{\mathrel{\raise .4ex\hbox{\rlap{$\propto$}\lower 1.2ex\hbox{$\sim$}}}}
%
\def\deg{\ifmmode^\circ\else$^\circ$\fi}
\def\pdeg{\ifmmode $\setbox0=\hbox{$^{\circ}$}\rlap{\hskip.11\wd0 .}$^{\circ}
          \else \setbox0=\hbox{$^{\circ}$}\rlap{\hskip.11\wd0 .}$^{\circ}$\fi}
\def\arcs{\ifmmode {^{\scriptstyle\prime\prime}}
          \else $^{\scriptstyle\prime\prime}$\fi}
\def\arcm{\ifmmode {^{\scriptstyle\prime}}
          \else $^{\scriptstyle\prime}$\fi}
\newdimen\sa  \newdimen\sb
\def\parcs{\sa=.07em \sb=.03em
     \ifmmode \hbox{\rlap{.}}^{\scriptstyle\prime\kern -\sb\prime}\hbox{\kern -\sa}
     \else \rlap{.}$^{\scriptstyle\prime\kern -\sb\prime}$\kern -\sa\fi}
\def\parcm{\sa=.08em \sb=.03em
     \ifmmode \hbox{\rlap{.}\kern\sa}^{\scriptstyle\prime}\hbox{\kern-\sb}
     \else \rlap{.}\kern\sa$^{\scriptstyle\prime}$\kern-\sb\fi}
%
\def\ra[#1 #2 #3.#4]{#1\sup{h}#2\sup{m}#3\sup{s}\llap.#4}
\def\dec[#1 #2 #3.#4]{#1\deg#2\arcm#3\arcs\llap.#4}
\def\deco[#1 #2 #3]{#1\deg#2\arcm#3\arcs}
\def\rra[#1 #2]{#1\sup{h}#2\sup{m}}
%
\def\page{\vfill\eject}
\def\dots{\relax\ifmmode \ldots\else $\ldots$\fi}
%
%-----------------------------------------------------------------------
% units
%-----------------------------------------------------------------------
%
\def\WHzsr{\ifmmode $W\,Hz\mo\,sr\mo$\else W\,Hz\mo\,sr\mo\fi}
\def\mHz{\ifmmode $\,mHz$\else \,mHz\fi}
\def\GHz{\ifmmode $\,GHz$\else \,GHz\fi}
\def\mKs{\ifmmode $\,mK\,s$^{1/2}\else \,mK\,s$^{1/2}$\fi}
\def\muKs{\ifmmode \,\mu$K\,s$^{1/2}\else \,$\mu$K\,s$^{1/2}$\fi}
\def\muKRJs{\ifmmode \,\mu$K$_{\rm RJ}$\,s$^{1/2}\else \,$\mu$K$_{\rm RJ}$\,s$^{1/2}$\fi}
\def\muKHz{\ifmmode \,\mu$K\,Hz$^{-1/2}\else \,$\mu$K\,Hz$^{-1/2}$\fi}
\def\MJysr{\ifmmode \,$MJy\,sr\mo$\else \,MJy\,sr\mo\fi}
\def\MJysrmK{\ifmmode \,$MJy\,sr\mo$\,mK$_{\rm CMB}\mo\else \,MJy\,sr\mo\,mK$_{\rm CMB}\mo$\fi}
\def\microns{\ifmmode \,\mu$m$\else \,$\mu$m\fi}
\def\micron{\microns}
\def\muK{\ifmmode \,\mu$K$\else \,$\mu$\hbox{K}\fi}
\def\microK{\ifmmode \,\mu$K$\else \,$\mu$\hbox{K}\fi}
\def\muW{\ifmmode \,\mu$W$\else \,$\mu$\hbox{W}\fi}
\def\kms{\ifmmode $\,km\,s$^{-1}\else \,km\,s$^{-1}$\fi}
\def\kmsMpc{\ifmmode $\,\kms\,Mpc\mo$\else \,\kms\,Mpc\mo\fi}
%
%
%----------------------------------------------------------------------
% set up machinery to list Planck papers in roman numeral order.
%----------------------------------------------------------------------

%\providecommand{\sorthelp}[1]{}



\begin{document}

\newcommand*{\suphysics}{{\it Department of Physics, Stanford University, 382 Via Pueblo Mall, Stanford, CA 94305, USA}}
\newcommand{\icme}{{\it Institute for Computational and Mathematical Engineering, 475 Via Ortega, Stanford, CA 94305, USA}}

\title{\vspace{5cm}LSST DESC Style Guide}
\author{DESC Publications Board: Seth Digel$^1$ (Publication Manager), Pierre Astier,$^2$ David Kirkby,$^3$\\ Rachel Mandelbaum,$^4$ Adam Mantz,$^5$, Phil Marshall,$^6$ Hiranya Peiris,$^7$ and Michael Wood-Vasey$^8$
  \medskip\\
    {\small$^1$}\\
    {\small$^2$}\\
    {\small$^3$}\\
    {\small$^4$}\\
    {\small$^5$\suphysics}\\
    {\small$^6$}\\
    {\small$^7$}\\
    {\small$^8$}
}
\date{\version}
\maketitle
\thispagestyle{fancy}

\clearpage
\fancyhead{}
%\fancyfoot[C]{\thepage}
\fancyhead[L]{\version}
\fancyhead[R]{\thepage}
\pagenumbering{roman}
\setcounter{page}{1}

\tableofcontents

\clearpage
\fancyhead[C]{\rightmark}
\pagenumbering{arabic}
\setcounter{page}{1}

\section{PURPOSE}

This Style Guide is intended to help authors of \Planck-related papers prepare
high-quality manuscripts in a uniform style. It supplements the instructions
to authors provided by the  Astronomy \& Astrophysics Author's guide 

\url{http://www.aanda.org/doc_journal/instructions/aa_instructions.pdf}.

\noindent A\&A also has a useful English usage guide, available at 

\url{http://www.aanda.org/doc_journal/instructions/aa_english_guide.pdf}.

\noindent Both can be found under ``Author Information'' at
\url{http://www.aanda.org/}.


\section{\TeX\ STUFF}

The A\&A format is implemented in the A\&A document class file |aa.cls|,
available from 

\url{http://www.aanda.org/}

\noindent and from the \Planck\ SVN at

\url{https://scisvn01.esac.esa.int/Planck_Publication_Management}

\noindent In the 2013 papers, we used \verb|aa.cls v7.0|.  A\&A is now up to
\verb|v8.2|, although \verb|v7.0| is still allowed by A\&A.  Most of the differences between \verb|v7.0| and \verb|v8.2|
need no comment here, but there are three compatibility issues of note.

The ``structured'' abstract that was the default in \verb|v7.0| no longer
exists in \verb|v8.0|, eliminating the need for the \verb|[traditabstract]| option in
the \verb|\document{aa}| command at the beginning of the La\TeX\ file.  

The 2013 version of the \verb|Planck.tex| file (see below) is not compatible
with \verb|aa.cls v8.2|.  Make sure to use the current version of \verb|Planck.tex|,
always available at

\url{https://scisvn01.esac.esa.int/Planck_Publication_Management}

Up until September 2014, \Planck\ author lists were generated
with ``\verb|\\ \and|'' between institution names.  With \verb|aa.cls v7.0|, the
resulting double space between institution names at the end of \Planck\ papers
could be fixed by putting \verb|\raggedright| immediately before \verb|\end{document}|.
This kludge no longer works with \verb|aa.cls v8.2|.  Author lists are now generated
with ``\verb|\goodbreak \and|'' between institutions.  No \verb|\raggedright| is
required.  This is still a kludge to deal with a problem that A\&A should fix
in \verb|aa.cls|, but it works until they do.  

\medskip

Unfortunately, \verb|aa.cls v8.2| puts
long lists of institutions (the kind \Planck\ papers always have!) immediately
following the references, and before the appendices (if any).  This is weird,
to put it politely, and we are trying to get it changed.  {\bf Until we do, however, it is best to use aa.cls v7.0.  (Specify ``traditabstract,'' of course.
The institution list fix is backwards compatible.)}


\subsection{Planck.tex}

La\TeX\ commands useful and specific to \Planck\ papers are found in
\verb|Planck.tex|, available on the PPM web page.  To use, insert the line
\verb|\input Planck.tex| after the initial \verb|\documentclass| command in your input
file.

There are two important changes in \verb|Planck.tex| in 2015.  First,
as mentioned above, a command defined in earlier versions conflicted with
\verb|aa.cls| versions 8.0 and later.  That command has been removed.  If you use
\verb|Planck.tex| from 2013 with these later versions of \verb|aa.cls|, you'll get an
immediate and unhelpful error message.  Second, the \verb|\setsymbol{|$\ldots$\verb|}|
definition machinery to specify instrumental and other parameters proved
unworkable.  All of the \verb|\setsymbol| commands have been removed.  Attempts to
use this  machinery will fail.


\subsection{A fix for an annoying La\TeX\ problem}

Sometimes La\TeX\ puts figures in the wrong order.  For example,
single-column Fig.\,10 may appear before double-column Fig.\,9.  The solution
is to add

\verb|\usepackage{fixltx2e}|

\smallskip
\noindent a ``La\TeX\ bug fix package'' that usually fixes this problem and a
few others.

\subsection{Active links}

To include active links in the output .pdf file, include 
\verb|\usepackage{hyperref}| at the beginning of the La\TeX\ file and use
\verb|\url{}|, e.g., \verb|\footnote{\url{http://www.asdc.asi.it/fermibsl/}}|.

Occasionally La\TeX\ will complain because of a hyperlink being split across
a page break --- in such cases, then simplest solution is to find the
offending link and enclose it within \verb|\mbox{}|.


\section{HOW TO REFER TO PLANCK AND OTHER PROJECTS}

Refer to ``the \Planck\ project,''  ``the \Planck\ spacecraft,'' or ``\Planck.''
The name should be italicized.
This can be done in all font environments (e.g., normal text, bold titles, or
section headings) with \verb|\textit{Planck}|.  \verb|\Planck| is so defined in
\verb|Planck.tex| for convenience.  Additionally ``Planck'' is not in italics in
references that include the phrase ``Planck Collaboration.'' 

By the same logic ``Planck'' is not italicized in other proper noun phrases,
such as ``Planck Catalogue of Compact Sources.'' Italics are never used for
``Planck constant'' or ``Max Planck.''


\subsection{Other experiments}

For the 2013 papers, we requested that all spacecraft names be
italicized.  This conflicts with the A\&A house style, which we will
adopt for the 2015 papers (irrespective of whether we all agree that it
makes sense!).  In this scheme, an instrument name is {\it not\/}
italicized unless it is named after a person, so it is
``{\it Chandra},'' ``{\it Fermi},'' ``{\it Herschel},'' and ``{\it Spitzer},''
but ``Gaia,'' ``GALEX,'' ``IRAS,'' and ``ISO.''  The names of instruments and
non-satellite experiments remain in roman font, e.g., ``ACT,'' ``DIRBE,''
``HFI,'' ``LFI,'' ``SPIRE,'' and ``SPT.''  Note that the convention means that
you should write
``{\it Hubble\/} Space Telescope,'' 
``{\it Wilkinson\/} Microwave Anisotropy Probe,''
and ``XMM-{\it Newton},''
but just ``HST,'' ``WMAP,'' and ``XMM.''


\subsection{The possessive form}

The possessive version of \Planck\ is ``\Planck's,'' written |\Planck's|.
Note that the apostrophe and the ``s'' are roman, and that the italic
correction is built into the definition of |\Planck|, so you don't have to
worry about spacing.


\section{HOW TO REFER TO PLANCK SPECIFICS}

Use ``Planck Early Release Compact Source Catalogue (ERCSC)'' at the first
reference to it in the text, and ``ERCSC'' thereafter.  The ERCSC contains the
Early Sunyaev-Zeldovich (ESZ) catalogue and the Early Cold Cores (ECC)
catalogue, which should be referred to similarly.  Don't worry that the
expression ``the ESZ'' seems to be missing a noun.

Similarly, use ``Planck Catalogue of Compact Sources (PCCS)'' and
``Planck Cluster Catalogue (PCC)'' for the catalogues released in
March 2013.  Names for the second and likely final versions of the catalogues
to be released in 2015 are not yet settled.  When they are, they'll be added
here.

The  \Planck\ cryocoolers should be referred to as the
``sorption cooler,'' the ``\HeJT\ cooler,'' and the ``dilution cooler.''
The command \verb|\HeJT|, defined in \verb|Planck.tex|, produces \HeJT\ for convenience. 

Individual \Planck\ surveys are precisely defined subsets of the data and
should be referred to using capital letters, i.e., ``Survey~1,'' ``Survey~2,''
etc.  Similarly for ``Year~1,'' etc.

\section{DATES}

A\&A is flexible about the format for dates, ``as long as you remain
consistent.''
A\&A itself writes {\it Received\/} and {\it Accepted\/} dates in
``day month year'' format, e.g., 11 January 2011.  We will adopt that as the
standard for dates in the \Planck\ papers.

There are two exceptions.  The A\&A ``Guide to the English Editing''
(see Sect.~1 above) says ``When the
date is an integral part of the event's name, the use of the IAU format
is recommended but not mandatory (for example, ``the 2003 January 17 CME
event'').  Dates included in tables should be in IAU abridged format (for
example, 2003 Jul 4).''  


\section{ACRONYMS}

Use acronyms where appropriate, but define them when they are first used
(in both
abstract and main text).  If they are used only once or twice, write them out
in full.  Do not use an acronym if it makes the sentence harder to read aloud.

The choice of indefinite article (i.e., ``a'' or ``an'') before an
acronym is determined by how the acronym is conventionally pronounced,
so it is ``an SFR estimate,'' but ``a UFO.''  Although most acronyms are
pronounced as a series of letters, there are exceptions, which can also affect
the article, e.g., ``a NASA mission.''
If you use many acronyms, include a list of them at the end of the paper.


\section{TITLE}

The title of \Planck\ general and special ``early'' papers is of the form
``\Planck\ early results. XX. Specific title of paper,'' while the products
and cosmology papers from 2013 were similarly ``\Planck\ 2013 results. XX.
Specific title of paper,'' with XX being an
assigned sequence number.  Note the punctuation and capitalization.  In
general, only ``\Planck'' and the first word of the specific title are
capitalized; however, anything that would normally be capitalized in the middle
of a text sentence should also be capitalized.  Do not use abbreviations and
acronyms, except those that are in general use.  Try to avoid use of Greek
letters and other special symbols (indexing services often cannot reproduce
these accurately).  See Sect.~3 above about the italics.  

For intermediate papers use ``\Planck\ intermediate results. XX. Specific title
of paper'' (with no punctuation at the end).  For cosmology and product papers
in 2015 use ``\Planck\ 2015 results. XX. Specific title of paper'' (once again,
no punctuation at the end).

The ``running title'' (which appears at the top of each page of the paper)
should be the title, or a shorter version of the title, but does not need the
series name and roman numerals.  The ``running author'' should simply be
``Planck Collaboration.''
 

\section{ABSTRACT}

The abstract is a summary of the paper, not part of the paper.  Do not include
anything in the abstract that is not also included (usually at greater length)
in the text of the paper. Do not treat the abstract as an introduction to the
paper; the paper should make sense without the abstract.  Do not include
references in the abstract.

All significant or important conclusions of the paper should be contained
in the abstract, including numerical results (with uncertainties or confidence
levels) when appropriate.  Avoid vague statements such as ``We discuss the
implications of the observations.''  It is usually best to write an abstract
in an impersonal style (avoiding ``I'' and ``we'').

Use the ``traditional'' abstract style. {In \verb|aa.cls v7.0|, used for the
2011 and 2013 \Planck\ papers, the default was the A\&A ``structured'' abstract
with its amateurish-looking headings.  To obtain the ``traditional'' abstract,
one had to specify \verb|\documentclass[traditabstract]{aa}| on the first line of
the input file.  With \verb|aa.cls v8.2|, the ``structured'' abstract has
disappeared (good riddance!); all that's required is
\verb|\documentclass{aa}|.}

It is preferable to write the abstract as a single paragraph.  Shorter is
better.  References should be avoided in the abstract.  Acronyms should only
be defined if they are used again in the abstract.

Key words appear at the end of the abstract.  These must be selected from
the approved list:

\url{http://www.aanda.org/index.php?option=com_content\&task=view\&id=170\&Itemid=173},

\noindent
rather than just made up!  Key words should be separated by en-dashes (\verb|--| in
\TeX), not some other form of abbreviation.


\section{CORRESPONDING AUTHOR AND ACKNOWLEDGEMENTS}

The corresponding author, the single point of contact with A\&A, is
``generated'' during the submission process, and not specified in the paper
itself.  For that reason, submission must be performed by the corresponding
author.  Nonetheless, so that we can keep track of who is handling these
secretarial duties for all the papers, insert \verb|\thanks{Corresponding author: J. J. Doe \url{<email.address>}}| immediately after the corresponding author's
name and footnote reference in the \verb|Proj_Ref_n_n_authors_and_institutes.tex|
file (see Sect.~2.3 for the use of \verb|\url{}|).

\smallskip

Every paper should include the following footnote immediately after the first
instance of ``\Planck'' in the text.  Don't try to footnote ``\Planck'' in the
title or abstract!

\begin{verbatim}
\footnote{\Planck\ (\url{http://www.esa.int/Planck}) is a project of the European Space
Agency (ESA) with instruments provided by two scientific consortia funded by ESA member
states and led by Principal Investigators from France and Italy, telescope reflectors
provided through a collaboration between ESA and a scientific consortium led and funded
by Denmark, and additional contributions from NASA (USA).}
\end{verbatim}

A standard acknowledgement in both long and short forms was provided for use
at the end of the early and intermediate papers, and has also been used for the
cosmology and product papers.  The current version of the basic \Planck\
acknowledgement is:


\begin{verbatim}
The Planck Collaboration acknowledges the support of: ESA; CNES, and CNRS/INSU-IN2P3-INP
(France); ASI, CNR, and INAF (Italy); NASA and DoE (USA); STFC and UKSA (UK); CSIC,
MINECO, JA, and RES (Spain); Tekes, AoF, and CSC (Finland); DLR and MPG (Germany); CSA
(Canada); DTU Space (Denmark); SER/SSO (Switzerland); RCN (Norway); SFI (Ireland);
FCT/MCTES (Portugal); ERC and PRACE (EU). A description of the Planck Collaboration and
a list of its members, indicating which technical or scientific activities they have been
involved in, can be found at \url{http://www.cosmos.esa.int/web/planck/planck-collaboration}.
\end{verbatim}

Additional acknowledgements will be appropriate
for individual papers, for example, the use of the {\tt HEALPix} package, or
data from another instrument, mission, or repository, such as {\it XMM\/},
{\it Herschel\/}, 2MASS, SDSS, LAMBDA, etc.

The acknowledgements section should not contain references.  Instead, cite the
relevant paper somewhere in the main body of the text.



\section{UNITS}

\begin{enumerate}

\item Units should {\it always\/} be in a roman font, {\it never\/} in
italics!!  For example, $g = 9.8$\,m\,s$^{-2}$ is correct,
but $g=9.8\,m\,s^{-2}$ is
{\bf wrong}.  It may take extra work to keep the units out of math mode, or
control the font inside math mode, but it must be done!  Commands for many
common units are defined in Planck.tex so that they can be used in or out of
math mode, producing the correct (roman) fonts in either case.

\item Units should {\it always\/} be singular, {\it never\/} plural.  For
example, ``erg,'' not ``ergs'' (but see \#3, next).

\item When units are written out in text (as they should be when used without
a numerical value), they are not capitalized even if formed from a proper name,
and the plural is always formed by adding an ``s.''  For example, ``the flux
density values were converted to janskys,''  not ``janskies.''

\item Use SI units. Avoid as far as possible non-SI units (including ergs,
inches, cm, cu.ft.). Use the correct SI abbreviations, e.g., ``kV'' not
``KV,'' ``GHz'' not ``Ghz,''

\item Microns as a unit of length should be written ``$\!$\micron,'' defined in
Planck.tex as both \verb|\micron| and \verb|\microns| (so you don't have to remember).

\item Units should be separated from numbers (and from other units) by a
``thinspace,'' available in La\TeX\ in both math and non-math modes as
``\verb|\,|\,\,''  For example, $H_0 = 68$\kmsMpc\ is obtained by
\verb|$H_0=68$\,km\,s$^{-1}$\,Mpc$^{-1}$|
or \verb|$H_0=68\,{\rm km}\,{\rm s}^{-1}\,{\rm Mpc}^{-1}$|.
Note that in Planck.tex, \verb|\kmsMpc| is defined
to produce proper units either inside or outside of math mode.

\item Avoid units in subscripts --- it's better to define an appropriate
notation at the first use, e.g., ``the flux density at 70\,GHz,
$S_{70}$ ,$\ldots$.''

\item Write km\,s\mo, {\it not\/} km/s.  This applies to all compound
units, e.g., MJy\,sr$^{-1}$, rather than MJy/sr.

\item Write ``s,'' {\it not\/} ``sec.''

\item In figure labels and tables, enclose units in brackets, e.g.,
``Time [s],''  ``Frequency [Hz],'' ``Sensitivity [\muKs].''  A\&A apparently
prefer to have units in round brackets, e.g., ``Time (s).''  However,
we succeeded with some Early Papers and will persevere.  The less used
square brackets are more distinctive, and if used consistently make it easy to
distinguish units from other quantities.
See Sect.~18.13 for an example of how units should be
specified in a table (whatever kind of brackets are used).

\item Degrees, arcminutes, and arcseconds are generally typeset as symbols,
\deg, \arcm, \arcs, which can be obtained with \verb|\deg|, \verb|\arcm|, and \verb|\arcs|,
defined in Planck.tex.  Non-integer values should place the symbol over the
decimal point.  Spacing is a bit tricky because of the different shapes of the
digits 0--9, but the commands \verb|\pdeg|, \verb|\parcm|, and \verb|\parcs| in Planck.tex
work well.  For example, \verb|2\pdeg8|, \verb|2\parcm8|, and \verb|2\parcs8| produce 2\pdeg8,
2\parcm8, and 2\parcs8, respectively.  All of these commands work in or out of
math mode.  A\&A provides \verb|\degr|, \verb|\arcmin|, \verb|\arcsec|, \verb|\fdg|, \verb|\farcm|, and
\verb|\farcs|, for the same purpose, but the digits and angle symbols are not
spaced as well.

\item Use ``${\rm deg}^2$'' rather than ``square degrees'' or ``sq.\ deg.''

\item It is good practice to try to avoid ambiguity by making sure that units
apply to both values and uncertainties, e.g., $x=(3\pm1)\,$m, not $x=3\pm1\,$m.
 Similarly write ``we use the 70\,\% and 80\,\% masks'' rather than ``we use
the 70 and 80\,\% masks.''

\item If a cosmological model is required, use the fiducial \Planck\ model
found at

\url{http://hfilfi.planck.fr/index.php/Site/Commonalities}.

This will be updated for the 2015 papers.

\item For the temperature of the CMB, use $T_0=(2.7255\pm0.0006)\,$K
(Fixsen 2009).  That's the rounded version of what Fixsen says in the paper
for the fit to FIRAS plus other experiments.

\end{enumerate}


\section{NOTATION}

\begin{enumerate}

\item The acronym for cosmic microwave background is ``CMB,'' not ``CBR,''
``CMBR,'' or anything else.

\item Use $\ell$, obtained with \verb|$\ell$|, for the multipole index only.
Galactic longitude is written ``$l$,'' obtained with \verb|$l$|.

\item The plural of $C_\ell$ (\verb|$C_\ell$|) should be written $C_\ell$s
(\verb|$C_\ell$s|), {\it not\/} $C_\ell$'s (\verb|$C_\ell$'s|).
 
\item
``E'' and ``B'' should be in italics in phrases such as $B$-modes, as
well as in subscripts and superscripts such as $C_\ell^{EE}$.  The same applies
to $T$, $Q$, and $U$.  Although these are used as labels, they are also
variables, and this rule ensures consistency in sentences such as
``we estimate the $B$-mode power spectrum, $C_\ell^{BB}$.''


\item The symbol for flux density is $S_\nu$ (obtained with \verb|$S_\nu$|),
{\it not\/} $f_\nu$ or anything else.  \Planck\ measures flux density,
{\it not\/} ``flux,''  and the use of ``flux'' as shorthand for ``flux
density'' is {\bf never} allowed, since ``flux'' is a physically distinct
quantity.

\item Use the percent symbol ``\%'' rather than ``per cent'' or ``percent''
following a number, with a thin space (\verb|\,|) before the \% sign (e.g., 73\,\%).
But in a phrase such as ``a few percent'' write out the words.

\item Use \verb|\Msolar| and \verb|\Lsolar| for \Msolar\ and \Lsolar.

\item Write CO (or similar) transitions as ``CO $J$=1$\rightarrow$0,''
obtained with \verb|CO $J$=1$\rightarrow$0|.  Note that setting ``=1'' in text mode
bypasses the normal math mode spacing of the equals sign, which is too wide
for this situation; as an alternative use small spaces, i.e.,
``CO $J\,{=}\,1\,{\rightarrow}\,0$,''
obtained with \verb|CO $J\,{=}\,1\,{\rightarrow}\,0$|.
If there are many instances, abbreviation to
``CO(1$\rightarrow$0)'' after the first instance is all right. 

\item
The notation for all cosmological parameters should follow table~1 in the
2013 parameters paper, Planck Collaboration XVI.


\item
Avoid the use of excessive numbers of digits when presenting numerical results.
See Sect.~18.1 for more details.


\item
Ordinal numbers should be written $17$th, rather than $17$-th or $17^{\rm th}$.
The same applies to variables, e.g., $i$th.


\item
Specific WMAP releases and results can be referred to as, e.g.,
``the WMAP 9-year data set'' or shortened to ``WMAP-9.''


\item
Use ``$\gamma$-ray'' not ``gamma-ray'' (and mixing them in the same paper is
even worse!).  It should always be hyphenated, even when used as a noun,
just as in ``X-ray.''


\item
A\&A prefer to use ``S/N'' for ``signal-to-noise ratio,'' rather than ``SNR''
(which they say can be confused with ``supernova remnant'').  Let's agree
to go with ``S/N'' (even if we suspect that the two uses of ``SNR'' would
always be distinguished by context).  Note that this is an abbreviation for
``signal-to-noise ratio'' and not ``signal-to-noise.''

\item
For atomic ionization states, use the A\&A macro: e.g., ``\verb|\ion{H}{ii}|,''
which yields ``H\,{\sc II}.''


\item
The transpose symbol (for vectors and tensors) should be capital and should
{\it not\/} be in italic font, since that causes possible confusion with the
variable $T$.  The best solution is probably to use the sans-serif font,
e.g., ``\verb|\vec{x}^{\sf T}|'' or ``\verb|\vec{x}^\mathsf{T}|.''

\end{enumerate}


\section{PUNCTUATION, ABBREVIATIONS, AND CAPITALIZATION}

\begin{enumerate}

\item The abbreviations ``i.e.'' and ``e.g.'' should be in roman font and
should always be followed by a comma, e.g., ``i.e., something.''  The A\&A
Author's Guide provides no explicit instruction on this point, and is itself
inconsistent in its use of a following comma.  For uniformity, we will adopt
use of the comma.  The question of capitalization is irrelevant, because
``i.e.'' and ``e.g.'' should not be used at the beginning of a sentence.

\item Precise rules for the use of commas are complicated.  As a rough
guide, if adding a comma would help the reader to take a brief pause, which
would avoid ambiguities and make the sentence easier to understand, then
definitely the comma should be added.  However, in many instances the
inclusion of a comma is a matter of taste.  We have added a discussion of
some of the subtle issues in an Appendix.

\item A\&A uses the ``serial comma'' (also called the ``Oxford comma''),
i.e., a comma precedes the ``and'' before the final item in a list of three or
more, e.g., ``chocolate, strawberry, and vanilla.''  The same applies to
``or,'' e.g., ``stracciatella, fragola, or ``zuppa inglese.''

\item According to the A\&A Author's Guide, ``The following expressions should
always be abbreviated unless they appear at the beginning of a sentence (i.e.,
Sect., Sects., Fig., Figs., Col., Cols.).  Table is never abbreviated.''

\item  Every internal reference to a figure or table should use
\verb|\ref{label}| to ensure that a hyperlink is created in the final PDF file.


\item  The correct way to refer to a figure, table, or equation in
a paper is ``Fig.~1,'' ``Table~2,'' and ``Eq.~(3),'' respectively.  At
the beginning of a sentence, write ``Figure'' and ``Equation'' in
full.  When an equation is referred to in parentheses the brackets are
dropped, so ``(see Eq.~42).''  
To reference an equation number with parentheses,
``Eq.~(3),'' use \verb|\eqref|; to reference the number without parentheses,
``Eq.~3,'' use \verb|\ref|. 
If referring to a figure, table, or equation in {\it another\/} paper, it is
good practice to use the full uncapitalized words ``figure,'' ``table,'' or
``equation,'' to avoid any confusion between the numbered items in the paper
being referred to and those in the paper being written. 

\item The IAU formally recommends that the initial letters of the names of
individual astronomical objects should be printed as capitals (see the IAU
Style Manual, {\it Trans.~Int.~Astron.~Union\/}, vol.~20B, 1989, Chapt.~8,
p.~S30); e.g., Earth, Sun, Moon, etc. ``The Earth's equator''
and ``Earth is a planet in the Solar System'' are examples of correct spelling
according to these rules.  However, ``zodiacal'' and ``ecliptic''
should {\it not\/} be capitalized.

\item Capitalize ``Galactic'' when referring to the Milky Way, e.g.,
``Galactic plane.''

Capitalize ``Universe'' when referring to the cosmos in which we live,
reserving ``universe'' for different theoretical possibilities.

\item ``Zeldovich'' should be written without the apostrophe in the middle.
In general, names that are normally not written in Latin characters should be
transliterated following the preference of the owner of the name, as far as
possible.

\item Software program names should be set in the fixed-width ``tt'' font,
e.g., {\tt HEALPix}, {\tt PLIK}, and {\tt Commander} (obtained with
\verb|{\tt HEALPix}|, \verb|{\tt PLIK}|, and \verb|{\tt Commander}|). 


\item The abbreviation of declination is ``Dec,'' not ``DEC'' or ``Dec.''
The abbreviation of right ascension is ``RA,'' not ``R.A.''

\item  Words describing directions, such as ``north,'' ``southwest,'' and
``eastern,'' should not be capitalized unless they are part of a proper noun
(like ``North America'').


\item The Oxford English Dictionary (OED, the ultimate authority for standard
English) capitalizes ``Gaussian.''  We will adopt that as the \Planck\
convention.

\item
Italics should be reserved for emphasis.
Don't use italics for expressions in Latin (or other languages).  Italics should
also not be used for special phrases; it is better to define a special phrase
using quotations the first time it is introduced, and leave it at that.


\item
Itemized lists should be properly punctuated.  This is achieved through one
of two possibilities.  The first is a list (usually of fairly short
statements) introduced using a colon and with the items separated by
semi-colons, so the entire list is read as a single sentence.
The second is a list of longer entries, {\it not\/} introduced
with a colon, each item of which should start with a capital letter and end with
a period.

\smallskip

\noindent{Here's an example of the first kind of itemized list:}
\begin{itemize}
\item{this is the first item;}

\item{this is the second item;}

\item{and here's the third item, which can be long, but can only consist
of a single sentence in order to ensure the punctuation is consistent.}
\end{itemize}

\noindent{Here's an example of the second kind of itemized list,  which is {\it not\/} introduced with a colon.}
\begin{itemize}
\item{This is the first item, which should start with a capital letter
 and end with a full stop.}

\item{This is another item, which might be longer than any items in the
 first kind of list.}

\item{This is one more item, which can be long.  Items in this sort of
list can consist of multiple sentences and hence couldn't be part of a
semi-colon-separated list.}
\end{itemize}

\smallskip
\noindent
Whether these are numbered or unnumbered lists is a matter of choice, but numerical lists are preferred when the order is important, or if specific items are going to be referred to in the text.  The particular bullet symbol used is also a choice (within reason!), but the A\&A default is dashes.  If the items in a list are essentially whole paragraphs, then it may be better to use the {\tt $\backslash$paragraph$\{\dots\}$} environment, which gives a separate heading for each item.

\end{enumerate}


\section{CONTENT}

\begin{enumerate}

\item A common mistake is to include too much background material in the
introduction.  Journal articles are not review articles. Cite all
closely-related papers and any papers yours depends on, but don't review the
whole history of the subject.

\item There will be many \Planck\ papers and we will fatigue readers if we
reproduce the same description of the instruments in every one.
 There is no longer a requirement to use a ``standard paragraph,''
but it is still appropriate to cite {\it all\/} instrument and product papers
on which a new paper depends.
Refer (once)
to the relevant instrument-description papers, and summarize as briefly as
possible any characteristics of the instrument that the reader needs to know
to understand the present paper.

\item
There should be no text before the beginning of section~1 of a paper.
It is fine to have a brief block of introductory text at the beginning of a
section (section~3, say), but if this isn't short it is probably a good idea to
give it a numbered sub-section (i.e., call it section~3.1); this makes it
easier to refer to that part of the paper.  A section should not contain a
single sub-section, i.e., if there's section~7.1, but no section~7.2, then
either rename the introductory material as section~7.1 (and the sub-section
as section~7.2) or remove the sub-section heading, making the whole thing
just section~7.


\item The concluding section of the paper should be precisely that: a concise
statement of the conclusions.  It is not necessary to repeat material from the
abstract or introduction.

\item New material should not be introduced in a section headed
``Conclusions.''  Use a ``Discussion'' section for that.

\item Strive for conciseness; avoid unnecessary verbiage.

\end{enumerate}


\section{USE (AND MISUSE) OF THE ENGLISH LANGUAGE}

Use standard British words and spellings.  We take the Oxford English Dictionary (OED) as our authority on British spelling.  Examples follow.  The
first six highlight differences between standard British and European usage.
To repeat, use standard British usage!

\begin{enumerate}

\item {\it Performance\/} as it will be used in \Planck\ papers is a singular
noun.  For example, ``The {\bf performance} of \Planck'' is correct.  ``The
{\bf performances} of \Planck'' is not.

\smallskip

\begin{quote}
``Performance'' is a
noun with two related meanings.  One type of ``performance'' is quantized and
can be counted, the other is 
continuous and can be measured.  Example 1: ``The orchestra will give one
performance on Monday and two {\bf performances} on Thursday.''  Example 2:
``The {\bf performance} of the \Planck\ telescope at cryogenic temperatures
was difficult to measure.''  The key distinction is that the quantized,
countable type of performance has number, i.e., it is singular or plural
depending on the count.  But the continuous, measurable version, has no number.
It {\it never\/} has an ``s'' at the end.  The countable type of
``performance'' is unlikely in the \Planck\ papers.  Therefore a global search
and replace of ``performances'' with ``performance'' will safely eliminate
misuse.
\end{quote}

\item {\it Noise\/} as it will be used in \Planck\/ papers is also measurable
and continuous, and therefore a singular noun.  ``The {\bf noises} of
\Planck'' is not correct.

\item {\it Emission\/}, as in ``bright diffuse emission,'' and as it will most
likely be used in the \Planck\ papers, is also singular.
``Bright diffuse {\bf emissions}'' is not correct.

\item
 {\it Significance\/} is similarly used as a singular noun, so that
{\bf significances} is incorrect.  If necessary write ``levels of
significance.''

Use of the word ``significant'' when {\it not\/} discussing statistical
significance can be confusing.  Be careful, or don't do it.  Synonyms that
can be usefully substituted include ``considerable,'' ``sizable,'' and
``substantial.''

\item {\it Allow\/} is a transitive verb, and requires an object.

\begin{itemize}
\item``The accuracy of this model {\bf allows us} to remove the effects
of thermal fluctuations from the data directly'' is correct.

\item``The accuracy of this model {\bf allows removal of} the effects of
thermal fluctuations from the data directly'' is correct.

\item``The accuracy of this model {\bf allows to remove} the effects of
thermal fluctuations from the data directly'' is incorrect, because
{\bf to remove} is not an object.
\end{itemize}

\item {\it Permit\/} is also a transitive verb.  See item 5 above.

\item One of the weirdnesses of English usage, that sometimes verbs are
followed by an infinitive while at other times they are followed by a gerund,
is explained quite well at

\url{http://www.englishpage.com/gerunds/index.htm}.

\item ``{\it Modelisation\/}'' (or ``modelization'') is not in the
OED, at least not yet.  Don't use it.  ``Model'' or ``modelling'' are probably
what you want.

\item

{\it ``Associated to''\/} is usually incorrect in English, and should be
``associated with.''


\item All sentences must have a verb; subject and verb must match in number.

\item {\it That\/} and {\it which\/} should be used as explained in this paragraph from the A\&A English Guide:

``That'' (not in phrases such as ``enough \dots that \dots'') is never
preceded by a comma, because it introduces a restrictive clause.  If tempted
to use a comma there, then check that ``which'' is not more appropriate
(=non-restrictive).  That ``that'' is already used for so many functions makes
it all the more necessary to keep to the conventions.  Even though standard
English allows ``which'' to be used for the restrictive dependent clause,
scientific articles prefer to keep the difference to the non-restrictive even
clearer by using only ``that'' without comma or ``which'' with a comma when
non-restrictive.  Example: ``Both metallicity components appear to have a
common origin, which is different from that of the dark-matter halo.'' vs.\
``Both metallicity components appear to have a common origin that is different
from that of the dark-matter halo.''

If that doesn't all make sense, concentrate on the example, and remember that
no comma should precede ``that,'' but a comma should always precede ``which.''

\item A\&A itself is flexible on spelling.  Their basic principle is ``to ask
for consistency within an article, whether in the punctuation, capitalization,
spelling, or abbreviations.''  Our choice for the \Planck\ papers is to follow
common British conventions.  Table~1 gives examples of
differences.  \Planck\ will adopt the versions in maroon.  

\noindent Note that we
specify ``polarize'' instead of ``polarise.''  Although both ``-ize'' and
``-ise'' are used in British English, the OED prefers ``-ize.''
``Polarise'' certainly appears a few times in the Early and Intermediate
Results papers,
the major use of ``polarize'' and related nouns is now with us, so it
seems worth a minor inconsistency with previous papers in order to follow
the OED more closely in the future.

\item ``Between A and B,'' ``from A to B,'' or ``in the range A--B'' are
\hbox{OK}.  ``Between A to B,'' ``from A--B,'' and ``between A--B'' are not.

\item For the plural of ``halo'' write ``halos,'' not ``haloes.''

\item It is better to use the term ``uncertainties'' than ``errors.''  When
giving uncertainties, state the confidence interval and its probability
content, e.g., 68.3\% or 99.5\%.  Avoid using, e.g., $2\,\sigma$ or $3\,\sigma$,
especially if the underlying distribution is non-Gaussian or asymmetric.  An
uncertainty introduced by ``$\pm$'' (e.g., $x\pm y$) is taken to be a
symmetric 68.3\% confidence interval ($[x-y,\ x+y]$) unless otherwise stated.
Upper limits need careful explanation. 

\item After introducing an acronym, use only the acronym.

\item Use active voice when suitable, particularly when necessary for correct
syntax (e.g., ``To address this possibility, we constructed a $\lambda$ Zap
library$\ldots$,'' not ``To address this possibility, a $\lambda$ Zap library
was constructed$\ldots$'').  But see Sect.~8 on abstracts.

\item Write concisely (e.g., ``even though,'' not ``in spite of the fact
that'').

\item When two or more similar terms are used throughout text, either make
the usage consistent or clarify the distinctions(s), as appropriate.

\item Avoid using terms such as ``novel,'' ``first,'' or ``our laboratory has
pioneered$\ldots$'' to describe the present work.  The novelty should be
apparent without being highlighted.  Similarly try to minimize claims that a
result is ``interesting,'' ``important,'' or ``critical.''
Do not mention your own work in progress
within the text (cross reference the appropriate \Planck\ paper).

\item ``A and B'' or variants such as ``A together with B'' are plural
subjects and need a plural verb, e.g., ``A and B are\dots.''

\item Avoid using ``systematic'' or ``systematics'' as a noun.
Use ``systematic errors'' or ``systematic effects.'' 

\item The names of things don't usually need to be capitalized, e.g.,
``orthomode transducer,'' not ``Orthomode Transducer,'' even when defining an
acronym, e.g., ``active galactic nucleus (AGN),'' not ``Active Galactic
Nucleus (AGN),''
and ``cosmic microwave background (CMB),'' not
``Cosmic Microwave Background (CMB).''

\item ``Time-ordered data'' should be hyphenated.
``Bandpass,''
``feedhorn,''
``mapmaking,''
``nonlinear,''
``sidelobe,''
and
``submillimetre''
should not.  As another example, it
should be ``far sidelobes,'' rather than ``far-side lobes.''
Furthermore, ``stray light'' should be written as two words.
The rules of hyphenation can be daunting because there are so many cases
(see, e.g., \url{http://en.wikipedia.org/wiki/English_compound}), but most of them
involve nouns used as adjectives in multiple combinations, affecting a
relatively small number of cases.  If in doubt, don't hyphenate.

\item
One hyphenation guideline is clear, namely that a hyphen is included
in adjectival phrases but not in nouns.  So it is ``the power-law spectrum,''
but ``a power law was fit'' and ``the high-$\ell$ behaviour,'' but
``an effect seen at high $\ell$.''

\item
Nouns used adjectivally are {\bf never\/} plural in English, not even once!
For example, ``the galaxy redshifts'' is correct, even although there are
multiple galaxies, but ``the clusters masses'' is incorrect.

\item The ampersand, ``\&,'' is not acceptable in a sentence.  Write, e.g.,
``using the 100 and 143$\,$GHz channels,'' not ``using the 100 \& 143$\,$GHz
channels.''

\item Spell out numbers up to and including ten; use digits above ten except at
the beginning of a sentence.\footnote{And don't ask about the
pathological case of ranges like ``seven to 11!''}  However, numbers with units are always written with
digits, including things like ``$5\,\sigma$.''

Similarly, quantities that are multiplicative factors (even if they happen to
be integers) will often be better written using digits, e.g., ``5 times
higher'' or ``a factor of 2.''

\item
Sentences should not start with numbers or variables.  Rewrite the sentence
if necessary.

\item
Avoid the use of whole sentences in brackets.  This can be a useful device in
conversational emails, but when writing a paper its intended meaning is often
unclear.  Instead put the comment at the end of the previous sentence (in
brackets, or after a semi-colon), convert to a footnote if deemed necessary, or
simply remove the brackets altogether.

\item Use ``reliability'' rather than ``purity'' for the probability that a
source is real.  For example, the reliability of the catalogue is 0.95.

\item
Use ``data set'' rather than ``dataset'' or ``data-set.''

\item
The word ``data'' is always plural (e.g., ``these data show''), while ``none''
can be singular (e.g., ``none of the sky is masked'') {\it or\/} plural
(e.g., ``none of the galaxies were spirals'').

\item
The common abbreviation for ``root mean square'' should be ``rms,'' rather than
``r.m.s.'' or ``RMS.''

\item
It often sounds more elegant to avoid using the word ``do,'' but instead to
use ``perform'' or ``carry out.''  For example, ``we carried out the
calibration procedure'' rather than ``we did the calibration procedure.''

\item
As the word ``as'' can be used as an adverb or preposition, as well as as a
conjunction (note this example!), then sentences can sometimes be difficult
to parse.  Substitution of ``since'' or ``because'' for ``as'' where
appropriate (particularly as a conjunction) may avoid confusion.

\item
The word ``comprised'' is sometimes misused in the phrase ``is comprised of,''
which should be replaced with the more grammatically correct
``is composed of'' (or perhaps ``comprises'').

\item
The correct word is ``publicly,'' not ``publically.''  However, in common
examples, there is often a better way to express the same meaning without
using the word at all.

\item
``Non'' is a prefix, not a word, and hence a hyphen is required in words such
as ``non-Gaussian'' and ``non-relativistic.''

\end{enumerate}

\begin{table*}
  \begin{center}
    \caption{
       British and American Spelling Conventions$^{\rm a}$ and Examples
    }
    \label{tab:spelling}
    {\scriptsize
    \begin{tabular}{p{2.75in}ll}
      \hline
      \hline
      & \multicolumn{1}{c}{British} & \multicolumn{1}{c}{American} \\
      \hline

\multirow[t]{5}{=}{{\bf Words ending in -re}\\British English words that end in {\it -re\/} often end in {\it -er\/} in American English.} & centre (centred, centring) & center (centered, centering)\\
& fibre&fiber\\
& litre&liter\\
& metre&meter\\
& theatre&theater {\bf or} theatre\medskip\\

\multirow[t]{5}{=}{{\bf Words ending in -our}\\British English words ending in {\it -our\/} usually end in {\it -or\/} in American English.} & colour & color\\
& flavour&flavor\\
& humour&humor\\
& labour&labor\\
& neighbour&neighbor\medskip\\

\multirow[t]{10}{=}{{\bf Words ending in -ize or -ise}\\Verbs in British English that can be spelled with either {\it -ize\/} or {\it -ise\/} at the end are always spelled with {\it -ize\/} at the end in American English.} & apodize {\bf or} apodise&apodize\\
& apologize {\bf or} apologise&apologize\\
& ionize    {\bf or} ionise&ionize\\
& minimize  {\bf or} minimise&minimize\\
& normalize {\bf or} normalise&normalize\\
& organize  {\bf or} organise&organize\\
& polarize  {\bf or} polarise&polarize\\
& realize   {\bf or} realise&realize\\
& recognize {\bf or} recognise&recognize\\
& summarize {\bf or} summarise&summarize\medskip\\

\multirow[t]{4}{=}{Related nouns follow the same convention.} & apodization {\bf or} apodisation&apodization\\
& ionization    {\bf or} ionisation&ionization\\
& normalization {\bf or} normalisation&normalization\\
& polarization  {\bf or} polarisation&polarization\medskip\\

\multirow[t]{4}{=}{However, there are some exceptions, which never take the {\it -ize\/} form in either variant of English.$^{\rm b}$} &arise&arise\\
& comprise& comprise\\
& revise& revise\\
& surprise& surprise\medskip\\

{\bf Words ending in -yse}\\Verbs in British English that end in {\it -yse\/} are always spelled {\it -yze\/} in American English. & analyse&analyze\medskip\\

\multirow[t]{10}{=}{{\bf Words ending in a vowel plus l}\\In British spelling, verbs ending in a vowel plus {\it l\/} double the {\it l\/} when adding endings that begin with a vowel.  In American English, the {\it l\/} is not doubled.} & {\it fuel}&{\it fuel}\\
& fuelled&fueled\\
& fuelling&fueling\smallskip\\
&\it model&\it model\\
& modelled&modeled\\
& modelling&modeling\smallskip\\
&\it travel&\it travel\\
& travelled&traveled\\
& travelling&traveling\\
& traveller&traveler\medskip\\

\multirow[t]{4}{=}{{\bf Nouns ending with -ence}\\Some nouns that end with {\it -ence\/} in British English are spelled {\it -ense\/} in American English.} & defence&defense\\
& licence&license\\
& offence&offense\\
& pretence&pretense\medskip\\

\multirow[t]{8}{=}{{\bf Nouns ending with -ogue}\\Some nouns that end with {\it -ogue\/} in British English end with either {\it -og\/} or {\it -ogue\/} in American English. The distinctions here are not hard and fast.  The spelling {\it analogue\/} is acceptable but not very common in American English; {\it catalog\/} has become the US norm, but {\it catalogue\/} is not uncommon; {\it dialogue\/} is still preferred over {\it dialog\/}.} & analogue&analog {\bf or} analogue\\
& catalogue&catalog {\bf or} catalogue\\
& dialogue&dialog {\bf or} dialogue\\
&&\\
&&\\
&&\\
&&\\
&&\medskip\\

\multirow[t]{6}{=}{{\bf Other examples}\\A few more British versions that should be used in \Planck\ papers.} &artefact& artifact\\
& disc  (except for a computer disk)& disk\\
& formulae& formulas\\
& grey& gray\\
& manoeuvre& maneuver\\
& towards  {\bf or} toward&toward {\bf or} towards\\


      \hline 
   \end{tabular}
   }
  \end{center}
  {\scriptsize
   $^{\rm a}$ Mostly from {\tt http://oxforddictionaries.com/words/british-and-american-spelling}\\
   $^{\rm b}$ And note that ``noise'' is never ``noize,'' so you can't simply do a global search and replace for {\it -ise\/} words!
   }
\end{table*} 



\section{La\TeX\ AND \TeX; TYPESETTING MATHEMATICS}

\begin{enumerate}

\item {\it Paragraphs\/}.
Always indicate the start of a new paragraph by a blank line in the
\TeX\ code.  Special layout commands such as \verb|\vskip| and \verb|\noindent| are not
generally  required; the journal-supplied La\TeX\ styles will take care of
layout.

\item {\it Blank lines\/}. Don't leave a blank line before or
(particularly) after a displayed equation
in your input file unless you want a new paragraph.  If you want some visual
separation between displayed equations and text in your input file, use comment
lines, e.g.,

\begin{verbatim}
Text blah blah blah
%
\begin{equation}
E=mc^2,
\end{equation}
%
where $c$ is the speed of light in vacuum.
\end{verbatim}

\item {\it Quotation marks\/}. Use ``{\tt``}'' and ``{\tt''},'' not the
double-quote character found on English keyboards above the apostrophe.
A comma or period goes inside the quotation marks, not outside, e.g.,
``\Planck\ is a great mission.''

\item {\it Dashes\/}. Distinguish hyphen (-), produced with a single dash
({\tt -}) and used for compound words (e.g., ``free-free'') and word breaks;
en-dash (--), produced with two dashes ({\tt --}) and used for a range;
em-dash (---), produced with three dashes ({\tt ---}) and used (infrequently)
as a punctuation mark; and minus ($-$), produced by a dash in math mode
({\tt \$-\$}). {\bf Note that minus signs can only be typeset in math mode
(including in tables!).  Conversely, hyphens, en dashes, and em dashes cannot
be typeset in math mode.}  Always set the complete mathematical expression in
math-mode, e.g., ``{\tt\$-17.2\char`\\pm0.3\$}'' rather than
``{\tt\$-\$17.2\$\char`\\pm\$0.3}'' in order to get correct spacing.  The
former gives $-17.2\pm0.3$, while the latter gives $-$17.2$\pm$0.3.

\item {\it Commas\/}. To avoid adding extra space in math mode, a comma may be
put in brackets.  Compare the result of \verb|$a{,}b$| ($a{,}b$), with the result
of \verb|$a,b$| ($a,b$).

\item {\it More commas\/}. In English, a comma is never used for a decimal
point. 

\item {\it Symbols\/}. Use italics (the default in math mode)  for all
single-letter symbols that represent variables (i.e., quantities that have a
numerical value).  For example, use {\tt \$H\_0\$}, not {\tt H\$\_0\$}
($H_0$, not H$_0$).  Similarly, the redshift $z$ should always be in italics,
including in such expressions as ``high-$z$'' (obtained with \verb|high-$z$|).  

\item {\it Mixing symbols and words\/}.   Use relational symbols ($=$,  $<$, $\simeq$, etc.) in equations, not in
text.   Write an equation, or use words.  For example, write ``frequencies of
30\,GHz and above,'' rather than ``frequencies $\ge 30$ GHz.''  Similarly,
write ``an average factor of about 1.8,'' rather than ``an average factor of
$\simeq 1.8$.''  Better yet (see 15.9 below), write ``an average factor of
1.8,'' omitting the meaningless ``about.''  

\item {\it Approximations\/}.
Use ``about,'' ``around,'' and ``approximately'' in preference to
``$\sim$,'' but use all sparingly!  They are often almost meaningless, and
their use is a bad and annoying habit on the part of the writer.  If the
uncertainty in a numerical value cannot be represented reasonably by the
number of significant digits, specify the uncertainty explicitly.  A special
microlevel of hell is reserved for those who write ``about $\sim$.''

Don't use ``$\sim$''  when you mean ``$\propto$.''

Don't use both ``$\simeq$'' (\verb|\simeq|) and ``$\approx$'' (\verb|\approx|) in
the same paper unless you explicitly mean something different.  If in doubt,
use ``$\approx$'' (and see the point above!).

For ``less (and greater) than approximately'' either use the A\&A macro
``$\backslash$la'' (and ``$\backslash$ga'') or the Planck.tex macro
``$\backslash$lsim'' (and ``$\backslash$gsim''), but don't mix them.

Avoid using the ``$O(x)$'' or ``${\cal O}(x)$'' notation except for describing
asymptotic behaviour or scaling.  Instead just say ``about $10^{-6}$'' or
``approximately $10^{120}$.''

\item {\it Superscripts\/}.
Use roman fonts for tags or labels in subscripts, e.g., $n_{\rm e}$,
$z_{\rm rec}$, and for multi-letter operators.  This avoids ambiguities by
always explicitly distinguishing variables from abbreviations.  For example,
$z_i$ (obtained with \verb|$z_i$|) might be the $i$th redshift under consideration,
while $z_{\rm i}$ (obtained with \verb|$z_{\rm i}$|) might be defined as the
reionization redshift.

An exception is made for labels that are {\it also\/} variables, e.g.,
``the $x$-component of vector {\tt FATAL ERROR} %$\elevenboldmath V$
 is $V_x$.''  This will often
apply to $T$, $Q$, $U$, $E$, and $B$ (see Sect.~11, item 4).

Particle physicists sometimes write particle names in italics, e.g., $n_e$
instead of $n_{\rm e}$.  We suspect they are just being lazy.  But some
typesetters always use italics for a single-letter symbol, perhaps because
they don't know if it is a variable or a tag.  The important thing is that
multi-letter symbols should be in roman to avoid the confusion of whether
``{\it em\/}'' is a single symbol or ``$e$'' times ``$m$.''

\item {\it Functions\/}.
Always use the standard \TeX\ commands for operators, \verb|\log|, \verb|\cos|,
\verb|\sin|, \verb|\ln|, etc.  Right: \verb|$\log{S}$|.  Wrong: \verb|log($S$)|, or
\verb|${\rm log} S$|, or anything else.  Using the \TeX\ commands will also
preclude capitalization of these operators, which is almost always incorrect.

%\input boldmathfonts

\item {\it Vectors and tensors\/}.
A\&A recommends typesetting vectors and tensors with \verb|\vec{A}| and
\verb|\tens{B}|.  These produce bold italics and upright sans-serif characters,
i.e., {\tt FATAL ERROR} %$\elevenboldmath A$ and {\ss B}
, respectively.

\item {\it Brackets\/}. The usual ordering of brackets is
$\left\{ \left[ \left( \dots \right) \right] \right\}$.  Only deviate
if this if there is good reason, and never use the same type for adjacent
brackets.

Distinguish angle brackets
({\tt\$\char`\\langle\$, \$\char`\\rangle\$} producing $\langle$ and
$\rangle$), often used to denote expected value) from the inequality operators
$<$ and $>$.  Note that ``{\tt <}'' and ``{\tt >}'' must never be used outside
math mode.

\item {\it Acronyms\/}. Try to avoid using an acronym as a variable
(e.g., ``$SFR = 10$\,\Msolar\,yr\mo''), because it is cumbersome.  Define a new
symbol instead (e.g., ``the star formation rate, ${\cal R}$'').

If you {\it do\/} use a multi-letter symbol for a variable, it {\bf must}
be in roman, e.g., ``${\rm SFR} = 10\,\Msolar$\,yr\mo,'' obtained with
``\verb|${\rm SFR} = 10\,\Msolar$\,yr\mo|.''

\item {\it Long equations\/}. Don't try to make equations fit by using
\verb|\small|!  Instead use \verb|\eqnarray| or something similar to break lines.

\item {\it Tall equations\/}. It looks ugly when in-line equations contain
expressions or brackets that are high enough to force \TeX\ to insert extra
space between the lines of text.  As a general guide, if it increases the line
spacing, then it's time to use a displayed equation.

\item {\it $N_{\rm side}$}.  Write ``$N_{\rm side}$'' (\verb|$N_{\rm side}$|),
not ``nside.''

\item {$\chi^2$.}  Write $\chi^2$ rather than ``chi-square.''  Whether giving
$\chi^2$ or reduced $\chi^2$, always give $N_{\rm dof}$.

\item {\it Equation references\/}. From the A\&A Author's Guide: ``All equations
that you are referring to with \verb|\ref| must have the corresponding \verb|\label|
--- please use this mechanism only.  Punctuate a displayed equation in the
same way as ordinary text.''  This means that displayed equations should
usually be followed by a comma or period, which generally look better preceded
by a thin space \verb|\,| or a medium space \verb|\>|.

\item
{\it Equation numbering\/}.  Although not obligatory, it is good practice to
number all equations.  Even if you don't intend to refer to the equation
elsewhere in the paper, other people might want to do so (e.g., the referee).

\item {\it Stokes parameters\/}. Since his name was ``Stokes,'' write
``Stokes $Q$,'' not ``Stoke's $Q$.''

\item {\it Spacing of exponents\/}.  Now and then, depending on characters,
exponents come out too close to the exponentiated symbol, e.g., $\nu^\beta$.
Space can be added with, e.g., \verb|$\nu^{\,\beta}$|, giving $\nu^{\,\beta}$ or
(better) \verb|$\nu^{\hbox{\hglue 0.7pt}\beta}$|, giving
$\nu^{\hbox{\hglue 0.7pt}\beta}$, which makes it easy to ``tune'' the space.

\item {\it Thousands separator\/}.  A\&A doesn't give a preference for how to
write numbers with many digits.  The SI standard is to use spaces rather than
commas or periods (1,000,000 or 1.000.000) to separate thousands.  Thin spaces
(\verb|\,|) should be used, giving 1\,000\,000.  {\it Never\/} simply type a space
in the input file (\verb|10 000|), as that would allow a line break in the middle of
the number.  Numbers with only 4 digits probably don't need a space -- use
common sense.

\item {\it Writing ``$\sigma$''}.  Write
\verb|$5\,\sigma$| ($5\,\sigma$) or possibly
\verb|5\sigma| ($5\sigma$), but never \verb|5-$\sigma$| (5-$\sigma$), and
{\it especially\/} never \verb|$5-\sigma$| ($5-\sigma$).

\item
{\it Derivatives}.
Derivatives should be written with italic ``{\it d\/}''s, e.g., $dy/dx$.
A\&A writes derivatives with roman ``d''s, e.g., ${\rm d}y/{\rm d}x$, and will
change $d \rightarrow $ d in the proof stage.  The reason for not following
the A\&A style is that the \TeX\ input is more straightforward and less prone
to error with italic ``{\it d\/}''s, and the \Planck\ papers will require fewer
corrections in the final edit stage.  A\&A can then make the corrections
uniformly. 

\end{enumerate}


\section{REFERENCES}

\begin{enumerate}

\item Use \verb|Planck_bib.bib| in the aa.cls environment to ensure correct
references to the \Planck\ papers.  When referring to a \Planck\ paper in text,
use, e.g., \verb|as described in \cite{planck2011-7.2}|, which becomes ``as
described in Planck Collaboration XX (2011).''  Alternatively some sentences
could refer to work (rather than a specific paper) ``by the Planck
Collaboration''; note that here ``the'' is not capitalized.  For a
parenthetical reference, \verb|\citep{Planck2011-7.2}| produces ``(Planck
Collaboration XX 2011).'' 

\item Make sure that you use the current version of \verb|Planck_bib.bib|,
available at 

\url{http://www.sciops.esa.int/index.php?project=PLANCK&page=Repositories}

\item  The bibliography style file \verb|aat.bst| should be used, which will
give full titles (helpful for distinguishing \Planck\ papers) as well as
arXiv hyperlinks in the reference list.  This may require care over inclusion
(and/or exclusion) of appropriate packages (e.g., \verb|ifthen|); if there's a problem, follow what
was done in a previous \Planck\ paper.

\item ADS+Bib\TeX\ does not always put references in the correct
format for the journal.  Check them carefully.  Take special care with
articles in books and conference proceedings: include the title of the book,
the editors' names, the publisher and place of publication, and the page
number.  ADS+Bib\TeX+A\&A also usually omits the arXiv preprint number, which
is wrong.  This can be fixed by editing the .bib file.

SPIRES Bib\TeX\ entries for \Planck\ papers should not be
used at present, since they result in the author being ``Ade et al.''\ and do
not include the identifying roman numerals.  (We're trying to get this fixed,
but in the meantime don't use them.)

\item A useful reference for use of \verb|natbib| can be found at
\url{http://merkel.zoneo.net/Latex/natbib.php}.
There are many ways to produce bad combinations of parentheses, commas,
semicolons, and years.  Beware!

One example is the incorrect
\verb|\citep[as studied by][]{FirstPaper, SecondPaper, ThirdPaper}|, which produces
``(as studied by FirstPaper; SecondPaper; ThirdPaper).''  To obtain the
correctly punctuated ``(as studied by FirstPaper, SecondPaper, and ThirdPaper)''
requires three separate uses of \verb|\citealt|.

\item
Use a consistent format for journal abbreviations, i.e., avoid having a
mixture of ``ApJ'' and ``Astrophys.\ J.''  Best practice is to use the
built-in abbreviations, e.g., \verb|\aap| for ``A\&A,'' \verb|\apj| for ``ApJ,'' and
\verb|\mnras| for ``MNRAS.''

\end{enumerate}


\section{FIGURES}

\section{TABLES}

\appendix

\section{Use of commas}





%\bibliographystyle{bib/refs}
%\bibliography{bib/references}

\end{document}
