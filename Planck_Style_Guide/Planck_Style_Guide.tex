% Updated with comments from Douglas Scott and Tim Pearson 13 September 2010.
% Corrections made 28 September 2010.
% Many additions made 15 November 2010, with help from Douglas and Tim.
%    A conflict in the definition of \endtable with an aa.cls command has been 
%    eliminated by changing the name in Planck.tex to \endPlancktable.
%
% Extensive revisions completed in July 2011.
%
% Revisions in June 2012 include 10.6 (percent), 14.7 (mixing relational
% symbols and words), and 14.11.
%
% Revision in August 2012 includes a complete revision of the section on
% figures and a substantial update on tables
% Douglas updated 21st October 2012, including fixing lots of typos and
%     making it more uniform in style
%
% Need to add 11 Jan 2013: 
%                             use A&A package v8.1 after kinks are understood
%                             more details about acknowledgements.  See emails
%                             from around 14 March and before from Tim, Douglas,
%                             and Andrew Jaffe.
% Revision in July 2014 includes most of suggestions from 2013
% Revisions in September and October 2014 update figure style information and
%     issues with TeX (including A&A v8.2)
% Mild revisions in December 2014, collecting issues which came up during
%     editing of 2014 papers
% Update of acknowledgments, etc., in April 2015.
% Further revisions for issues arising with submitted 2015 papers
% October 2015: commented out subsection 1.1 on ``Changes since 2013'', and
%     turn off color-coding for changes and new stuff.
% January 2016: minor issues added that arose in dealing with revisions to
% 2015 papers

\input proposalmacros

\input Planck.tex
%
%       Have to redefine the table of contents macros, because Proposalmacros
%       uses ``|'' as a delimiter, and that's used as an active character in
%       this document for verbatim material.
%
\def\TCAP#1!#2.{\vskip 3pt\line{\hbox to \tocwidth{#1\hfil}#2\hfil}}
\def\TCA#1!#2!#3.{\vskip 3pt\line{\hbox to \tocwidth{#1\hfil}#2\hfil#3}}
\def\TCB#1!#2!#3.{\line{\hglue 2em#1\hskip 10pt#2\hfil#3}}
\def\TCC#1!#2!#3.{\line{\hglue 4em#1\hskip 10pt{\it #2}\hfil#3}}
%
\toksAP={\TCAP}
\outer\def\appendix#1;#2\par{
   \def\rightheadline{\elevenpoint\it\hfil\botmark\hfil\rm#1--\folio}
   \def\leftheadline{\elevenpoint\rm#1--\folio\hfil\it\botmark\hfil}
   \pageno=1\mark{APPENDIX~#1~#2}
   \bigskip\leftline{\twelvebf APPENDIX~#1~#2}
  {\let\the=0\edef\next{\write\toc{\the\toksAP !APPENDIX~#1\quad#2.}}\next}%
     \nobreak\medskip}
%
\toksA={\TCA}
\outer\def\section#1\par{\advance\sectnumber by 1\footnotenumber=0\subsectnumber=0%
%     \def\verso{\number\sectnumber~#1}
%     \def\recto{\number\sectnumber~#1}
     \vskip 17pt plus 4pt minus 3pt\goodbreak
     \leftline{{\twelvebf\number\sectnumber\enspace#1}}
     \mark{\number\sectnumber~#1}
  {\let\the=0\edef\next{\write\toc{\the\toksA
     \number\sectnumber!#1!\the\count0.}}\next}%
     \nobreak\medskip\noindent}
%
\toksB={\TCB}
\outer\def\subsection#1\par{\subsubsectnumber=0\advance\subsectnumber by1%
    \ifdim\lastskip=\medskipamount\nobreak\vskip 8pt plus 2pt minus 2pt
      \else \vskip 14pt plus 3pt minus 3pt\goodbreak\fi
     \leftline{{\bf\number\sectnumber.\number\subsectnumber\enspace#1}}%
  {\let\the=0\edef\next{\write\toc{\the\toksB
     \number\sectnumber.\number\subsectnumber!#1!\the\count0.}}\next}%
     \nobreak\medskip\noindent}
%
\toksC={\TCC}
\outer\def\subsubsection#1\par{\advance\subsubsectnumber by1%
  \ifdim\lastskip=\medskipamount\nobreak\vskip 8pt plus 2pt minus 2pt
    \else \vskip 14pt plus 3pt minus 3pt\goodbreak\fi
   \leftline{{\elevenpoint
   \it\number\sectnumber.\number\subsectnumber.\number\subsubsectnumber\enspace
     #1}}%
  {\let\the=0\edef\next{\write\toc{\the\toksC \number\sectnumber.%
       \number\subsectnumber.\number\subsubsectnumber!#1!\the\count0.}}\next}%
     \nobreak\medskip\noindent}
%
%  Have to define \columnwidth to use the table macros of Planck.tex, which are
%  set up for A&A LaTeX.
%
\newdimen{\columnwidth}
\let\columnwidth=\hsize
%
%
%
%
%
\def\today{\number\day/\number\month/\number\year}
\def\rightheadline{%
     \rlap{\eightpoint Version \today}
     \elevenpoint\it\hfil\botmark\hfil\rm\folio}
\def\leftheadline{\elevenpoint\rm\folio\hfil\it\botmark\hfil
     \llap{\eightpoint Version~\today}}
%
\elevenpoint
\font\ib=cmbxti10 at 11pt
\def\Planck{{\it Planck\/}}
\def\PLANCK{{\it PLANCK\/}}
\def\Planckb{{\ib Planck\/}}
\def\PLANCKB{{\ib PLANCK\/}}
\font\title=cmbx12 at 30pt
\font\titlep=cmbxti10 at 30pt
\font\ittitle=cmti12 at 20pt
\font\csctitle=cmcsc10 at 14pt
\font\csc=cmcsc10 at 11pt
\font\heleight=Helvetica at 8pt
\font\helten=Helvetica at 10pt
\font\heltwelve=Helvetica at 12pt
\font\ss=cmss10 at 11pt
%
\def\boxit#1{\vbox{\hrule\hbox{\vrule\hskip 4pt
    \vbox{\vskip 4pt #1 \vskip 4pt}\hskip 4pt\vrule}\hrule}}

\def\example#1{\begingroup\smallskip\leftskip=2em\rightskip=2em\eightpoint\noindent\hangindent=1em\hangafter=1 #1\par\endgroup}
%
\def\next{\smallskip\noindent
\advance\bullnumber by 1
\the\bullnumber) }

\hyphenation{manu-scripts}
%
% define commands to change the color of new and revised material
%\def\New{\red}
%\def\Modified{\blue}
\def\New{\black}
\def\Modified{\black}
%
%
% macros for verbatim scanning
\newskip\ttglue
\ttglue=.5em plus.25em minus.15em

\chardef\other=12
\def\ttverbatim{\begingroup
  \catcode`\\=\other
  \catcode`\{=\other
  \catcode`\}=\other
  \catcode`\$=\other
  \catcode`\&=\other
  \catcode`\#=\other
  \catcode`\%=\other
  \catcode`\~=\other
  \catcode`\_=\other
  \catcode`\^=\other
  \obeyspaces \obeylines \tt}

\outer\def\begintt{$$\let\par=\endgraf \ttverbatim \parskip=\z@
  \catcode`\|=0 \rightskip-5pc \ttfinish}
{\catcode`\|=0 |catcode`|\=\other % | is temporary escape character
  |obeylines % end of line is active
  |gdef|ttfinish#1^^M#2\endtt{#1|vbox{#2}|endgroup$$}}

\catcode`\|=\active
{\obeylines \gdef|{\ttverbatim \spaceskip\ttglue \let^^M=\  \let|=\endgroup}}

% Things for The METAFONTbook only
%\ifx\MFmanual\!\else\endinput\fi

\openout\toc=toc
%
%
%
%
\pageno=0
\vglue 3cm
\baselineskip=46pt
\centerline{\title STYLE GUIDE}
\centerline{\title for}
\centerline{{\titlep PLANCK\/\ } {\title PAPERS}}
\centerline{\title 2015}

\vskip 0pt plus1600fill

\font\csclg=cmcsc10 at 24pt
\font\csc=cmcsc10 at 14pt

\vskip 0pt plus1600fill

\baselineskip=23pt

\centerline{\csc C. R. Lawrence, T. J. Pearson, Douglas Scott,}
\centerline{\csc L. Spencer, A. Zonca}
\vskip 0pt plus1600fill

\centerline{\csc \today}


\page
\hbox{}
\pageno=0
\page
 
\pageno=-1

\elevenpoint


\centerline{\twelvebf TABLE OF CONTENTS}

\vskip 1cm

\begingroup

\setuptoc

\input toc

\endgroup


\page 


\pageno=1


\section PURPOSE

This Style Guide is intended to help authors of \Planck-related papers prepare
high-quality manuscripts in a uniform style. It supplements the instructions
to authors provided by the  Astronomy \& Astrophysics Author's guide 

\hglue 1.0cm \maroon |http://www.aanda.org/doc_journal/instructions/aa_instructions.pdf|\black

\noindent A\&A also has a useful English usage guide, available at 

\hglue 1.0cm \maroon |http://www.aanda.org/doc_journal/instructions/aa_english_guide.pdf|\black.

\noindent Both can be found under ``Author Information'' at
\maroon |http://www.aanda.org/|\black.


\black


%\New 
%\subsection Changes since 2013\black


%Changes since the 2013 edition of this document are highlighted with the
%following colour-coding:

%\hglue 1.0cm \New this is an example of ``new material;''\black

%\hglue 1.0cm \Modified this is an example of ``modified material.''\black


\New\section \TeX\ STUFF\black

The A\&A format is implemented in the A\&A document class file |aa.cls|,
available from 

\hglue 1.0cm\maroon|http://www.aanda.org/|\black

\noindent and from the \Planck\ SVN at

\hglue 1.0cm\maroon|https://scisvn01.esac.esa.int/Planck_Publication_Management|\black

\noindent\New In the 2013 papers, we used |aa.cls v7.0|.  A\&A is now up to
|v8.2|, although |v7.0| is still allowed by A\&A.  Most of the differences between |v7.0| and |v8.2|
need no comment here, but there are three compatibility issues of note.

\next The ``structured'' abstract that was the default in |v7.0| no longer
exists in |v8.0|, eliminating the need for the |[traditabstract]| option in
the |\document{aa}| command at the beginning of the La\TeX\ file.  

\next The 2013 version of the |Planck.tex| file (see below) is not compatible
with |aa.cls v8.2|.  Make sure to use the current version of |Planck.tex|,
always available at

\hglue 1.0cm\maroon|https://scisvn01.esac.esa.int/Planck_Publication_Management|\black

\Modified\next Up until September 2014, \Planck\ author lists were generated
with ``|\\ \and|'' between institution names.  With |aa.cls v7.0|, the
resulting double space between institution names at the end of \Planck\ papers
could be fixed by putting |\raggedright| immediately before |\end{document}|.
This kludge no longer works with |aa.cls v8.2|.  Author lists are now generated
with ``|\goodbreak \and|'' between institutions.  No |\raggedright| is
required.  This is still a kludge to deal with a problem that A\&A should fix
in |aa.cls|, but it works until they do.  

\medskip

Unfortunately, |aa.cls v8.2| puts
long lists of institutions (the kind \Planck\ papers always have!) immediately
following the references, and before the appendices (if any).  This is weird,
to put it politely, and we are trying to get it changed.  {\bf Until we do, however, it is best to use aa.cls v7.0.  (Specify ``traditabstract,'' of course.
The institution list fix is backwards compatible.)}

\medskip
\black


\subsection Planck.tex

La\TeX\ commands useful and specific to \Planck\ papers are found in
|Planck.tex|, available on the PPM web page.  To use, insert the line
|\input Planck.tex| after the initial |\documentclass| command in your input
file.

\New There are two important changes in |Planck.tex| in 2015.  First,
as mentioned above, a command defined in earlier versions conflicted with
|aa.cls| versions 8.0 and later.  That command has been removed.  If you use
|Planck.tex| from 2013 with these later versions of |aa.cls|, you'll get an
immediate and unhelpful error message.  Second, the |\setsymbol{|$\ldots$|}|
definition machinery to specify instrumental and other parameters proved
unworkable.  All of the |\setsymbol| commands have been removed.  Attempts to
use this  machinery will fail.
\black

\subsection A fix for an annoying La\TeX\ problem

Sometimes La\TeX\ puts figures in the wrong order.  For example,
single-column Fig.\,10 may appear before double-column Fig.\,9.  The solution
is to add

|\usepackage{fixltx2e}|

\smallskip
\noindent a ``La\TeX\ bug fix package'' that usually fixes this problem and a
few others.

\subsection Active links

To include active links in the output .pdf file, include 
|\usepackage{hyperref}| at the beginning of the La\TeX\ file and use
|\url{}|, e.g., |\footnote{\url{http://www.asdc.asi.it/fermibsl/}}|.
\Modified
Occasionally La\TeX\ will complain because of a hyperlink being split across
a page break --- in such cases, then simplest solution is to find the
offending link and enclose it within |\mbox{}|.
\black

\section HOW TO REFER TO \PLANCKB\ AND OTHER PROJECTS

Refer to ``the \Planck\ project,''  ``the \Planck\ spacecraft,'' or ``\Planck.''
The name should be italicized.
This can be done in all font environments (e.g., normal text, bold titles, or
section headings) with |\textit{Planck}|.  |\Planck| is so defined in
|Planck.tex| for convenience.  Additionally ``Planck'' is not in italics in
references that include the phrase ``Planck Collaboration.'' 
\New
By the same logic ``Planck'' is not italicized in other proper noun phrases,
such as ``Planck Catalogue of Compact Sources.'' Italics are never used for
``Planck constant'' or ``Max Planck.''
\black

\Modified
\subsection Other experiments

For the 2013 papers, we requested that all spacecraft names be
italicized.  This conflicts with the A\&A house style, which we will
adopt for the 2015 papers (irrespective of whether we all agree that it
makes sense!).  In this scheme, an instrument name is {\it not\/}
italicized unless it is named after a person, so it is
``{\it Chandra},'' ``{\it Fermi},'' ``{\it Herschel},'' and ``{\it Spitzer},''
but ``Gaia,'' ``GALEX,'' ``IRAS,'' and ``ISO.''  The names of instruments and
non-satellite experiments remain in roman font, e.g., ``ACT,'' ``DIRBE,''
``HFI,'' ``LFI,'' ``SPIRE,'' and ``SPT.''  Note that the convention means that
you should write
``{\it Hubble\/} Space Telescope,'' 
``{\it Wilkinson\/} Microwave Anisotropy Probe,''
and ``XMM-{\it Newton},''
but just ``HST,'' ``WMAP,'' and ``XMM.''
\black

\subsection The possessive form

The possessive version of \Planck\ is ``\Planck's,'' written |\Planck's|.
Note that the apostrophe and the ``s'' are roman, and that the italic
correction is built into the definition of |\Planck|, so you don't have to
worry about spacing.

%|Planck.tex| includes the definitions |\Planck|, |Planckb|, |\PLANCK|,
%and |\PLANCKB| to obtain \Planck, \Planckb, \PLANCK, and \PLANCKB.  


\section HOW TO REFER TO \PLANCKB\ SPECIFICS
%CATALOGUES, CRYOCOOLERS, AND SURVEYS

Use ``Planck Early Release Compact Source Catalogue (ERCSC)'' at the first
reference to it in the text, and ``ERCSC'' thereafter.  The ERCSC contains the
Early Sunyaev-Zeldovich (ESZ) catalogue and the Early Cold Cores (ECC)
catalogue, which should be referred to similarly.  Don't worry that the
expression ``the ESZ'' seems to be missing a noun.

Similarly, use ``Planck Catalogue of Compact Sources (PCCS)'' and
``Planck Cluster Catalogue (PCC)'' for the catalogues released in
March 2013.  Names for the second and likely final versions of the catalogues
to be released in 2015 are not yet settled.  When they are, they'll be added
here.

The  \Planck\ cryocoolers should be referred to as the
``sorption cooler,'' the ``\HeJT\ cooler,'' and the ``dilution cooler.''
The command |\HeJT|, defined in |Planck.tex|, produces \HeJT\ for convenience. 

\New
Individual \Planck\ surveys are precisely defined subsets of the data and
should be referred to using capital letters, i.e., ``Survey~1,'' ``Survey~2,''
etc.  Similarly for ``Year~1,'' etc.
\black


\section DATES

A\&A is flexible about the format for dates, ``as long as you remain
consistent.''
A\&A itself writes {\it Received\/} and {\it Accepted\/} dates in
``day month year'' format, e.g., 11 January 2011.  We will adopt that as the
standard for dates in the \Planck\ papers.

There are two exceptions.  The A\&A ``Guide to the English Editing''
(see Sect.~1 above) says ``When the
date is an integral part of the event's name, the use of the IAU format
is recommended but not mandatory (for example, ``the 2003 January 17 CME
event'').  Dates included in tables should be in IAU abridged format (for
example, 2003 Jul 4).''  


\section ACRONYMS

Use acronyms where appropriate, but define them when they are first used
(in both
abstract and main text).  If they are used only once or twice, write them out
in full.  Do not use an acronym if it makes the sentence harder to read aloud.
\New The choice of indefinite article (i.e., ``a'' or ``an'') before an
acronym is determined by how the acronym is conventionally pronounced,
so it is ``an SFR estimate,'' but ``a UFO.''  Although most acronyms are
pronounced as a series of letters, there are exceptions, which can also affect
the article, e.g., ``a NASA mission.''\black\
If you use many acronyms, include a list of them at the end of the paper.


\section TITLE

The title of \Planck\ general and special ``early'' papers is of the form
``\Planck\ early results. XX. Specific title of paper,'' while the products
and cosmology papers from 2013 were similarly ``\Planck\ 2013 results. XX.
Specific title of paper,'' with XX being an
assigned sequence number.  Note the punctuation and capitalization.  In
general, only ``\Planck'' and the first word of the specific title are
capitalized; however, anything that would normally be capitalized in the middle
of a text sentence should also be capitalized.  Do not use abbreviations and
acronyms, except those that are in general use.  Try to avoid use of Greek
letters and other special symbols (indexing services often cannot reproduce
these accurately).  See Sect.~3 above about the italics.  

For intermediate papers use ``\Planck\ intermediate results. XX. Specific title
of paper'' (with no punctuation at the end).  For cosmology and product papers
in 2015 use ``\Planck\ 2015 results. XX. Specific title of paper'' (once again,
no punctuation at the end).

The ``running title'' (which appears at the top of each page of the paper)
should be the title, or a shorter version of the title, but does not need the
series name and roman numerals.  The ``running author'' should simply be
``Planck Collaboration.''
 

\section ABSTRACT

The abstract is a summary of the paper, not part of the paper.  Do not include
anything in the abstract that is not also included (usually at greater length)
in the text of the paper. Do not treat the abstract as an introduction to the
paper; the paper should make sense without the abstract.  Do not include
references in the abstract.

All significant or important conclusions of the paper should be contained
in the abstract, including numerical results (with uncertainties or confidence
levels) when appropriate.  Avoid vague statements such as ``We discuss the
implications of the observations.''  It is usually best to write an abstract
in an impersonal style (avoiding ``I'' ``and ``we'').

Use the ``traditional'' abstract style.  \New{In |aa.cls v7.0|, used for the
2011 and 2013 \Planck\ papers, the default was the A\&A ``structured'' abstract
with its amateurish-looking headings.  To obtain the ``traditional'' abstract,
one had to specify |\documentclass[traditabstract]{aa}| on the first line of
the input file.  With |aa.cls v8.2|, the ``structured'' abstract has
disappeared (good riddance!); all that's required is
|\documentclass{aa}|.}\black

\New
It is preferable to write the abstract as a single paragraph.  Shorter is
better.  References should be avoided in the abstract.  Acronyms should only
be defined if they are used again in the abstract.\black

\New
Key words appear at the end of the abstract.  These must be selected from
the approved list:

\maroon |http://www.aanda.org/index.php?option=com_content\&task=view\&id=170\&Itemid=173| \black,

\noindent
\New
rather than just made up!  Key words should be separated by en-dashes (|--| in
\TeX), not some other form of abbreviation.\black


\section CORRESPONDING AUTHOR AND ACKNOWLEDGEMENTS

The corresponding author, the single point of contact with A\&A, is
``generated'' during the submission process, and not specified in the paper
itself.  For that reason, submission must be performed by the corresponding
author.  Nonetheless, so that we can keep track of who is handling these
secretarial duties for all the papers, insert |\thanks{Corresponding author:
J. J. Doe \url{<email.address>}}| immediately after the corresponding author's
name and footnote reference in the |Proj_Ref_n_n_authors_and_institutes.tex|
file (see Sect.~2.3 for the use of |\url{}|).

\smallskip

Every paper should include the following footnote immediately after the first
instance of ``\Planck'' in the text.  Don't try to footnote ``\Planck'' in the
title or abstract!

\Modified
|\footnote{\Planck\ (\url{http://www.esa.int/Planck}) is a project of the 
European Space Agency (ESA) with instruments provided by two scientific 
consortia funded by ESA member states and led by Principal Investigators 
from France and Italy, telescope reflectors provided through a collaboration 
between ESA and a scientific consortium led and funded by Denmark, and 
additional contributions from NASA (USA).}|
\black

A standard acknowledgement in both long and short forms was provided for use
at the end of the early and intermediate papers, and has also been used for the
cosmology and product papers.  The current version of the basic \Planck\
acknowledgement is:

\Modified
|The Planck Collaboration acknowledges the support of: ESA; CNES, and 
CNRS/INSU-IN2P3-INP (France); ASI, CNR, and INAF (Italy); NASA and DoE 
(USA); STFC and UKSA (UK); CSIC, MINECO, JA, and RES (Spain); Tekes, AoF, 
and CSC (Finland); DLR and MPG (Germany); CSA (Canada); DTU Space 
(Denmark); SER/SSO (Switzerland); RCN (Norway); SFI (Ireland); 
FCT/MCTES (Portugal); ERC and PRACE (EU). A description of the Planck 
Collaboration and a list of its members, indicating which technical 
or scientific activities they have been involved in, can be found at 
\url{http://www.cosmos.esa.int/web/planck/planck-collaboration}.|
\black

Additional acknowledgements will be appropriate
for individual papers, for example, the use of the {\tt HEALPix} package, or
data from another instrument, mission, or repository, such as {\it XMM\/},
{\it Herschel\/}, 2MASS, SDSS, LAMBDA, etc.
\New
The acknowledgements section should not contain references.  Instead, cite the
relevant paper somewhere in the main body of the text.
\black

\bullnumber=0

\section UNITS

\def\next{\smallskip\noindent
\advance\bullnumber by 1
\the\bullnumber) }
\vskip -11pt

\next Units should {\it always\/} be in a roman font, {\it never\/} in
italics!!  For example, $g = 9.8$\,m\,s$^{-2}$ is correct,
but $g=9.8\,m\,s^{-2}$ is
{\bf wrong}.  It may take extra work to keep the units out of math mode, or
control the font inside math mode, but it must be done!  Commands for many
common units are defined in Planck.tex so that they can be used in or out of
math mode, producing the correct (roman) fonts in either case.

\next Units should {\it always\/} be singular, {\it never\/} plural.  For
example, ``erg,'' not ``ergs'' (but see \#3, next).

\next When units are written out in text (as they should be when used without
a numerical value), they are not capitalized even if formed from a proper name,
and the plural is always formed by adding an ``s.''  For example, ``the flux
density values were converted to janskys,''  not ``janskies.''

\next Use SI units. Avoid as far as possible non-SI units (including ergs,
inches, cm, cu.ft.). Use the correct SI abbreviations, e.g., ``kV'' not
``KV,'' ``GHz'' not ``Ghz,''

\next Microns as a unit of length should be written ``$\!$\micron,'' defined in
Planck.tex as both |\micron| and |\microns| (so you don't have to remember).

\next Units should be separated from numbers (and from other units) by a
``thinspace,'' available in La\TeX\ in both math and non-math modes as
``|\,|\,\,''  For example, $H_0 = 68$\kmsMpc\ is obtained by
|$H_0=68$\,km\,s$^{-1}$\,Mpc$^{-1}$|
or |$H_0=68\,{\rm km}\,{\rm s}^{-1}\,{\rm Mpc}^{-1}$|.
Note that in Planck.tex, |\kmsMpc| is defined
to produce proper units either inside or outside of math mode.

\next Avoid units in subscripts --- it's better to define an appropriate
notation at the first use, e.g., ``the flux density at 70\,GHz,
$S_{70}$ ,$\ldots$.''

\next Write km\,s\mo, {\it not\/} km/s.  This applies to all compound
units, e.g., MJy\,sr$^{-1}$, rather than MJy/sr.

\next Write ``s,'' {\it not\/} ``sec.''

\next In figure labels and tables, enclose units in brackets, e.g.,
``Time [s],''  ``Frequency [Hz],'' ``Sensitivity [\muKs].''  A\&A apparently
prefer to have units in round brackets, e.g., ``Time (s).''  However,
we succeeded with some Early Papers and will persevere.  The less used
square brackets are more distinctive, and if used consistently make it easy to
distinguish units from other quantities.
See Sect.~18.13 for an example of how units should be
specified in a table (whatever kind of brackets are used).

\next Degrees, arcminutes, and arcseconds are generally typeset as symbols,
\deg, \arcm, \arcs, which can be obtained with |\deg|, |\arcm|, and |\arcs|,
defined in Planck.tex.  Non-integer values should place the symbol over the
decimal point.  Spacing is a bit tricky because of the different shapes of the
digits 0--9, but the commands |\pdeg|, |\parcm|, and |\parcs| in Planck.tex
work well.  For example, |2\pdeg8|, |2\parcm8|, and |2\parcs8| produce 2\pdeg8,
2\parcm8, and 2\parcs8, respectively.  All of these commands work in or out of
math mode.  A\&A provides |\degr|, |\arcmin|, |\arcsec|, |\fdg|, |\farcm|, and
|\farcs|, for the same purpose, but the digits and angle symbols are not
spaced as well.

\next Use ``${\rm deg}^2$'' rather than ``square degrees'' or ``sq.\ deg.''

\New
\next It is good practice to try to avoid ambiguity by making sure that units
apply to both values and uncertainties, e.g., $x=(3\pm1)\,$m, not $x=3\pm1\,$m.
 Similarly write ``we use the 70\,\% and 80\,\% masks'' rather than ``we use
the 70 and 80\,\% masks.''
\black

\next If a cosmological model is required, use the fiducial \Planck\ model
found at

\maroon |http://hfilfi.planck.fr/index.php/Site/Commonalities|\black.

This will be updated for the 2015 papers.

\next For the temperature of the CMB, use $T_0=(2.7255\pm0.0006)\,$K
(Fixsen 2009).  That's the rounded version of what Fixsen says in the paper
for the fit to FIRAS plus other experiments.


\bullnumber=0

\section NOTATION

\next The acronym for cosmic microwave background is ``CMB,'' not ``CBR,''
``CMBR,'' or anything else.

\next Use $\ell$, obtained with |$\ell$|, for the multipole index only.
Galactic longitude is written ``$l$,'' obtained with |$l$|.

\next The plural of $C_\ell$ (|$C_\ell$|) should be written $C_\ell$s
(|$C_\ell$s|), {\it not\/} $C_\ell$'s (|$C_\ell$'s|).
 
\next
\Modified
``E'' and ``B'' should be in italics in phrases such as $B$-modes, as
well as in subscripts and superscripts such as $C_\ell^{EE}$.  The same applies
to $T$, $Q$, and $U$.  Although these are used as labels, they are also
variables, and this rule ensures consistency in sentences such as
``we estimate the $B$-mode power spectrum, $C_\ell^{BB}$.''
\black

\next The symbol for flux density is $S_\nu$ (obtained with |$S_\nu$|),
{\it not\/} $f_\nu$ or anything else.  \Planck\ measures flux density,
{\it not\/} ``flux,''  and the use of ``flux'' as shorthand for ``flux
density'' is {\bf never} allowed, since ``flux'' is a physically distinct
quantity.

\next Use the percent symbol ``\%'' rather than ``per cent'' or ``percent''
following a number, with a thin space (|\,|) before the \% sign (e.g., 73\,\%).
But in a phrase such as ``a few percent'' write out the words.

\next Use |\Msolar| and |\Lsolar| for \Msolar\ and \Lsolar.

\next Write CO (or similar) transitions as ``CO $J$=1$\rightarrow$0,''
obtained with |CO $J$=1$\rightarrow$0|.  Note that setting ``=1'' in text mode
bypasses the normal math mode spacing of the equals sign, which is too wide
for this situation; as an alternative use small spaces, i.e.,
``CO $J\,{=}\,1\,{\rightarrow}\,0$,''
obtained with |CO $J\,{=}\,1\,{\rightarrow}\,0$|.
If there are many instances, abbreviation to
``CO(1$\rightarrow$0)'' after the first instance is all right. 

\next
\New
The notation for all cosmological parameters should follow table~1 in the
2013 parameters paper, Planck Collaboration XVI.
\black

\next
\New
Avoid the use of excessive numbers of digits when presenting numerical results.
See Sect.~18.1 for more details.
\black

\next
\New
Ordinal numbers should be written $17$th, rather than $17$-th or $17^{\rm th}$.
The same applies to variables, e.g., $i$th.
\black

\next
\New
Specific WMAP releases and results can be referred to as, e.g.,
``the WMAP 9-year data set'' or shortened to ``WMAP-9.''
\black

\next
\New
Use ``$\gamma$-ray'' not ``gamma-ray'' (and mixing them in the same paper is
even worse!).  It should always be hyphenated, even when used as a noun,
just as in ``X-ray.''
\black

\next
\New
A\&A prefer to use ``S/N'' for ``signal-to-noise ratio,'' rather than ``SNR''
(which they say can be confused with ``supernova remnant'').  Let's agree
to go with ``S/N'' (even if we suspect that the two uses of ``SNR'' would
always be distinguished by context).  Note that this is an abbreviation for
``signal-to-noise ratio'' and not ``signal-to-noise.''

\next
\New
For atomic ionization states, use the A\&A macro: e.g., ``|\ion{H}{ii}|,''
which yields ``H\,{\sc II}.''
\black

\next
\New
The transpose symbol (for vectors and tensors) should be capital and should
{\it not\/} be in italic font, since that causes possible confusion with the
variable $T$.  The best solution is probably to use the sans-serif font,
e.g., ``|\vec{x}^{\sf T}|'' or ``|\vec{x}^\mathsf{T}|.''
\black


\bullnumber=0

\section PUNCTUATION, ABBREVIATIONS, AND CAPITALIZATION

\next The abbreviations ``i.e.'' and ``e.g.'' should be in roman font and
should always be followed by a comma, e.g., ``i.e., something.''  The A\&A
Author's Guide provides no explicit instruction on this point, and is itself
inconsistent in its use of a following comma.  For uniformity, we will adopt
use of the comma.  The question of capitalization is irrelevant, because
``i.e.'' and ``e.g.'' should not be used at the beginning of a sentence.

\next Precise rules for the use of commas are complicated.  As a rough
guide, if adding a comma would help the reader to take a brief pause, which
would avoid ambiguities and make the sentence easier to understand, then
definitely the comma should be added.  However, in many instances the
inclusion of a comma is a matter of taste.  We have added a discussion of
some of the subtle issues in an Appendix.

\next A\&A uses the ``serial comma'' (also called the ``Oxford comma''),
i.e., a comma precedes the ``and'' before the final item in a list of three or
more, e.g., ``chocolate, strawberry, and vanilla.''  The same applies to
``or,'' e.g., ``stracciatella, fragola, or ``zuppa inglese.''

\next According to the A\&A Author's Guide, ``The following expressions should
always be abbreviated unless they appear at the beginning of a sentence (i.e.,
Sect., Sects., Fig., Figs., Col., Cols.).  Table is never abbreviated.''

\next \New Every internal reference to a figure or table should use
|\ref{label}| to ensure that a hyperlink is created in the final PDF file.
\black

\next \New The correct way to refer to a figure, table, or equation in
a paper is ``Fig.~1,'' ``Table~2,'' and ``Eq.~(3),'' respectively.  At
the beginning of a sentence, write ``Figure'' and ``Equation'' in
full.  When an equation is referred to in parentheses the brackets are
dropped, so ``(see Eq.~42).''  
To reference an equation number with parentheses,
``Eq.~(3),'' use |\eqref|; to reference the number without parentheses,
``Eq.~3,'' use |\ref|. 
If referring to a figure, table, or equation in {\it another\/} paper, it is
good practice to use the full uncapitalized words ``figure,'' ``table,'' or
``equation,'' to avoid any confusion between the numbered items in the paper
being referred to and those in the paper being written. 

\black

\next The IAU formally recommends that the initial letters of the names of
individual astronomical objects should be printed as capitals (see the IAU
Style Manual, {\it Trans.~Int.~Astron.~Union\/}, vol.~20B, 1989, Chapt.~8,
p.~S30); e.g., Earth, Sun, Moon, etc. ``The Earth's equator''
and ``Earth is a planet in the Solar System'' are examples of correct spelling
according to these rules.  However, ``zodiacal'' and ``ecliptic''
should {\it not\/} be capitalized.

\next Capitalize ``Galactic'' when referring to the Milky Way, e.g.,
``Galactic plane.''
\Modified
Capitalize ``Universe'' when referring to the cosmos in which we live,
reserving ``universe'' for different theoretical possibilities.
\black

\next ``Zeldovich'' should be written without the apostrophe in the middle.
In general, names that are normally not written in Latin characters should be
transliterated following the preference of the owner of the name, as far as
possible.

\next Software program names should be set in the fixed-width ``tt'' font,
\Modified
e.g., {\tt HEALPix}, {\tt PLIK}, and {\tt Commander} (obtained with
|{\tt HEALPix}|, |{\tt PLIK}|, and |{\tt Commander}|). 
\black

\next The abbreviation of declination is ``Dec,'' not ``DEC'' or ``Dec.''
The abbreviation of right ascension is ``RA,'' not ``R.A.''

\next \New Words describing directions, such as ``north,'' ``southwest,'' and
``eastern,'' should not be capitalized unless they are part of a proper noun
(like ``North America'').
\black

\next The Oxford English Dictionary (OED, the ultimate authority for standard
English) capitalizes ``Gaussian.''  We will adopt that as the \Planck\
convention.

\next
\New
Italics should be reserved for emphasis.
Don't use italics for expressions in Latin (or other languages).  Italics should
also not be used for special phrases; it is better to define a special phrase
using quotations the first time it is introduced, and leave it at that.
\black

\next
\New
Itemized lists should be properly punctuated.  This is achieved through one
of two possibilities.  The first is a list (usually of fairly short
statements) introduced using a colon and with the items separated by
semi-colons, so the entire list is read as a single sentence.
The second is a list of longer entries, {\it not\/} introduced
with a colon, each item of which should start with a capital letter and end with
a period.
\black
\smallskip
\New
\noindent{Here's an example of the first kind of itemized list:}

\example{-- this is the first item;}

\example{-- this is the second item;}

\example{-- and here's the third item, which can be long, but can only consist
of a single sentence in order to ensure the punctuation is consistent.}

\noindent{Here's an example of the second kind of itemized list,  which is {\it not\/} introduced with a colon.}

\example{-- This is the first item, which should start with a capital letter
 and end with a full stop.}

\example{-- This is another item, which might be longer than any items in the
 first kind of list.}

\example{-- This is one more item, which can be long.  Items in this sort of
list can consist of multiple sentences and hence couldn't be part of a
semi-colon-separated list.}
\smallskip
\noindent
Whether these are numbered or unnumbered lists is a matter of choice, but numerical lists are preferred when the order is important, or if specific items are going to be referred to in the text.  The particular bullet symbol used is also a choice (within reason!), but the A\&A default is dashes.  If the items in a list are essentially whole paragraphs, then it may be better to use the {\tt $\backslash$paragraph$\{\dots\}$} environment, which gives a separate heading for each item.
\black


\bullnumber=0

\section CONTENT

\next A common mistake is to include too much background material in the
introduction.  Journal articles are not review articles. Cite all
closely-related papers and any papers yours depends on, but don't review the
whole history of the subject.

\next There will be many \Planck\ papers and we will fatigue readers if we
reproduce the same description of the instruments in every one.
\New There is no longer a requirement to use a ``standard paragraph,''
but it is still appropriate to cite {\it all\/} instrument and product papers
on which a new paper depends.\black\
Refer (once)
to the relevant instrument-description papers, and summarize as briefly as
possible any characteristics of the instrument that the reader needs to know
to understand the present paper.

\next
\New
There should be no text before the beginning of section~1 of a paper.
It is fine to have a brief block of introductory text at the beginning of a
section (section~3, say), but if this isn't short it is probably a good idea to
give it a numbered sub-section (i.e., call it section~3.1); this makes it
easier to refer to that part of the paper.  A section should not contain a
single sub-section, i.e., if there's section~7.1, but no section~7.2, then
either rename the introductory material as section~7.1 (and the sub-section
as section~7.2) or remove the sub-section heading, making the whole thing
just section~7.
\black

\next The concluding section of the paper should be precisely that: a concise
statement of the conclusions.  It is not necessary to repeat material from the
abstract or introduction.

\next New material should not be introduced in a section headed
``Conclusions.''  Use a ``Discussion'' section for that.

\next Strive for conciseness; avoid unnecessary verbiage.


\bullnumber=0

\section USE (AND MISUSE) OF THE ENGLISH LANGUAGE

\noindent Use standard British words and spellings.  We take the Oxford English Dictionary (OED) as our authority on British spelling.  Examples follow.  The
first six highlight differences between standard British and European usage.
To repeat, use standard British usage!

\next {\it Performance\/} as it will be used in \Planck\ papers is a singular
noun.  For example, ``The {\bf performance} of \Planck'' is correct.  ``The
{\bf performances} of \Planck'' is not.

\smallskip

{\leftskip=2em\rightskip=2em\parindent=0pt\eightpoint ``Performance'' is a
noun with two related meanings.  One type of ``performance'' is quantized and
can be counted, the other is 
continuous and can be measured.  Example 1: ``The orchestra will give one
performance on Monday and two {\bf performances} on Thursday.''  Example 2:
``The {\bf performance} of the \Planck\ telescope at cryogenic temperatures
was difficult to measure.''  The key distinction is that the quantized,
countable type of performance has number, i.e., it is singular or plural
depending on the count.  But the continuous, measurable version, has no number.
It {\it never\/} has an ``s'' at the end.  The countable type of
``performance'' is unlikely in the \Planck\ papers.  Therefore a global search
and replace of ``performances'' with ``performance'' will safely eliminate
misuse.

}

\next {\it Noise\/} as it will be used in \Planck\/ papers is also measurable
and continuous, and therefore a singular noun.  ``The {\bf noises} of
\Planck'' is not correct.

\next {\it Emission\/}, as in ``bright diffuse emission,'' and as it will most
likely be used in the \Planck\ papers, is also singular.
``Bright diffuse {\bf emissions}'' is not correct.

\next
\New {\it Significance\/} is similarly used as a singular noun, so that
{\bf significances} is incorrect.  If necessary write ``levels of
significance.''\black

Use of the word ``significant'' when {\it not\/} discussing statistical
significance can be confusing.  Be careful, or don't do it.  Synonyms that
can be usefully substituted include ``considerable,'' ``sizable,'' and
``substantial.''

\next {\it Allow\/} is a transitive verb, and requires an object.

\example{``The accuracy of this model {\bf allows us} to remove the effects
of thermal fluctuations from the data directly'' is correct.}  

\example{``The accuracy of this model {\bf allows removal of} the effects of
thermal fluctuations from the data directly'' is correct.}  

\example{``The accuracy of this model {\bf allows to remove} the effects of
thermal fluctuations from the data directly'' is incorrect, because
{\bf to remove} is not an object.}

\next {\it Permit\/} is also a transitive verb.  See item 5 above.

\next One of the weirdnesses of English usage, that sometimes verbs are
followed by an infinitive while at other times they are followed by a gerund,
is explained quite well at \hfill\break
\indent \maroon|http://www.englishpage.com/gerunds/index.htm|\black.

\next ``{\it Modelisation\/}'' (or ``modelization'') is not in the
OED, at least not yet.  Don't use it.  ``Model'' or ``modelling'' are probably
what you want.

\next
\New
{\it ``Associated to''\/} is usually incorrect in English, and should be
``associated with.''
\black

\next All sentences must have a verb; subject and verb must match in number.

\next {\it That\/} and {\it which\/} should be used as explained in this paragraph from the A\&A English Guide:

``That'' (not in phrases such as ``enough \dots that \dots'') is never
preceded by a comma, because it introduces a restrictive clause.  If tempted
to use a comma there, then check that ``which'' is not more appropriate
(=non-restrictive).  That ``that'' is already used for so many functions makes
it all the more necessary to keep to the conventions.  Even though standard
English allows ``which'' to be used for the restrictive dependent clause,
scientific articles prefer to keep the difference to the non-restrictive even
clearer by using only ``that'' without comma or ``which'' with a comma when
non-restrictive.  Example: ``Both metallicity components appear to have a
common origin, which is different from that of the dark-matter halo.'' vs.\
``Both metallicity components appear to have a common origin that is different
from that of the dark-matter halo.''

If that doesn't all make sense, concentrate on the example, and remember that
no comma should precede ``that,'' but a comma should always precede ``which.''



\next A\&A itself is flexible on spelling.  Their basic principle is ``to ask
for consistency within an article, whether in the punctuation, capitalization,
spelling, or abbreviations.''  Our choice for the \Planck\ papers is to follow
common British conventions.  Table~1 gives examples of
differences.  \Planck\ will adopt the versions in maroon.  

\noindent Note that we
specify ``polarize'' instead of ``polarise.''  Although both ``-ize'' and
``-ise'' are used in British English, the OED prefers ``-ize.''
``Polarise'' certainly appears a few times in the Early and Intermediate
Results papers,
the major use of ``polarize'' and related nouns is now with us, so it
seems worth a minor inconsistency with previous papers in order to follow
the OED more closely in the future.

%\centerline{\epsfxsize=6.0in \epsfbox{AAspelling_chart.pdf}}

\goodbreak

%\font\link=cmssq8
\font\link=cmtt8
\midinsert
\begingroup
\hglue 2em {\bf Table 1.} \csc British and American Spelling
 Conventions$^{\rm a}$ and Examples
\vskip -16pt
\eightpoint
\setbox\tablebox=\vbox{
% 
   \newdimen\digitwidth
   \setbox0=\hbox{\rm 0}
   \digitwidth=\wd0
   \catcode`*=\active
   \def*{\kern\digitwidth} 
%
   \newdimen\signwidth
   \setbox0=\hbox{+}
   \signwidth=\wd0
   \catcode`!=\active 
   \def!{\kern\signwidth}

% 
\halign{\vbox{\hsize 2.8in\hangafter=0\hangindent=1.5em\noindent\strut#\strut\par} 
    \tabskip 2.2em&
       #\hfil&
       #\hfil\tabskip 0pt\cr
\noalign{\doubleline\vskip 2pt}
\omit&\omit\hfil British\hfil&\omit\hfil American\hfil\cr
\noalign{\vskip 4pt\hrule\vskip 6pt}
\omit\bf Words ending in -re\hfil\cr
British English words that end in {\it -re\/} often end in {\it -er\/} in
American English.\cr
\noalign{\vskip -27pt}&\maroon centre (centred, centring)\black&center (centered,
centering)\cr
&\maroon fibre\black&fiber\cr
&\maroon litre\black&liter\cr
&\maroon metre\black&meter\cr
&\maroon theatre\black&theater {\bf or} theatre\cr
\noalign{\vskip 4pt}
\omit\bf Words ending in -our\hfil\cr
British English words ending in {\it -our\/} usually end in {\it -or\/} in
American English.\cr
\noalign{\vskip -27pt}
&\maroon colour\black&color\cr
&\maroon flavour\black&flavor\cr
&\maroon humour\black&humor\cr
&\maroon labour\black&labor\cr
&\maroon neighbour\black&neighbor\cr
\noalign{\vskip 4pt}
\omit\bf Words ending in -ize or -ise\hfil\cr
Verbs in British English that can be spelled with either {\it -ize\/} or
{\it -ise\/} at the end are always spelled with {\it -ize\/} at the end in
American English.  \cr
\noalign{\vskip -36pt}
&\maroon apodize   \black{\bf or} apodise&apodize\cr
&\maroon apologize \black{\bf or} apologise&apologize\cr
&\maroon ionize    \black{\bf or} ionise&ionize\cr
&\maroon minimize  \black{\bf or} minimise&minimize\cr
&\maroon normalize \black{\bf or} normalise&normalize\cr
&\maroon organize  \black{\bf or} organise&organize\cr
&\maroon polarize  \black{\bf or} polarise&polarize\cr
&\maroon realize   \black{\bf or} realise&realize\cr
&\maroon recognize \black{\bf or} recognise&recognize\cr
&\maroon summarize \black{\bf or} summarise&summarize\cr
\noalign{\vskip 4pt}
Related nouns follow the same convention.
&\maroon apodization   \black{\bf or} apodisation&apodization\cr
&\maroon ionization    \black{\bf or} ionisation&ionization\cr
&\maroon normalization \black{\bf or} normalisation&normalization\cr
&\maroon polarization  \black{\bf or} polarisation&polarization\cr
\noalign{\vskip 4pt}
However, there are some exceptions, which never take the {\it -ize\/}
form in either variant of English.$^{\rm b}$\cr
\noalign{\vskip -18pt}
&\maroon arise&\black arise\cr
&\maroon comprise&\black comprise\cr
&\maroon revise&\black revise\cr
&\maroon surprise&\black surprise\cr
\noalign{\vskip 3pt}
\omit\bf Words ending in -yse\hfil\cr
Verbs in British English that end in {\it -yse\/} are always spelled
{\it -yze\/} in American English.\cr
\noalign{\vskip -27pt}
&\maroon analyse\black&analyze\cr
\noalign{\vskip 22pt}
\omit\bf Words ending in a vowel plus l\hfil\cr
In British spelling, verbs ending in a vowel plus {\it l\/} double the
{\it l\/} when adding endings that begin with a vowel.  In American English,
the {\it l\/} is not doubled.\cr
\noalign{\vskip -36pt}
&\it fuel&\it fuel\cr
&\maroon fuelled\black&fueled\cr
&\maroon fuelling\black&fueling\cr
\noalign{\vskip 2pt}
&\it model&\it model\cr
&\maroon modelled\black&modeled\cr
&\maroon modelling\black&modeling\cr
\noalign{\vskip 2pt}
&\it travel&\it travel\cr
&\maroon travelled\black&traveled\cr
&\maroon travelling\black&traveling\cr
&\maroon traveller\black&traveler\cr
\noalign{\vskip 4pt}
\omit\bf Nouns ending with -ence\hfil\cr
Some nouns that end with {\it -ence\/} in British English are spelled
{\it -ense\/} in American English.\cr
\noalign{\vskip -27pt}
&\maroon defence\black&defense\cr
&\maroon licence\black&license\cr
&\maroon offence\black&offense\cr
&\maroon pretence\black&pretense\cr
\noalign{\vskip 4pt}
\omit\bf Nouns ending with -ogue\hfil\cr
Some nouns that end with {\it -ogue\/} in British English end with either
{\it -og\/} or {\it -ogue\/} in American English.\cr
\noalign{\vskip -27pt}
&\maroon analogue\black&analog {\bf or} analogue\cr
&\maroon catalogue\black&catalog {\bf or} catalogue\cr
&\maroon dialogue\black&dialog {\bf or} dialogue\cr
\noalign{\vskip -1pt}
The distinctions here are not hard and fast.  The spelling {\it analogue\/} is
acceptable but not very common in American English; {\it catalog\/} has become
the US norm, but {\it catalogue\/} is not uncommon; {\it dialogue\/} is still
preferred over {\it dialog\/}.\cr
\noalign{\vskip 4pt}
\omit\bf Other examples\hfil\cr
A few more British versions that should be used in \Planck\ papers.\cr
\noalign{\vskip -27pt}
&\maroon artefact\black& artifact\cr
&\maroon disc \black (except for a computer disk)& disk\cr
&\maroon formulae\black& formulas\cr
&\maroon grey\black& gray\cr
&\maroon manoeuvre\black& maneuver\cr
&\maroon towards \black {\bf or} toward&toward {\bf or} towards\cr
\noalign{\vskip 3pt\hrule\vskip 4pt}
} }
\endPlancktable           
\tablenote a Mostly from {\link http://oxforddictionaries.com/words/british-and-american-spelling}\par
\tablenote b And note that ``noise'' is never ``noize,'' so you can't
simply do a global search and replace for {\it -ise\/} words!\par


\endgroup

\endinsert

\goodbreak


\next ``Between A and B,'' ``from A to B,'' or ``in the range A--B'' are
\hbox{OK}.  ``Between A to B,'' ``from A--B,'' and ``between A--B'' are not.

\next For the plural of ``halo'' write ``halos,'' not ``haloes.''

\next It is better to use the term ``uncertainties'' than ``errors.''  When
giving uncertainties, state the confidence interval and its probability
content, e.g., 68.3\% or 99.5\%.  Avoid using, e.g., $2\,\sigma$ or $3\,\sigma$,
especially if the underlying distribution is non-Gaussian or asymmetric.  An
uncertainty introduced by ``$\pm$'' (e.g., $x\pm y$) is taken to be a
symmetric 68.3\% confidence interval ($[x-y,\ x+y]$) unless otherwise stated.
Upper limits need careful explanation. 

\next After introducing an acronym, use only the acronym.

\next Use active voice when suitable, particularly when necessary for correct
syntax (e.g., ``To address this possibility, we constructed a $\lambda$ Zap
library$\ldots$,'' not ``To address this possibility, a $\lambda$ Zap library
was constructed$\ldots$'').  But see Sect.~8 on abstracts.

\next Write concisely (e.g., ``even though,'' not ``in spite of the fact
that'').

\next When two or more similar terms are used throughout text, either make
the usage consistent or clarify the distinctions(s), as appropriate.

\next Avoid using terms such as ``novel,'' ``first,'' or ``our laboratory has
pioneered$\ldots$'' to describe the present work.  The novelty should be
apparent without being highlighted.  Similarly try to minimize claims that a
result is ``interesting,'' ``important,'' or ``critical.''
Do not mention your own work in progress
within the text (cross reference the appropriate \Planck\ paper).

\next ``A and B'' or variants such as ``A together with B'' are plural
subjects and need a plural verb, e.g., ``A and B are\dots.''

\next Avoid using ``systematic'' or ``systematics'' as a noun.
Use ``systematic errors'' or ``systematic effects.'' 

\next The names of things don't usually need to be capitalized, e.g.,
``orthomode transducer,'' not ``Orthomode Transducer,'' even when defining an
acronym, e.g., ``active galactic nucleus (AGN),'' not ``Active Galactic
Nucleus (AGN),''
\Modified
and ``cosmic microwave background (CMB),'' not
``Cosmic Microwave Background (CMB).''
\black


\next ``Time-ordered data'' should be hyphenated.
``Bandpass,''
``feedhorn,''
``mapmaking,''
``nonlinear,''
``sidelobe,''
and
``submillimetre''
should not.  As another example, it
should be ``far sidelobes,'' rather than ``far-side lobes.''
Furthermore, ``stray light'' should be written as two words.
The rules of hyphenation can be daunting because there are so many cases
(see, e.g., |http://en.wikipedia.org/wiki/English_compound|), but most of them
involve nouns used as adjectives in multiple combinations, affecting a
relatively small number of cases.  If in doubt, don't hyphenate.

\next
\New
One hyphenation guideline is clear, namely that a hyphen is included
in adjectival phrases but not in nouns.  So it is ``the power-law spectrum,''
but ``a power law was fit'' and ``the high-$\ell$ behaviour,'' but
``an effect seen at high $\ell$.''
\black

\next
\New
Nouns used adjectivally are {\bf never\/} plural in English, not even once!
For example, ``the galaxy redshifts'' is correct, even although there are
multiple galaxies, but ``the clusters masses'' is incorrect.\black

\next The ampersand, ``\&,'' is not acceptable in a sentence.  Write, e.g.,
``using the 100 and 143$\,$GHz channels,'' not ``using the 100 \& 143$\,$GHz
channels.''

\next Spell out numbers up to and including ten; use digits above ten except at
the beginning of a sentence.\footnote{$^{\rm \dag}$}{And don't ask about the
pathological case of ranges
like ``seven to 11!''}  However, numbers with units are always written with
digits, including things like ``$5\,\sigma$.''
\Modified
Similarly, quantities that are multiplicative factors (even if they happen to
be integers) will often be better written using digits, e.g., ``5 times
higher'' or ``a factor of 2.''
\black

\next
\New
Sentences should not start with numbers or variables.  Rewrite the sentence
if necessary.
\black

\next
\New
Avoid the use of whole sentences in brackets.  This can be a useful device in
conversational emails, but when writing a paper its intended meaning is often
unclear.  Instead put the comment at the end of the previous sentence (in
brackets, or after a semi-colon), convert to a footnote if deemed necessary, or
simply remove the brackets altogether.
\black

\next Use ``reliability'' rather than ``purity'' for the probability that a
source is real.  For example, the reliability of the catalogue is 0.95.

\next
\New
Use ``data set'' rather than ``dataset'' or ``data-set.''
\black

\next
\New
The word ``data'' is always plural (e.g., ``these data show''), while ``none''
can be singular (e.g., ``none of the sky is masked'') {\it or\/} plural
(e.g., ``none of the galaxies were spirals'').
\black

\next
\New
The common abbreviation for ``root mean square'' should be ``rms,'' rather than
``r.m.s.'' or ``RMS.''
\black

\next
\New
It often sounds more elegant to avoid using the word ``do,'' but instead to
use ``perform'' or ``carry out.''  For example, ``we carried out the
calibration procedure'' rather than ``we did the calibration procedure.''
\black

\next
\New
As the word ``as'' can be used as an adverb or preposition, as well as as a
conjunction (note this example!), then sentences can sometimes be difficult
to parse.  Substitution of ``since'' or ``because'' for ``as'' where
appropriate (particularly as a conjunction) may avoid confusion.
\black

\next
\New
The word ``comprised'' is sometimes misused in the phrase ``is comprised of,''
which should be replaced with the more grammatically correct
``is composed of'' (or perhaps ``comprises'').
\black

\next
\New
The correct word is ``publicly,'' not ``publically.''  However, in common
examples, there is often a better way to express the same meaning without
using the word at all.
\black

\next
\New
``Non'' is a prefix, not a word, and hence a hyphen is required in words such
as ``non-Gaussian'' and ``non-relativistic.''
\black


\bullnumber=0

\section La\TeX\ AND \TeX; TYPESETTING MATHEMATICS

\next {\it Paragraphs\/}.
Always indicate the start of a new paragraph by a blank line in the
\TeX\ code.  Special layout commands such as |\vskip| and |\noindent| are not
generally  required; the journal-supplied La\TeX\ styles will take care of
layout.

\next {\it Blank lines\/}. Don't leave a blank line before or
(particularly) after a displayed equation
in your input file unless you want a new paragraph.  If you want some visual
separation between displayed equations and text in your input file, use comment
lines, e.g.,

|Text blah blah blah|

|%|

|\begin{equation}|

|E=mc^2,|

|\end{equation}|

|%|

|where $c$ is the speed of light in vacuum.|

\next {\it Quotation marks\/}. Use ``{\tt``}'' and ``{\tt''},'' not the
double-quote character found on English keyboards above the apostrophe.
A comma or period goes inside the quotation marks, not outside, e.g.,
``\Planck\ is a great mission.''

\next {\it Dashes\/}. Distinguish hyphen (-), produced with a single dash
({\tt -}) and used for compound words (e.g., ``free-free'') and word breaks;
en-dash (--), produced with two dashes ({\tt --}) and used for a range;
em-dash (---), produced with three dashes ({\tt ---}) and used (infrequently)
as a punctuation mark; and minus ($-$), produced by a dash in math mode
({\tt \$-\$}). {\bf Note that minus signs can only be typeset in math mode
(including in tables!).  Conversely, hyphens, en dashes, and em dashes cannot
be typeset in math mode.}  Always set the complete mathematical expression in
math-mode, e.g., ``{\tt\$-17.2\char`\\pm0.3\$}'' rather than
``{\tt\$-\$17.2\$\char`\\pm\$0.3}'' in order to get correct spacing.  The
former gives $-17.2\pm0.3$, while the latter gives $-$17.2$\pm$0.3.

\next {\it Commas\/}. To avoid adding extra space in math mode, a comma may be
put in brackets.  Compare the result of |$a{,}b$| ($a{,}b$), with the result
of |$a,b$| ($a,b$).

\next {\it More commas\/}. In English, a comma is never used for a decimal
point. 

\next {\it Symbols\/}. Use italics (the default in math mode)  for all
single-letter symbols that represent variables (i.e., quantities that have a
numerical value).  For example, use {\tt \$H\_0\$}, not {\tt H\$\_0\$}
($H_0$, not H$_0$).  Similarly, the redshift $z$ should always be in italics,
including in such expressions as ``high-$z$'' (obtained with |high-$z$|).  

\next {\it Mixing symbols and words\/}.   Use relational symbols ($=$,  $<$, $\simeq$, etc.) in equations, not in
text.   Write an equation, or use words.  For example, write ``frequencies of
30\,GHz and above,'' rather than ``frequencies $\ge 30$ GHz.''  Similarly,
write ``an average factor of about 1.8,'' rather than ``an average factor of
$\simeq 1.8$.''  Better yet (see 15.9 below), write ``an average factor of
1.8,'' omitting the meaningless ``about.''  

\next {\it Approximations\/}.
Use ``about,'' ``around,'' and ``approximately'' in preference to
``$\sim$,'' but use all sparingly!  They are often almost meaningless, and
their use is a bad and annoying habit on the part of the writer.  If the
uncertainty in a numerical value cannot be represented reasonably by the
number of significant digits, specify the uncertainty explicitly.  A special
microlevel of hell is reserved for those who write ``about $\sim$.''

Don't use ``$\sim$''  when you mean ``$\propto$.''

Don't use both ``$\simeq$'' (|\simeq|) and ``$\approx$'' (|\approx|) in
the same paper unless you explicitly mean something different.  If in doubt,
use ``$\approx$'' (and see the point above!).

\New
For ``less (and greater) than approximately'' either use the A\&A macro
``$\backslash$la'' (and ``$\backslash$ga'') or the Planck.tex macro
``$\backslash$lsim'' (and ``$\backslash$gsim''), but don't mix them.
\black

\New
Avoid using the ``$O(x)$'' or ``${\cal O}(x)$'' notation except for describing
asymptotic behaviour or scaling.  Instead just say ``about $10^{-6}$'' or
``approximately $10^{120}$.''
\black

\next {\it Superscripts\/}.
Use roman fonts for tags or labels in subscripts, e.g., $n_{\rm e}$,
$z_{\rm rec}$, and for multi-letter operators.  This avoids ambiguities by
always explicitly distinguishing variables from abbreviations.  For example,
$z_i$ (obtained with |$z_i$|) might be the $i$th redshift under consideration,
while $z_{\rm i}$ (obtained with |$z_{\rm i}$|) might be defined as the
reionization redshift.

\New
An exception is made for labels that are {\it also\/} variables, e.g.,
``the $x$-component of vector $\elevenboldmath V$ is $V_x$.''  This will often
apply to $T$, $Q$, $U$, $E$, and $B$ (see Sect.~11, item 4).
\black

Particle physicists sometimes write particle names in italics, e.g., $n_e$
instead of $n_{\rm e}$.  We suspect they are just being lazy.  But some
typesetters always use italics for a single-letter symbol, perhaps because
they don't know if it is a variable or a tag.  The important thing is that
multi-letter symbols should be in roman to avoid the confusion of whether
``{\it em\/}'' is a single symbol or ``$e$'' times ``$m$.''

\next {\it Functions\/}.
Always use the standard \TeX\ commands for operators, |\log|, |\cos|,
|\sin|, |\ln|, etc.  Right: |$\log{S}$|.  Wrong: |log($S$)|, or
|${\rm log} S$|, or anything else.  Using the \TeX\ commands will also
preclude capitalization of these operators, which is almost always incorrect.

\input boldmathfonts

\next {\it Vectors and tensors\/}.
A\&A recommends typesetting vectors and tensors with |\vec{A}| and
|\tens{B}|.  These produce bold italics and upright sans-serif characters,
i.e., $\elevenboldmath A$ and {\ss B}, respectively.

\next {\it Brackets\/}. The usual ordering of brackets is
$\left\{ \left[ \left( \dots \right) \right] \right\}$.  Only deviate
if this if there is good reason, and never use the same type for adjacent
brackets.

Distinguish angle brackets
({\tt\$\char`\\langle\$, \$\char`\\rangle\$} producing $\langle$ and
$\rangle$), often used to denote expected value) from the inequality operators
$<$ and $>$.  Note that ``{\tt <}'' and ``{\tt >}'' must never be used outside
math mode.

\next {\it Acronyms\/}. Try to avoid using an acronym as a variable
(e.g., ``$SFR = 10$\,\sol\,yr\mo''), because it is cumbersome.  Define a new
symbol instead (e.g., ``the star formation rate, ${\cal R}$'').

If you {\it do\/} use a multi-letter symbol for a variable, it {\bf must}
be in roman, e.g., ``${\rm SFR} = 10\,\Msolar$\,yr\mo,'' obtained with
``|${\rm SFR} = 10\,\Msolar$\,yr\mo|.''

\next {\it Long equations\/}. Don't try to make equations fit by using
|\small|!  Instead use |\eqnarray| or something similar to break lines.

\New
\next {\it Tall equations\/}. It looks ugly when in-line equations contain
expressions or brackets that are high enough to force \TeX\ to insert extra
space between the lines of text.  As a general guide, if it increases the line
spacing, then it's time to use a displayed equation.
\black

\next {\it $N_{\rm side}$}.  Write ``$N_{\rm side}$'' (|$N_{\rm side}$|),
not ``nside.''

\next {$\chi^2$.}  Write $\chi^2$ rather than ``chi-square.''  Whether giving
$\chi^2$ or reduced $\chi^2$, always give $N_{\rm dof}$.

\next {\it Equation references\/}. From the A\&A Author's Guide: ``All equations
that you are referring to with |\ref| must have the corresponding |\label|
--- please use this mechanism only.  Punctuate a displayed equation in the
same way as ordinary text.''  This means that displayed equations should
usually be followed by a comma or period, which generally look better preceded
by a thin space |\,| or a medium space |\>|.

\next
\New
{\it Equation numbering\/}.  Although not obligatory, it is good practice to
number all equations.  Even if you don't intend to refer to the equation
elsewhere in the paper, other people might want to do so (e.g., the referee).
\black

\next {\it Stokes parameters\/}. Since his name was ``Stokes,'' write
``Stokes $Q$,'' not ``Stoke's $Q$.''

\next {\it Spacing of exponents\/}.  Now and then, depending on characters,
exponents come out too close to the exponentiated symbol, e.g., $\nu^\beta$.
Space can be added with, e.g., |$\nu^{\,\beta}$|, giving $\nu^{\,\beta}$ or
(better) |$\nu^{\hbox{\hglue 0.7pt}\beta}$|, giving
$\nu^{\hbox{\hglue 0.7pt}\beta}$, which makes it easy to ``tune'' the space.

\next {\it Thousands separator\/}.  A\&A doesn't give a preference for how to
write numbers with many digits.  The SI standard is to use spaces rather than
commas or periods (1,000,000 or 1.000.000) to separate thousands.  Thin spaces
(|\,|) should be used, giving 1\,000\,000.  {\it Never\/} simply type a space
in the input file (|10 000|), as that would allow a line break in the middle of
the number.  Numbers with only 4 digits probably don't need a space -- use
common sense.

\next {\it Writing ``$\sigma$''}.  Write
|5\,\sigma$| ($5\,\sigma$) or possibly
|5\sigma| ($5\sigma$), but never |5-$\sigma$| (5-$\sigma$), and
{\it especially\/} never |$5-\sigma$| ($5-\sigma$).

\next
\New
{\it Derivatives}.
Derivatives should be written with italic ``{\it d\/}''s, e.g., $dy/dx$.
A\&A writes derivatives with roman ``d''s, e.g., ${\rm d}y/{\rm d}x$, and will
change $d \rightarrow $ d in the proof stage.  The reason for not following
the A\&A style is that the \TeX\ input is more straightforward and less prone
to error with italic ``{\it d\/}''s, and the \Planck\ papers will require fewer
corrections in the final edit stage.  A\&A can then make the corrections
uniformly. 
\black


\bullnumber=0

\section REFERENCES

\Modified
\next Use |Planck_bib.bib| in the aa.cls environment to ensure correct
references to the \Planck\ papers.  When referring to a \Planck\ paper in text,
use, e.g., |as described in \cite{planck2011-7.2}|, which becomes ``as
described in Planck Collaboration XX (2011).''  Alternatively some sentences
could refer to work (rather than a specific paper) ``by the Planck
Collaboration''; note that here ``the'' is not capitalized.  For a
parenthetical reference, |\citep{Planck2011-7.2}| produces ``(Planck
Collaboration XX 2011).'' \black

\next Make sure that you use the current version of |Planck_bib.bib|,
available at 

\hglue 1.5cm \maroon|http://www.sciops.esa.int/index.php?project=PLANCK&page=Repositories|\black

\next \New The bibliography style file |aat.bst| should be used, which will
give full titles (helpful for distinguishing \Planck\ papers) as well as
arXiv hyperlinks in the reference list.  This may require care over inclusion
(and/or exclusion) of appropriate packages (e.g., |ifthen|); if there's a problem, follow what
was done in a previous \Planck\ paper.
\black

\next ADS+Bib\TeX\ does not always put references in the correct
format for the journal.  Check them carefully.  Take special care with
articles in books and conference proceedings: include the title of the book,
the editors' names, the publisher and place of publication, and the page
number.  ADS+Bib\TeX+A\&A also usually omits the arXiv preprint number, which
is wrong.  This can be fixed by editing the .bib file.

\New
SPIRES Bib\TeX\ entries for \Planck\ papers should not be
used at present, since they result in the author being ``Ade et al.''\ and do
not include the identifying roman numerals.  (We're trying to get this fixed,
but in the meantime don't use them.)
\black

\next A useful reference for use of |natbib| can be found at
\maroon|http://merkel.zoneo.net/Latex/natbib.php|\black.
There are many ways to produce bad combinations of parentheses, commas,
semicolons, and years.  Beware!

One example is the incorrect
|\citep[as studied by][]{FirstPaper, SecondPaper, ThirdPaper}|, which produces
``(as studied by FirstPaper; SecondPaper; ThirdPaper).''  To obtain the
correctly punctuated ``(as studied by FirstPaper, SecondPaper, and ThirdPaper)''
requires three separate uses of |\citealt|.

\next
\New
Use a consistent format for journal abbreviations, i.e., avoid having a
mixture of ``ApJ'' and ``Astrophys.\ J.''  Best practice is to use the
built-in abbreviations, e.g., |\aap| for ``A\&A,'' |\apj| for ``ApJ,'' and
|\mnras| for ``MNRAS.''
\black

%\next Insertion of the command |\allearlypapers| will ensure inclusion of
%{\it all\/} \Planck\ early papers in the reference list, whether they are
%referenced in the paper or not.  In this way, the letter designations (2011a,
%2011b,\dots2011m\dots) will stay fixed from paper to paper.  It can be
%inserted in your input file pretty much anywhere, but right after the
%|\maketitle| command is a good place. 

%The Intermediate Results papers will not need to cite all Early Results
%papers.  Whether the |\allearlypapers| sort of command is necessary in the
%future remains to be seen. 
%There have been many problems with the Early Results papers caused by the
%fact that A\&A's standard practices were not set up to handle such a set of
%papers.  Hopefully these issues will be resolved.  This item will be updated
%when and if they are. 


\section FIGURES

\subsection General comments on figures

Figures in the cosmology and product papers must satisfy a much higher standard
of quality and uniformity than was the case with the Early Results and earlier
Intermediate papers.  To meet this high standard, requirements are given
for various types of figures.  These specifications must be followed in general
by all figures.  Exceptions can be made, but only for good reason.  

Scripts have been developed showing how to drive {\tt Python} (Andrea Zonca),
{\tt IDL} (Locke Spencer) and {\tt PGPlot} (Tim Pearson, to be posted soon)
to produce figures satisfying the requirements.  The companion paper in A\&A
La\TeX\ format ``Paperplots.tex'' and ``Paperplots.pdf'' show examples
produced by the scripts and how they are incorporated into papers.  The
scripts themselves are available at 

|http://github.com/zonca/paperplots/|

\noindent This is a public site for convenience; no \Planck\ data are used in
the scripts.  If you need help, ask Andrea, Locke, or Tim
(see |http://www.rssd.esa.int/index.php?project=IDIS&page=people|).

\New{Pay particular attention to the size of characters (Sect.~17.3.3) and
extraneous white space around figures (Sect.~17.7.2).  The most common problems
in draft figures in the 2013 papers were in these two areas.}
\black

\medskip

We start with an extract from the A\&A Author's Guide giving the sizes
of figures used in the journal and some general guidance for multi-panel
figures and captions (legends).

\subsection From the A\&A Author's Guide

``Figures submitted to the Journal must be of the highest quality to
ensure accuracy and clarity in the final published copy. We urge the
author to limit the empty space in and around figures. Artwork should
be in sharp focus, with clean, clear numbers and letters and with
sharp black lines. Thin lines should be avoided, particularly in
figures requiring considerable reduction.

``The author is warned that changes in the size and arrangement of
figures can be made by the publisher at the production stage. Because of
the bulk of the Journal [we realize this isn't English!],
the production office will reduce most
figures to fit a one-column format ($88\,$mm). If necessary, figures may
extend across the entire page width (max.\ $180\,$mm). Intermediate widths
with a side caption are also possible (max.\ $120\,$mm). The illustrations
should be placed at the top of the column and flush-left according to
layout conventions.

``If lettered parts of a figure (e.g., 1a, 1b, 1c, etc.) are referred to
in the figure legend, each part of the figure should be labeled with
the appropriate letter within the image area. Symbols should be
explained in the caption and not in the figure.

``Figure legends should concisely label and explain figures and parts of
figures.  The first sentence of each figure legend should be a
descriptive phrase, omitting the initial article (the, a, an). In
multipart figures, the legends should distinguish (a), (b), (c), etc.,
components of the figure.  Note that if parts are identified in the
legend as (a), (b), (c), particularly for single figures composed of
multiple panels, these letters should be clearly labeled in the figure
itself. Otherwise panels should be referred to by position (top right,
top left, middle, bottom, etc.). All lines (solid, dashed, dot-dashed,
dash-dotted, etc.)\ and symbols (filled or open circles, squares,
triangles, crosses, arrows, etc.) should be explained in the
legend. Graphics should not be used in figure legends.  The scientific
discussion of the table or figure contents should appear in the main
body of the article, not in the table title or figure legend.''

\Modified
In general, the first part of a caption should be a phrase ending with a period that functions as a title. This means that captions usually will {\it not\/} start with an article (``a'' or ``the'').  However, there may be exceptions, and brevity should be the rule!  Captions should certainly never start with ``This
figure shows \dots''
\black


\bullnumber=0

\subsection Line plots

\next {\it Format\/}. Output should be EPS, PDF, or other vector format
compatible with the La\TeX\ |includegraphics| command.

\next {\it Size\/}.  A\&A prints figures in three widths,
88\,mm, 120\,mm, and up to 
180\,mm.  \red Each figure should be designed for its final size, whether
single-column, side caption, or two-column\black. Simply rescaling a figure
to fit is rarely acceptable, because font sizes and line widths will be wrong.
Anisotropic scaling of a figure to make it fit is {\it never\/} appropriate.
Circles become ellipses, and letters are distorted.  Figures designed for
posters or presentations are unlikely to be appropriate for the journal.

The second paragraph of section~16.1 says ``The author is warned that \dots
the production office will reduce most figures to fit a one-column format
(88\,mm).''  Therefore, design figures for a single column unless there is a
strong reason why they must be larger.

 \New{In some cases a ``full width'' figure will fit on the page better or
look better at a smaller width than the 180\,mm maximum set by the width of
the printed page.  For example, a full-page figure may be too tall to fit if
its width is 180\,mm.  Start with 180\,mm and adjust if it seems
necessary.} \black 

\next {\it Lettering\/}.  There are differing opinions about whether labels
look better in a similar font to the paper text or in a sans-serif font.  For
consistency in \Planck\ papers the latter preference has been adopted.  Most
lettering (including axis labels) should be 
8\,pt\footnote {$^{\rm\dag}$}{In the digital publishing world,
$1\,{\rm pt} \equiv 1/72\,{\rm in} = 0.352\bar{7}\,{\rm mm}$.}
Helvetica, Arial (the Microsoft
equivalent of Helvetica), or equivalent sans-serif font, with upper-case
letters 1.9--2\,mm high {\it as printed in the journal\/}. Smaller 
or larger lettering (6\,pt--12\,pt) can be used sparingly for exponents, or
where it increases clarity.  Mathematical symbols should match as closely as
possible those used by A\&A in the text (i.e., La\TeX\ math fonts).  


\medskip\noindent

\boxit{\hbox{\heleight This is 8-pt Helvetica: Redshift 0, 1, 10, 1000.}}

\medskip\noindent

\boxit{\hbox{\helten This is 10-pt Helvetica: Redshift 0, 1, 10, 1000.}}

\medskip\noindent

\boxit{\hbox{\heltwelve This is 12-pt Helvetica: Redshift 0, 1, 10, 1000.}}


\next {\it Line widths for axes, tick marks, etc.\/} Line widths in the range 0.5--0.8\,pt 
as printed by A\&A work well.  Lines of width 1\,pt are a bit too heavy for
the boundary box of a figure, although heavier lines can be used when
appropriate for important figure elements to make a clear presentation.  In
some cases thinner lines may be needed in parts of a figure.  Quantization and
resolution effects vary from printer to printer; what is seen in test prints
may not be exactly what will be seen in A\&A.

\next {\it Axes\/}. The figure should be enclosed in a frame on all four
sides, labelled with tick marks, numerical values, and axis labels.  Text
labels should not overlap.

\next {\it Orientation of numerical values on axes\/}. Numerical values should
be oriented parallel to the axes.  The reason is straightforward.  If the
numerical values on the vertical axis are perpendicular to the axis, the space
they take up horizontally depends on the numbers.  The size of the figure box
must adjust so that the overall width of the figure is 88, 120, or \Modified{180}\black\,mm.  
The figure frame size then depends on the numerical values on the axis.  Figure
frames that should be exactly the same size will not be.  Individual images in
composites will vary in size, and be impossible to align.  The solution,
fortunately, is simple: run the numbers parallel to the axis.

\next {\it Background grid\/}.  If a grid is needed, it should be drawn with
thin lines (0.5pt?) or grey lines, not dashed or dotted lines.

\next {\it Legends\/}.  A legend within the plot identifying colours and
symbols should be included when necessary.  Opaque is preferable to
transparent, but in any case  the legend should not obscure data.


\subsection Examples of line plots

Here is a sample figure set in the three sizes used by A\&A: single-column
width (Fig.~1), 120\,mm width (Fig.~2), and two-column width (Fig.~3).
\New{For more examples, and {\tt IDL} and {\tt Python} scripts that generate
them, see Paperplots.pdf and associated scripts at}

\maroon{|http://github.com/zonca/paperplots/|}\black

% Single-column figure

\def\fcaption{Comparison of the joint power spectrum estimates from the three
CBI mosaics with the measurements from BOOMERANG, DASI, and MAXIMA.  The
rectangles indicate the 68\% confidence intervals on band-power; for
BOOME\-RANG, the solid rectangles indicate the 68\% confidence interval for
the statistical and sample variance errors, while the hatched rectangles show
the amount by which a $\pm1\sigma$ error in the beamwidth
($12\parcm9 \pm 1\parcm4$) would shift the estimates (all up or all down
together). The {\it black curve} is the joint model (see text).}


\midinsert
\global\advance\figureno by 1
\epsfxsize=88mm \epsfbox{PlanckFig_lineplot_python_88mm.pdf}
{\ninepoint\noindent F{\sc IG} \number\figureno .---\fcaption}
\endinsert


% Figure with side caption

\midinsert
\global\advance\figureno by 1
\epsfxsize=120mm \epsfbox{PlanckFig_lineplot_python_120mm.pdf}
\vskip -2in
\hglue 115mm\vtop{\hsize=42mm {\ninepoint\noindent F{\sc IG}
\number\figureno .---Figure 120\,mm wide, with caption on the
side.}}\vglue 1.7in
\endinsert


%\midinsert
%\global\advance\figureno by 1
%\epsfxsize=180mm \epsfbox{aafig_3_resized_cropped.pdf}
%{\ninepoint\noindent F{\sc IG} \number\figureno .---\fcaption}
%\endinsert

\midinsert
\global\advance\figureno by 1
\epsfxsize=180mm \epsfbox{PlanckFig_lineplot_python_180mm.pdf}
{\ninepoint\noindent F{\sc IG} \number\figureno .---\fcaption}
\endinsert

%\midinsert
%\global\advance\figureno by 1
%\epsfxsize=180mm \epsfbox{aafig_3b_resized_cropped.pdf}
%{\ninepoint\noindent F{\sc IG} \number\figureno .---\fcaption}
%\endinsert

The required La\TeX\ commands are

|\begin{figure} % Single-column figure|

|  \includegraphics[width=\hsize]{filename.eps}|

|  \caption{\fcaption} |

|  \label{fig1col}|

|\end{figure}|

\bigskip

|\begin{figure*} % Figure with side caption|

|  \sidecaption|

|  \includegraphics[width=12cm]{filename.eps} |

|  \caption{\fcaption} |

|  \label{figsidecaption}|

|\end{figure*}|

\bigskip

|\begin{figure*} % Two-column figure|

|  \includegraphics[width=18cm]{filename.eps}|

|  \caption{\fcaption} |

|  \label{fig2col}|

|\end{figure*}|


\subsection Maps 

\New{\Planck\ maps are in general ``pseudocolour'' maps, that is, they map
intensity or another quantity onto a colour scale chosen to highlight the
features of interest.  
%
%For consistency and clarity among the papers, and to help distinguish ``\Planck\ maps'' from others, we have developed two different colour scales, one suitable for low dynamic range maps (e.g., the CMB), the other suitable for high dynamic range mapes (e.g., frequency maps).  These work well for a wide range of maps showing radiation temperature, intensity, or surface brightness, and should be taken as the default. 
%
Making a pseudocolour map of a quantity $z$ involves the following steps.

\smallskip\noindent {\it Range\/}. Choose the minimum and maximum values to
display, $z_1$ and $z_2$ (values outside this range should be replaced by the
minimum or maximum as appropriate, or replaced by an ``undefined'' value).

\smallskip\noindent {\it Transfer function\/}. Apply a linear or nonlinear
transformation to map the range from $z_1$ to $z_2$ to an integer range of
colour indices (often 0 to 255; 256 different colour levels is usually
sufficient, although a larger range can be useful in some cases if the
software supports it). One very nonlinear transfer function is ``histogram
equalization;'' while this may be useful for  a first look at a map, it is
only very rarely appropriate for publication.

\smallskip\noindent {\it Colour map\/}. Choose colour values for each colour
index (R, G, B or C, M, Y, K values). Choice of a colour map requires careful
deliberation, depending on what features of the map are to be highlighted.
``Undefined'' or unmeasured pixel values should be indicated using a colour
that does not appear in the colour map (e.g., white or grey).

\smallskip

One of three standard transfer functions should be used for most map figures.

\bull The transfer function/colour map used in figure~15 of {\it Planck 2013
results.~I\/}.  This is suitable for maps with low dynamic range and zero mean
(e.g., the CMB).

\bull The transfer function/colour map used in figures~9 and 10 of
{\it Planck 2013 results.~I\/}.  This is suitable for maps with high dynamic
range and zero mean (e.g., the \Planck\ frequency maps).  It can be used for
$I$, $Q$, and $U$ maps at all frequencies (but note that different scalings
are needed for the $I$ maps at 545 and 853\,GHz, which have different units).

\bull A one-sided version of the high dynamic range transfer function in the
previous item, suitable for maps of polarization amplitude $P$ or foregrounds,
quantities which are inherently positive.  

\smallskip

Details of these, with code for use in {\tt IDL} and {\tt Python}, can be found
in Paperplots.pdf and the code repository (specify).

If the quantity being mapped is quite different, a completely different colour
scheme to set it apart may be appropriate.  For example, the ``lensing
potential map of the Universe'' in the 2013 mission paper (Planck 2013 results.
I.), figure~18.  These should be discussed with the EB as early as possible to
ensure consistency between papers.  As far as possible, similar maps should be
displayed using the same pseudocolour display across all papers. 

\smallskip\noindent {\it Projection\/}.  Full-sky maps should normally be
displayed in Mollweide projection in Galactic coordinates.  Sky-patch maps
should usually be displayed in the orthographic (tan) projection.  When several
patches are displayed together, they should be on the same scale (mm per
degree) even if they cover different areas.  Maps from different sources
intended for direct comparison should always have the same sky coverage, scale,
and projection; this is best achieved by plotting them with the same graphics
program.

\smallskip\noindent {\it Resolution\/}.
Rendering a map for display on screen or paper involves repixelization.
Typically the printer will use no more than 300 pixels per inch.  Beware of
information loss in this process.  If the original {\tt HEALPix} pixels are
smaller than the print pixels, it is advisable to smooth the map before
rendering.  The visual effect of even mild smoothing can be dramatic.
Any smoothing should be noted in the caption.

\smallskip\noindent {\it Annotation\/}.  Annotation should be added to the map
image using a vector-based graphics program, preferably in a separate layer.
All maps need proper annotation, including colour scale (always) and scale and
location (except for the standard Mollweide all-sky view in Galactic
coordinates centred on the Galactic centre).  It should be possible to read
off the sky location and pixel value.  The caption for a figure containing a
map or maps should state clearly the source of the data (including the name of
the file in the PLA when appropriate), the projection, and the coordinate
system.  For large-area maps in which the lines of longitude and latitude are
strongly curved, a coordinate grid should be superimposed in the map.  For
small areas, it may be sufficient to label the coordinates around the edge.
The colour bar should be labelled in such a way that the value corresponding
to specific colour can be read off precisely.  In general, and definitely
for a nonlinear mapping, this requires more than just labelling the end points.
} \black

See Paperplots.pdf and associated scripts at |http://github.com/zonca/paperplots/| .

\subsection Requirements for figures with shading 

See Paperplots.pdf and associated scripts at |http://github.com/zonca/paperplots/| .


\bullnumber = 0

\subsection Additional points

\next {\it File formats\/}.  Use graphics software that produces scaleable,
vector-graphic output, such as EPS, PDF, or \hbox{SVG}.  Do not use bitmap
formats (e.g., PNG, JPEG) except for photographs or images without annotation.
If an image needs to be annotated, the bitmap should be embedded in a vector
format such as \hbox{PDF}.  Never convert the annotation to a bitmap format.
If it is impossible to use a scalable vector format, then use a non-lossy,
high-resolution ($>300$\,pixels\,inch\mo) bitmap format such as \hbox{TIFF}.
Do not use JPEG: it will blur the annotation.

\New
\next {\it File sizes\/}.  Giant files are awkward to deal with, and may make
submission to the journal and the arXiv difficult.  There is no specific limit
to the number of MB in any particular file, but general guidelines are: not to
use more pixels per inch than needed in images; and to apply (modest) JPEG
compression if necessary (but do not apply JPEG compression to line drawings
or to the annotation in images).  As a specific example, it is not possible to
represent all the information in an $N_{\rm side}=2048$ image in a journal
figure, so you shouldn't try.  It may be a good idea to smooth images before
downsampling them.  Note also that PostScript is an inefficient format for
figures; conversion to PDF will usually reduce the file size
(without loss) quite effectively.
\black

\next {\it Bounding Boxes\/}.  If you are working with an EPS or PDF file,
make sure that the bounding box is set correctly within the file.  The bounding
box should be the minimum required to enclose the visible parts of the figure.
If you have several figures that should all be displayed on the same scale,
make sure they have identical bounding boxes.
\New{The Python and IDL scripts described and used in Paperplots.pdf and found
at |http://github.com/zonca/paperplots/| show how to do this.  As a last
resort, white space can be cropped out one figure at a time with {\tt Preview}
on Apple computers (click on the dotted square in the tool bar, draw a box
tight to the visible parts of the figure, copy, paste into a new file, and
save) or with the ``Document Setup/Edit Artboards'' tool in
{\tt Illustrator}.}\black

\next {\it More on fonts\/}.  Some graphics software packages use outline or
``Hershey'' fonts designed for use with pen-plotters.  They usually include a
sans-serif font that looks similar to Helvetica, but you may need to adjust
the line-thickness as well as the character height to get characters of
appropriate weight.  Typically the line-thickness should be about 1/10 of the
character height  {\it to be checked}.

\next {\it Tick marks and numeric labels\/}. Have tick marks projecting out of
the frame only if it is necessary to make them visible.  Use sensible
(rational) tick separations.  Choose units to get numbers without big
exponents.  Avoid overlapping labels.  At least two, and preferably more than
two labels should appear on each axis.  Do not use more significant figures
than are needed in labels (i.e., 10, not 10.00 or 9.999).  It is not necessary
to have the same number of decimal places in axis labels.  For example,
|0.01, 0.1, 1, 10, 100| is better than |0.01, 0.10, 1.00, 10.00, 100.00|.
Avoid unnecessary trailing zeros.

Sky images should always have coordinates indicated by a labelled frame or
graticule.  Use sexagesimal notation (e.g., h,m,s of RA and d,m,s of Dec) or
decimal degrees, but do not mix the two: if you refer to a source RA of
$12^{\rm h}\,30^{\rm m}$ in the text or a table, do not use 187\pdeg5 in a
figure!

\next {\it Captions and titles\/}.  Figures have captions.  They {\it should
not\/} have titles or anything else above the frame of the figure.

\next {\it Axis labels\/}. Label axes with the name or description of the
quantity plotted, possibly a symbol, and units if applicable, with the units
in square brackets, e.g., {\ss Detector temperature $T_{\rm det}$ [$\mu$K]},
{\ss Multipole order $\ell$}, or {\ss Length [mm]}.  However, if logarithms
are involved, {\ss Log(length/mm)} is preferable.  {\ss Detector volts [V]}
is incorrect; it should be {\ss Detector reading [V]} or possibly
{\ss Detector voltage [V]}.  In labels as elsewhere (Sect.~10.8), use exponents,
not fractions in units: {\ss km\,s$^{-1}$}, not {\ss km/s}.  Don't use a dot
to separate units ({\ss K\,km\,s$^{-1}$}, not {\ss K.km\,s$^{-1}$}).  Don't
invent units for dimensionless quantities (e.g., say {\ss Normalized hit
count}, not {\ss Hit Count [norm.]})  Capitalize labels as normal text (first
letter and proper names only).  Do not use ``{\ss \#}'' as a synonym for
``{\ss number}.''

\next {\it Lettering in figures\/}.  Lettering should be in lower-case type,
with the first letter capitalized and no full stop.  Layering type directly
over shaded or textured areas and using reversed type (white lettering on a
coloured background) should be avoided where possible.

\next {\it Colour\/}.  In general, use colour only if it adds significantly to
the scientific clarity of the figure.  For example, use colour rather than grey
and dashed lines if there are many components to be distinguished (more than
three or four, say).  Always provide a colour bar scale for grey-scale or
pseudo-colour images.  A graphical legend in the figure, or labelling curves
individually, is better than describing elements in the caption (avoid ``the
green dotted line'' etc.; the reader may be viewing a black-and-white copy of
the paper, or may be colour blind).  

\next {\it Consistency of style and colour across figures\/}.  Use a uniform
style for figures throughout the paper.  If a quantity is represented by a red
dashed line in one figure, use the same style for it in other related figures.  

\next {\it Multipanel figures\/}.  For a large set of similar figures (e.g.,
SEDs for many objects), use a multipanel figure, which can extend over multiple
pages if necessary.  If there are more than two or three panels, it is usually
best to identify each panel internally (e.g., with the object name) rather
than in the caption.  Avoid phrases like ``middle of third row'' or ``lower
right'' in the text or caption.  Use ``(a),''  ``(b),'' $\dots$ labels if
necessary.  In case two panels are referenced in the caption, use
``{\it Left\/}: \dots,'' ``{\it Right\/}: \dots ,'' ``{\it Upper\/}: \dots,''
``{\it Lower\/}: \dots,''
etc.  To achieve this, type, e.g., |\emph{Left}:| \dots.
Note that the colon is outside the brackets.

\next {\it Time axes in days\/}.  Time should be plotted in days rather than
operational days (since an operational day is well-defined operationally, but
it is not a well-defined time interval).  Time axes labelled with dates are
typically quite hard to read, and should be avoided.

\next {\it SEDs\/}.  SEDs should be plotted with frequency increasing to the
right.  If you insist on plotting against wavelength, have it increase to the
left.  

N.B. --- This instruction was not followed in some Early Results papers, on
the grounds that long-established traditions in sub-fields of astrophysics
used different conventions.  Time was short, and we didn't insist.  However,
consistency across \Planck\ papers takes precedence.  For the Intermediate
Results and Cosmology and Product Papers, we will insist. 

\next {\it Mathematical symbols in figures\/}.  Mathematical symbols in
figures should match those in the text. Use \TeX/La\TeX\ to make the symbols
and paste them into the figure if necessary.

\next {\it Geometric distortion\/}.  {\bf Do not} scale the height and width of a figure independently to make it fit a spot in a paper.  The geometrical distortion introduced is unacceptable.  The consequences are particularly undesirable for Mollweide projections, whose geometrical characteristics are ruined.

\page

\bullnumber=0

\section TABLES

\subsection General comments on tables

La\TeX\ packages, including A\&A's aa.cls, try to simplify tables by writing
elaborate overlays for |\halign|, a basic \TeX\ command.    Unfortunately,
they introduce their own complexity at the same time, as they reduce
flexibility and make it harder to do many things that are important for the
readability of tables, things that are simple enough in |\halign|.  Overall,
it's easier and better to just use |\halign|.  

|\halign| works just fine inside the table environments of aa.cls.  For a
one-column table, use the |\begin{table}|\dots|\end{table}| environment.  For
a two-column table, use the
|\begin{table*}|\dots||\goodbreak\noindent\dots|\end{table*}| environment.
A template table file is given in |PlanckTable.tex|, which can be found at

\maroon |http:\\www.sciops.esa.int/index.php?project=PLANCK&page=Repositories| \black .

A tutorial on how to use |\halign| to produce tables for \Planck\ is given in
Sect.~18.2.  Before that, we list some things to do and not to do in tables.

\next
\Modified
Excessive numbers of digits should not be given.  In general, the number
of digits used for a quantity should be driven by the uncertainty on that
quantity; in many cases a
single digit on uncertainties is sufficient.  We will follow the specific
policy of the Particle Data Group, described on page~5 of

\maroon |http://pdg.lbl.gov/2013/reviews/rpp2013-rev-rpp-intro.pdf|\black, 

\Modified
\noindent which says ``The basic rule states that if the three
highest-order digits of the error lie between 100 and 354, we round to two
significant digits.  If they lie between 355 and 949, we round to one
significant digit.  Finally, if they lie between 950 and 999, we round up to
1000 and keep two significant digits.  In all cases, the central value is given
with a precision that matches that of the error.  So, for example, the result
$0.827\pm0.119$ would appear as $0.83\pm0.12$, while
$0.827\pm0.367$ would turn into $0.8\pm0.4$.''

There {\it may\/} be cases where using more precision is justified, e.g.,
when there is an important scientific point to make by comparing uncertainties.
However, this should be the exception, and there should be a clear rationale
for using increased numbers of digits.
\black

\next In Sect.~15, item~4 we emphasized that minus signs can only be typeset in
math mode, including in tables.  One of the most pervasive mistakes in tables
in the Early Results papers was the use of hyphens instead of minus signs in
tables.  {\bf Don't do it!}

\next Footnotes in tables should be identified by roman superscript letters,
which can be set with |\tablefootmark{}| and |\tablefoottext{}|.  Better yet, make tables with |\halign| as described below starting in Sect.~18.2, and use the prescription given in Sect.~18.15.

\next Don't use check marks in tables as a substitute for ``yes'' or ``Y.''
It's tacky.  Better still, don't use boolean columns in tables at all.

\next Missing values should be indicated by an ellipsis, ``\dots,'' obtained
most easily with the |\dots| command defined in Planck.tex, which works in
both text and math modes.  Don't try to produce an ellipsis with periods (full
stops).  The spacing will be wrong...

\next When entries in a column are to be compared, align the decimal points in
both the quantity and the uncertainty (Example 1 in Table~2).  When entries in
a column are not to be compared, centre the entries and uncertainties in the
column.  Do not align the decimal point (Example~2 in Table~2).  {\bf Note
that the style and placement of the table numbers and captions in the examples
in this section are not those used by \hbox{A\&A}.}  The A\&A scheme is better
suited to two-column output and gives weird-looking results in the
single-column format we use here.  The |\caption| command from A\&A La\TeX\
used in the |PlanckTable.tex| template file produces results in the standard
A\&A format.  A\&A prefers the caption to be ''as concise as possible so all
additional description should be displayed in a footnote to the table.''  As
you may have noticed in the 2013 results papers, the copy editors will change
long captions to title phrases and put the rest in the Notes.  Since it's
easier for us to check footnotes to tables if they are separate, and A\&A will
change to their preferred format at the end, follow the prescription for
footnotes given in Sect.~18.15.


\midinsert
\begingroup
\ninepoint
\centerline{\bf Table 2.} 
\centerline{\csc Decimal Point Alignment}
\vglue -10pt
\setbox\tablebox=\vbox{
   \newdimen\digitwidth 
   \setbox0=\hbox{\rm 0} 
   \digitwidth=\wd0 
   \catcode`*=\active 
   \def*{\kern\digitwidth}
%
   \newdimen\decimalwidth 
   \setbox0=\hbox{$.0$} 
   \decimalwidth=\wd0 
   \catcode`!=\active 
   \def!{\kern\decimalwidth}
%
\halign{\hbox to 1.4in{#\leaderfil}\tabskip=0.75em&
   \hfil#\hfil\tabskip=0pt\cr
\noalign{\doubleline}
\omit&30\,GHz\cr
\noalign{\vskip 5pt\hrule\vskip 5pt}
\omit\hfil Example 1\hfil\cr
\noalign{\vskip 4pt}
$\Delta T_{30}$ [$\mu$K]          &$38.6\pm13.2$\cr
$\Delta T_{30}/T$                 &$*1.7\pm*0.7$\cr
\omit&\cr
\omit\hfil Example 2\hfil\cr
\noalign{\vskip 4pt}
$\Delta T/T$ per pixel            &$(5.9\pm0.6)\times10^{-3}$\cr
$N_{\rm feeds}$                   &$1.2\pm0.2$\cr
\noalign{\vskip 5pt\hrule\vskip 3pt}}}
\endPlancktable

\endgroup
\endinsert


\next Tables should have a double rule at the top, a single rule between
headings and the body of the table, and a rule at the bottom.  See Sect.~18.9.

\next The spacing of lines in tables often requires no special attention, at
least when there is no complicated structure.  However, one case that occurs
fairly frequently in \Planck\ papers and requires special attention is that of
table values with asymmetrical errors (Table~3).  Extra space in needed and
easily obtained in this example with |\noalign{\vskip 4 pt}| between the lines
(see Sect.~18.7).

\midinsert
\begingroup
\ninepoint
\centerline{\bf Table 3.} 
\centerline{\csc Spacing With Asymmetric Errors}
\vglue -25pt
\setbox\tablebox=\vbox{
   \newdimen\digitwidth 
   \setbox0=\hbox{\rm 0} 
   \digitwidth=\wd0 
   \catcode`*=\active 
   \def*{\kern\digitwidth}
%
   \newdimen\signwidth 
   \setbox0=\hbox{+} 
   \signwidth=\wd0 
   \catcode`!=\active 
   \def!{\kern\signwidth}
%
\halign{\hbox to 1in{#\leaderfil}\tabskip=1.0em&\hfil#\hfil&\hfil#\hfil&\hfil#\hfil&\hfil#\hfil&\hfil#\hfil&\hfil#\hfil&\hfil#\hfil&\hfil#\hfil\tabskip=0pt\cr 
\noalign{\vskip 5pt}
\noalign{\doubleline}
\omit& $r_{x}$\cr
\omit\hfil Sector\hfil &[Mpc]& $D_n$& $M_n$& $D_T$& $M_T$& $D_n\times D_T$& $M_{nT}$& $M_{\rm SZ}$\cr 
\noalign{\vskip 3pt\hrule\vskip 5pt}
\multispan9\hfil{\bf Bad}\hfil\cr
\noalign{\vskip 5pt}
West     &$1.173^{+0.0003}_{-0.003}$ & $2.00^{+0.03}_{-0.03}$ & $1.73^{+0.03}_{-0.03}$ &$3.0^{+0.7}_{-0.6}$  & $2.6^{+0.4}_{-0.4}$  & $6.0^{+1.4}_{-1.1}$&$2.3^{+0.2}_{-0.2}$&$1.95^{+0.45}_{-0.02}$\cr
Southeast&$0.9778^{+0.0002}_{-0}$    & $2.43^{+0.02}_{-0.02}$ & $2.10^{+0.01}_{-0.01}$ &$1.3^{+1.8}_{-0.6}$ & $1.3^{+1.3}_{-1.3}$  & $3.1^{+1.6}_{-1.1}$&$1.6^{+0.3}_{-0.1}$&$2.03^{+0.14}_{-0.04}$\cr
\noalign{\vskip 5mm}
\multispan9\hfil {\bf Good}\hfil\cr
\noalign{\vskip 5pt}
West     &$1.173^{+0.0003}_{-0.003}$ & $2.00^{+0.03}_{-0.03}$ & $1.73^{+0.03}_{-0.03}$ &$3.0^{+0.7}_{-0.6}$  & $2.6^{+0.4}_{-0.4}$  & $6.0^{+1.4}_{-1.1}$&$2.3^{+0.2}_{-0.2}$&$1.95^{+0.45}_{-0.02}$\cr
\noalign{\vskip 4pt}
Southeast&$0.9778^{+0.0002}_{-0}$    & $2.43^{+0.02}_{-0.02}$ & $2.10^{+0.01}_{-0.01}$ &$1.3^{+1.8}_{-0.6}$ & $1.3^{+1.3}_{-1.3}$  & $3.1^{+1.6}_{-1.1}$&$1.6^{+0.3}_{-0.1}$&$2.03^{+0.14}_{-0.04}$\cr
\noalign{\vskip 5pt\hrule\vskip 3pt}}}
%\endPlancktable                    % ends one-column \halign
\endPlancktable                 % ends two-column \halign
\endgroup

\endinsert


\next Blank spaces in the input La\TeX\ file are often ignored, and adding
spaces in table input files can improve the human readability of a table.  But
there is one case in tables to watch out for.   In |\halign|, spaces following
ampersands before a table entry are ignored; however, {\it spaces between a
table entry and the next ampersand are {\bf not} ignored}, and change the
spacing of column entries.  Consider the following lines, used in Table~4.  

\goodbreak

|Quantity&15&4&10&8\cr|

|Quantity& 15& 4& 10& 8\cr|

|Quantity& 15 & 4 & 10 & 8 \cr|

\medskip

\midinsert
\begingroup
\ninepoint
\centerline{\bf Table 4.} 
\centerline{\csc Some Blanks Matter}
\vglue -10pt
\setbox\tablebox=\vbox{
   \newdimen\digitwidth 
   \setbox0=\hbox{\rm 0} 
   \digitwidth=\wd0 
   \catcode`*=\active 
   \def*{\kern\digitwidth}
%
   \newdimen\decimalwidth 
   \setbox0=\hbox{$.0$} 
   \decimalwidth=\wd0 
   \catcode`!=\active 
   \def!{\kern\decimalwidth}
%
\halign{\hbox to 1.0in{#\leaderfil}\tabskip=0.75em&
   \hfil#\hfil&
   \hfil#\hfil&
   \hfil#\hfil&
   \hfil#\hfil\tabskip=0pt\cr
\noalign{\doubleline}
Quantity&15&4&10&8\cr
Quantity& 15& 4& 10& 8\cr
Quantity& 15 & 4 & 10 & 8 \cr
\noalign{\vskip 5pt\hrule\vskip 3pt}}}
\endPlancktable

\endgroup
\endinsert


\noindent The first two lines produce identical output, but the
third line produces different output.  Given the importance of precise
alignment of column entries (as discussed above in Table~2), the rule is
{\bf no spaces before ``|&|''s and ``|\cr|''s}.


\subsection Tutorial on Tables

\noindent The best way to construct tables for astronomical journals is to use
|\halign|.  The best way to use |\halign| for \Planck\ tables is the following:

\bull Insert the file |PlanckTable.tex| into your La\TeX\ input file to start
a table.

\bull Set up the ``template'' line.  If you haven't used the
$\backslash${\tt halign} command before, look through the following examples,
find what you want to do, and copy the instructions used to produce it.

\bull Fill in the table information, following the relevant examples.

\bull Admire your table.  (N.B.: A\&A has some strange notions about spacing
in tables, whether it's the space above and below ``rules,'' between columns,
or at the margins.  They may change your carefully constructed table in the
typesetting process.  Nevertheless, experience with the earlier papers shows
that they leave the basic structure of the table alone, so it's worth a good
deal of effort to develop a good structure.)

\bull If you need help, call or email Charles Lawrence (+1 818 642 1784,
charles.lawrence@jpl.nasa.gov).

\vskip 2.3cm\noindent
Here's what's in PlanckTable.tex\,:
\bigskip
\begingroup\obeylines

|\begin{\table}[tmb]   or  % use table for a one-column table|
|\begin{table*}[tmb]       % use table* for a two-column table|
|\begingroup               % this + \endgroup at the end keep table things local |
|\newdimen\tblskip \tblskip=5pt|
|\caption{Table caption goes here.}|
|\nointerlineskip|
|\vskip -3mm|
|\footnotesize              % good font size for a table, but can be changed|
|\setbox\tablebox=\vbox{    %|
|   \newdimen\digitwidth       % see \S\,18.14 for the purpose of the next 10 lines|
|   \setbox0=\hbox{\rm 0} |
|   \digitwidth=\wd0 |
|   \catcode`*=\active |
|   \def*{\kern\digitwidth}|
|%|
|   \newdimen\signwidth |
|   \setbox0=\hbox{+} |
|   \signwidth=\wd0 |
|   \catcode`!=\active |
|   \def!{\kern\signwidth}|
|%|
|\halign{                   % template goes here.  See examples.|
|\noalign{\doubleline}|
\hglue 0pt|                           % heading goes here.  See examples.|
|\noalign{\vskip 3pt\hrule\vskip 5pt}|
\hglue 0pt|                           % table lines go here.  See examples.|
|\noalign{\vskip 5pt\hrule\vskip 3pt}}}|
|\endPlancktable      or         % for a one-column table; defined in Planck.tex|
|\endPlancktablewide             % for a two-column table; defined in Planck.tex|
|\endgroup|
|\end{table}          or| 
|\end{table*}|

\endgroup


\subsection Columns

Here is a skeleton three-column table: 

\begingroup
\obeylines
|\halign{ & & \cr|
|apples&bears&yams\cr|
|oranges&elephants&corn\cr|
|watermelons&brontosauruses&rutabagas\cr| 
|peaches&llamas&peas\cr|
|}|
\endgroup

\bigskip
\noindent
This produces
\bigskip

\halign{#&#&#\cr
apples&bears&yams\cr
oranges&elephants&corn\cr
watermelons&brontosauruses&rutabagas\cr
peaches&llamas&peas\cr
}


\bigskip

The width of each column is the width of the widest entry in that column.

\subsection Centring, etc.~in columns

%(From the OED: ``[f. CENTRE v. + -ING; the standard spelling (on the analogy of settle, etc.) is now centring, but as the word is of 3 syllables (in careful pronunciation), centering (more rarely centreing) has freq. been used, esp. in technical senses.]''  Thus both spellings are acceptable.  For \Planck\, we'll use the spelling that doesn't need ``careful pronunciation,'' i.e., ``centering.'')

To left justify entries in a column, use |#\hfil| as the template (this is the
default).  To centre entries in a column, use |\hfil#\hfil|.  To right
justify, use |\hfil#|.  Modifying the example in Sect.~18.2, 

\bigskip

\begingroup
\obeylines
|\halign{#\hfil&|
\hglue 4em|\hfil#\hfil&|
\hglue 4em|\hfil#\cr|
|apples&bears&yams\cr|
|oranges&elephants&corn\cr|
|watermelons&brontosauruses&rutabagas\cr| 
|peaches&llamas&peas\cr|
|}|
\endgroup

\bigskip
\noindent
produces
\bigskip

\halign{#\hfil&\hfil#\hfil&\hfil#\cr
apples&bears&yams\cr
oranges&elephants&corn\cr
watermelons&brontosauruses&rutabagas\cr
peaches&llamas&peas\cr
}

\bigskip
\noindent
Alternatively,
\bigskip


\begingroup
\obeylines
|\halign{\hfil#&|
\hglue 0em|        \hfil#\hfil&|
\hglue 0em|        #\hfil\cr|
|apples&bears&yams\cr|
|oranges&elephants&corn\cr|
|watermelons&brontosauruses&rutabagas\cr| 
|peaches&llamas&peas\cr|
|}|
\endgroup

\bigskip
\noindent
produces
\bigskip

\halign{\hfil#&\hfil#\hfil&#\hfil\cr
apples&bears&yams\cr
oranges&elephants&corn\cr
watermelons&brontosauruses&rutabagas\cr
peaches&llamas&peas\cr
}

\bigskip\noindent

\bigskip

Notice that the template line in the |\halign| command is broken into separate
lines following the ampersands (|&|s).  Spaces following an |&| in the input
file are discarded, i.e., they have no effect on the output.  As will be seen
below, this can also be exploited to line up column entries in the input file
for easier reading by humans.

\subsection Rules

Horizontal lines, ``rules'' in typesetting parlance, set off the top and bottom
of tables, and separate headings from table entries.  The following example
shows how to produce rules, and also how to use the template specification to
do a couple of other things.

\bigskip
\begingroup\obeylines
|\halign{\it#\hfil&|
\hglue 0em|        \hfil# and mice\hfil&|
\hglue 0em|        \sl\hfil#\cr|
|\noalign{\hrule}|
|apples&bears&yams\cr| 
|oranges&elephants&corn\cr| 
|watermelons&brontosauruses&rutabagas\cr| 
|peaches&llamas&peas\cr|
|\noalign{\hrule}|
|}|
\endgroup

\bigskip


\halign{\it#\hfil&
        \hfil# and mice\hfil&
        \sl\hfil#\cr 
\noalign{\hrule} 
apples&bears&yams\cr 
oranges&elephants&corn\cr 
watermelons&brontosauruses&rutabagas\cr 
peaches&llamas&peas\cr 
\noalign{\hrule} 
} 

\bigskip\noindent
See Sect.~18.8 for more on rules after headings are introduced.

\subsection Space between columns

In general, space between columns is needed.  The best way to do this is with
``tabskip glue.''   Tabskip glue is added before the first column, between all
columns (in other words, where the ampersands are), and after the last column.
If you haven't specified |\tabskip= \dots| then the default value of zero is
used.  The value of tabskip in effect when \TeX\ reads the |{| following
|\halign| will be used before the first column (i.e., at the left edge of the
table); the value in effect when \TeX\ reads the |&| after the template for the
first column will be used between the first and second columns, and so on.  The
value of tabskip in effect when \TeX\ reads the |\cr| after the template for
the last column will be used after the last column.  The most general
specification is, e.g., |\tabskip=7pt plus7pt minus7pt| (allowing the glue
between columns to stretch and shrink if necessary), and can appear any number
of times in the template line. 

Consider the following two examples:
\bigskip

\begingroup\obeylines
|{\tabskip=2em           %\tabskip is inside brackets so its effect ends with the table|
|\halign{\it#\hfil&|
\hglue 0pt|        \hfil# and mice\hfil&|
\hglue 0pt|        \sl\hfil#\cr|
|\noalign{\hrule}|
|apples&bears&yams\cr| 
|oranges&elephants&corn\cr| 
|watermelons&brontosauruses&rutabagas\cr| 
|peaches&llamas&peas\cr|
|\noalign{\hrule}|
|}|
|}|

\endgroup

\bigskip\noindent produces

\bigskip


{\tabskip=2em
\halign{\it#\hfil&
        \hfil# and mice\hfil&
        \sl\hfil#\cr 
\noalign{\hrule} 
apples&bears&yams\cr 
oranges&elephants&corn\cr 
watermelons&brontosauruses&rutabagas\cr 
peaches&llamas&peas\cr 
\noalign{\hrule} 
} 
}
% 
% 
\bigskip\noindent
Space like this at the left and right margins is often seen in tables in
astronomical journals, although not as exaggerated as here.   Such space is
produced with standard A\&A table commands.  It's a matter of taste whether
this is considered good or bad (see comment about A\&A and tables in
Sect.~18.1), but you have no control over it in the A\&A table style.

\bigskip\noindent
In fact, it is easy to get control of this space, as follows:
\bigskip
\begingroup\obeylines
|\halign{\it#\hfil\tabskip=2em&|
\hglue 0pt|        \hfil# and mice\hfil&|
\hglue 0pt|        \sl\hfil#\/\tabskip=0pt\cr| 
|\noalign{\hrule}|
|apples&bears&yams\cr| 
|oranges&elephants&corn\cr| 
|watermelons&brontosauruses&rutabagas\cr| 
|peaches&llamas&peas\cr|
|\noalign{\hrule}|
|}|
\endgroup

\bigskip\noindent
produces

\bigskip 
\halign{\it#\hfil\tabskip=2em&
        \hfil# and mice\hfil&
        \sl\hfil#\/\tabskip=0pt\cr 
\noalign{\hrule} 
apples&bears&yams\cr 
oranges&elephants&corn\cr 
watermelons&brontosauruses&rutabagas\cr 
peaches&llamas&peas\cr 
\noalign{\hrule} 
}

\bigskip\noindent
Much better!

\subsection Space between rows
 
Use |\openup 4pt| before the |\halign| to increase the space between all lines
by 4\,pt. 

\bigskip\noindent
\begingroup\obeylines

|{\openup 4pt                            % Notice that \openup is inside brackets|
|\halign{\it#\hfil&|
\hglue 0pt|        \hfil# and mice\hfil&|
\hglue 0pt|        \sl\hfil#\cr| 
|\noalign{\hrule}|
|apples&bears&yams\cr|
|oranges&elephants&corn\cr|
|watermelons&brontosauruses&rutabagas\cr| 
|peaches&llamas&peas\cr| 
|\noalign{\hrule}| 
|}                                       % This ends the \halign|
|}                                       % This closes the \openup 4pt group|
\endgroup



\bigskip\noindent
Usually, however, one wants to increase the space between some lines in the
table, but not all (not heading lines, for example).  The following will
{\it not\/} work, because space between lines isn't set until the {\it end\/}
of the |\halign|, and by then the |\openup-3pt| will have cancelled the
|\openup+3pt|:

\bigskip\noindent
\begingroup\obeylines
|\halign{...template...\cr| 
|first line\cr|
|\noalign{\openup 3pt}|
|second line\cr|
|third line\cr|
|\noalign{\openup -3pt}|
|fourth line\cr|
|etc.|
|}| 
\endgroup

\bigskip\noindent
Instead, add space explicitly where it is needed with |\noalign{\vskip ?pt}|
between lines.  For example,


\bigskip
\begingroup\obeylines
|\halign{\it#\hfil\tabskip=2em&|
\hglue 0pt|        \hfil# and mice\hfil&|
\hglue 0pt|        \sl\hfil#\tabskip=0pt\cr| 
|\noalign{\hrule}|
|apples&bears&yams\cr| 
|\noalign{\vskip 10pt}|
|oranges&elephants&corn\cr| 
|watermelons&brontosauruses&rutabagas\cr| 
|\noalign{\vskip 5pt}|
|peaches&llamas&peas\cr|
|\noalign{\hrule}|
|}|
\endgroup

\bigskip\noindent
produces

\bigskip

\halign{\it#\hfil\tabskip=2em&
        \hfil# and mice\hfil&
        \sl\hfil#\tabskip=0pt\cr 
\noalign{\hrule}
apples&bears&yams\cr 
\noalign{\vskip 10pt}
oranges&elephants&corn\cr
watermelons&brontosauruses&rutabagas\cr 
\noalign{\vskip 5pt}
peaches&llamas&peas\cr
\noalign{\hrule}
}

\bigskip

\subsection How to override the template for a particular entry

Suppose you want to add column headings to the table above, but you want the
headings in boldface and centred over each column.  The command |\omit| at the
beginning of a given entry tells \TeX\ to {\it not\/} use the template for that
column. but to use what you are about to tell it instead.  To take a common
example, suppose you want to centre a heading over a column for which the
template specifies left or right justification:

\bigskip

\begingroup\obeylines

|\halign{\it#\hfil\tabskip=02em&|
\hglue 0pt|        \hfil# and mice\hfil&|
\hglue 0pt|        \sl\hfil#\tabskip=0pt\cr| 
|\omit\bf\hfil Fruits\hfil&\omit\bf\hfil Animals\hfil&\omit\bf\hfil Vegetables\hfil\cr| 
|apples&bears&yams\cr|
|oranges&elephants&corn\cr| 
|watermelons&brontosauruses&rutabagas\cr| 
|peaches&llamas&peas\cr|
|}|

\bigskip\noindent which gives

\endgroup
\bigskip

\halign{\it#\hfil\tabskip=02em&
        \hfil# and mice\hfil&
        \sl\hfil#\tabskip=0pt\cr 
\omit\bf\hfil Fruits\hfil&\omit\bf\hfil Animals\hfil&\omit\bf\hfil Vegetables\hfil\cr % headfngs % in boldface, centred apples&bears&yams\cr 
oranges&elephants&corn\cr 
watermelons&brontosauruses&rutabagas\cr 
peaches&llamas&peas\cr 
}


\bigskip\noindent
Note that although the centre column is already centred by the template, using
|\omit| also prevents template addition of ``and mice'' to the heading.  


\subsection More on rules

\Planck\ tables should have a double rule at the top, a single rule between
headings and the body of the table, and a rule at the bottom.  Here's how to
do it:

\bigskip
\begingroup\obeylines

|\halign{\it#\hfil\tabskip=02em&|
\hglue 0pt|        \hfil# and mice\hfil&|
\hglue 0pt|        \sl\hfil#\tabskip=0pt\cr| 
|\noalign{\doubleline}|
|\omit\bf\hfil Fruits\hfil&\omit\bf\hfil Animals\hfil&\omit\bf\hfil Vegetables\hfil\cr|
|\noalign{\vskip 5pt\hrule\vskip 5pt}|
|apples&bears&yams\cr| 
|oranges&elephants&corn\cr|
|watermelons&brontosauruses&rutabagas\cr| 
|peaches&llamas&peas\cr|
|\noalign{\vskip 3pt\hrule}|
|}|
\endgroup

\bigskip

This uses the convenient |\doubleline| command defined in Planck.tex\,:
\medskip
|\def\doubleline{\vskip 3pt\hrule \vskip 1.5pt \hrule \vskip 5pt}|
\bigskip\noindent
The output begins to have a familiar look:


\bigskip

\halign{\it#\hfil\tabskip=02em&
        \hfil# and mice\hfil&
        \sl\hfil#\tabskip=0pt\cr 
\noalign{\doubleline}
\omit\bf\hfil Fruits\hfil&\omit\bf\hfil Animals\hfil&\omit\bf\hfil
 Vegetables\hfil\cr
\noalign{\vskip 5pt\hrule\vskip 5pt}
apples&bears&yams\cr 
oranges&elephants&corn\cr 
watermelons&brontosauruses&rutabagas\cr 
peaches&llamas&peas\cr 
\noalign{\vskip 3pt\hrule}
}


\subsection Centring a table

One way to centre a table is to use the |\center| (La\TeX\ uses American
spelling!) environment inside the |table| or |table*| environment:

\bigskip
\begingroup\obeylines

|\begin{table*}[tmb]|
|\begin{center}|
|\caption{Sensitivity budget for LFI units.}|
\hglue 0pt
\hglue 0pt|(\halign stuff goes in here)|
\hglue 0pt
|\end{center}|
|\end{table*}|

\endgroup

\bigskip\noindent
A\&A doesn't centre the table number and title over the table, but rather left
justifies them.  A little weird, but that's what they do.  Don't waste any
time trying to figure out why the the |\caption|, located inside the centring
environment, doesn't centre!

This is easy, but it can lead to entries in tables being centred sometimes
when you don't want them to be, and it takes a good deal of {La\TeX}nical
investigation to sort out the order of operations and prevent that.
Sect.~18.14 shows a better way, the one that is used in the |PlanckTable.tex|
file.


\subsection How to make a table of a specified width

To make a table a certain width $x$, say |\halign to |$x$, where $x$ is a
dimension.  For example,

\medskip
|\halign to 4in{               %which makes the table 4\,in wide|

\medskip\noindent
or
\medskip
|\halign to \hsize             %which makes the table the width of the column or page|
\bigskip\noindent
If you do this, make sure that the |\tabskip| glue between columns can stretch
or shrink, e.g., |\tabskip=2em plus 2em minus 1em|.


\subsection How to run something across more than one column

This often is required in multilevel headings.  Use |\multispan|$n$, where $n$
is the number of columns to be spanned.  For example, to add a higher level
heading across the first two columns of our favourite table, type 

\medskip\noindent 
|\multispan2\hfil\csc Fruits \& Anima1s\hfil\cr|
\medskip
\medskip\noindent
Note that the |\cr| ends the line after two columns.  The third column will be
blank on this line.  If $n>9$, it must be enclosed in braces, e.g.,
|\multispan{13}|.   

Let's add a ``fruits \& animals'' heading across the first two columns of the
previous table.  The span of a multiple-column heading should be indicated with
a rule, and it works well to set the highest level heading in each column in a
caps-small caps font.  Doing all of this at once, including some
|\noalign{\vskip}| tweaks to the spacing of the rules, 

\bigskip
\begingroup\obeylines
|\halign{\it#\tabskip=2em&|
\hglue 0pt|        \hfil# and mice\hfil&|
\hglue 0pt|        \sl\hfil#\tabskip=0pt\cr|
|\noalign{\doubleline\vskip 5pt}|
|\multispan2\hfil\csc Fruits \& Animals\hfil\cr|
|\noalign{\vskip -3pt}|
|\multispan2\hrulefill\cr|
|\noalign{\vskip 2pt}|
|\omit\hfil Fruits\hfil&\omit\hfil Animals\hfil&\omit\hfil\csc Vegetables\hfil\cr|
|\noalign{\vskip 5pt\hrule\vskip 8pt}|
|apples&bears&yams\cr|
|\noalign{\vskip 4pt}|
|oranges&elephants&corn\cr|
|\noalign{\vskip 4pt}|
|watermelons&brontosauruses&rutabagas\cr|
|\noalign{\vskip 4pt}|
|peaches&llamas&peas\cr|
|\noalign{\vskip 4pt\hrule}|
|}|
\endgroup
\bigskip\noindent
produces

%\page

\halign{\it#\tabskip=2em&
        \hfil# and mice\hfil&
        \sl\hfil#\tabskip=0pt\cr
\noalign{\doubleline\vskip 5pt}
\multispan2\hfil\csc Fruits \& Animals\hfil\cr
\noalign{\vskip -3pt}
\multispan2\hrulefill\cr
\noalign{\vskip 2pt}
\omit\hfil Fruits\hfil&\omit\hfil Animals\hfil&\omit\hfil\csc Vegetables\hfil\cr
\noalign{\vskip 5pt\hrule\vskip 8pt}
apples&bears&yams\cr
\noalign{\vskip 4pt}
oranges&elephants&corn\cr
\noalign{\vskip 4pt}
watermelons&brontosauruses&rutabagas\cr
\noalign{\vskip 4pt}
peaches&llamas&peas\cr
\noalign{\vskip 4pt\hrule}
}

\bigskip\noindent


\subsection Leaders: filling up a column with rules or dots

Tables in astronomical journals are typically a matrix of information, with
rows and columns representing two different slices through the data.  In that
case, the entries in the leftmost column are names for the rows, and need to
be distinguished from the other columns.  A good way to do this is with
``leaders,'' rows of dots that lead the eye.

Define 
\medskip
|\def\leaderfil{\leaders\hbox to 5pt{\hss.\hss}\hfil}|
\medskip\noindent
After this definition, |\leaderfil| will fill up space with dots instead of
rules. 

Recall, though, that \TeX\ sets a column to the width of its widest entry.
|\leaderfil| doesn't make the column wider, it just fills out the narrower
entries with dots to equal the width of the widest entry.  The solution is to
fix the width of the first column to something wider than the widest entry.
How wide is a matter of taste and space.  Too wide and too narrow both look
bad.  In the following example the first column is set to 2\,in wide.  Note
that to get a legitimate minus sign you type a hyphen in math mode, i.e.,
|$-$|.

\bigskip
\begingroup\obeylines

|\halign{\hbox to 2.15in{#\leaderfil}\tabskip 2.2em&|
\hglue 0pt|        \hfil#\hfil&|
\hglue 0pt|        \hfil#\hfil&|
\hglue 0pt|        \hfil#\hfil\tabskip 0pt\cr |
|\noalign{\doubleline\hrule\vskip 5pt} |
|\omit&Latitude&Longitude&Diameter\cr |
|\omit\hfil Station\hfil&[\deg]&[\deg]&[m]\cr|
|\noalign{\vskip 4pt\hrule\vskip 6pt} |
|Bologna, Italy&44\pdeg5&$-11.3$&32\cr| 
|Crimea, USSR&44.5&$-34.0$&22\cr |
|Effelsberg, FRG&50.3&$-6.8$&100\cr| 
|Jodrell Bank, UK&53.1&2.3&25\cr |
|Noto, Italy&36.7&$-12.8$&32\cr |
|Onsala, Sweden&57.2&$-11.9$&20\cr |
|\noalign{\vskip 3pt\hrule\vskip 4pt} |
|}|

\endgroup

%\page


\halign{\hbox to 2.15in{#\leaderfil}\tabskip 2.2em&
        \hfil#\hfil&
        \hfil#\hfil&
        \hfil#\hfil\tabskip 0pt\cr 
\noalign{\doubleline\vskip 5pt} 
\omit&Latitude&Longitude&Diameter\cr 
\omit\hfil Station\hfil&[\deg]&[\deg]&[m]\cr
\noalign{\vskip 4pt\hrule\vskip 6pt} 
Bologna, Italy&44.5&$-11.3$&32\cr 
Crimea, USSR&44.5&$-34.0$&22\cr 
Effelsberg, FRG&50.3&$-6.8$&100\cr 
Jodrell Bank, UK&53.1&2.3&25\cr 
Noto, Italy&36.7&$-12.8$&32\cr 
Onsala, Sweden&57.2&$-11.9$&20\cr 
\noalign{\vskip 3pt\hrule\vskip 4pt} 
}

\bigskip
This is a nice looking table, but with one glaring defect considered in the
next section.


\subsection How to line up columns of numbers

It's easy to left justify, centre, or right justify entries in a column, but
the entries in astronomical tables are often numbers, which should be lined up
on the decimal point if they have decimal points, and on the right-most digit
for integers, but should generally be centred under the column heading.  The
problem is that the number of digits both before and after the decimal point,
or in the integer, may not be fixed, and the numbers can be negative.  The
third and fourth columns of the table in Sect.~18.12 show what happens when all
entries are simply centred.  It's ugly and hard to read.  The most
straightforward way to solve this looks a bit tricky and complicated at first,
but after some practice becomes merely tedious.

The idea is to use a little \TeX\ character-parsing trickery to redefine a
single character to take up the space of one numeral, and another to take up
the space of a plus or minus sign (in case you've never noticed, all
numerals have the same width, and plus and minus signs have the same width).

\begingroup\obeylines

|   \newdimen\digitwidth               % These five lines change what an asterisk |
|   \setbox0=\hbox{\rm 0}              % means to TeX. Instead of meaning|
|   \digitwidth=\wd0                   % "print an '*' here", it now means "leave| 
|   \catcode`*=\active                 % as much blank space as a single number |
|   \def*{\kern\digitwidth}            % takes up".| 
|%|
|   \newdimen\signwidth                % These five lines change the meaning of an|
|   \setbox0=\hbox{+}                  % exclamation mark in the same way, so that it| 
|   \signwidth=\wd0                    % leaves as much space as a plus or minus sign.|
|   \catcode`!=\active                 % These definitions will disappear at the end of| 
|   \def!{\kern\signwidth}             % the \vbox.|


\endgroup

\bigskip\noindent
These definitions need to be active only for the table, otherwise |*| and |!|
would be unavailable for their normal purposes.  Of course, you can use any
characters you want in these definitions, so if you need to use |*| or |!| in a table,
use |?| or some other character not needed in the table instead.

The previous table would become

\bigskip
\begingroup\obeylines
|{|
|%|
|   \newdimen\digitwidth|
|   \setbox0=\hbox{\rm 0}|
|   \digitwidth=\wd0| 
|   \catcode`*=\active|
|   \def*{\kern\digitwidth}| 
|%|
|   \newdimen\signwidth|
|   \setbox0=\hbox{+}| 
|   \signwidth=\wd0|
|   \catcode`!=\active| 
|   \def!{\kern\signwidth}|
|%|
|\halign{\hbox to 2.15in{#\leaderfil}\tabskip 2.2em&|
\hglue 0pt|        \hfil#\hfil&|
\hglue 0pt|        \hfil#\hfil&|
\hglue 0pt|        \hfil#\hfil\tabskip 0pt\cr| 
|\noalign{\doubleline\vskip 2pt}|
|\omit&Latitude&Longitude&Diameter\cr| 
|\omit\hfil Station\hfil&[\deg]&[\deg]&[m]\cr|
|\noalign{\vskip 4pt\hrule\vskip 6pt}|
|Bologna, Italy&  44.5&$-11.3$&*32\cr|
|Crimea, USSR&    44.5&$-$34.0&*22\cr|
|Effelsberg, FRG& 50.3&*$-6.8$&100\cr|
|Jodrell Bank, UK&53.1&!*2.3&*25\cr|
|Noto, Italy&     36.7&$-$12.8&*32\cr| 
|Onsala, Sweden&  57.2&$-$11.9&*20\cr| 
|\noalign{\vskip 3pt\hrule\vskip 4pt}}}|
\endgroup

\bigskip

{ 
% 
   \newdimen\digitwidth                               % These five lines change what an asterisk 
   \setbox0=\hbox{\rm 0}                              % means to TeX. Instead of meaning
   \digitwidth=\wd0                                   % "print an '*' here", it now means "leave 
   \catcode`*=\active                                 % as much blank space as a single number 
   \def*{\kern\digitwidth}                            % takes up" (N.B., all digits are the same width). 
%
   \newdimen\signwidth                                % These five lines change the meaning of an
   \setbox0=\hbox{+}                                  % exclamation mark in the same way, so that it 
   \signwidth=\wd0                                    % leaves as much space as a plus or minus sign.
   \catcode`!=\active                                 % These definitions will disappear at the end of 
   \def!{\kern\signwidth}                             % the \vbox.
% 
\halign{\hbox to 2.15in{#\leaderfil}\tabskip 2.2em&
        \hfil#\hfil&
        \hfil#\hfil&
        \hfil#\hfil\tabskip 0pt\cr 
\noalign{\doubleline\vskip 2pt} 
\omit&Latitude&Longitude&Diameter\cr 
\omit\hfil Station\hfil&[\deg]&[\deg]&[m]\cr
\noalign{\vskip 4pt\hrule\vskip 6pt} 
Bologna, Italy&  44.5&$-11.3$&*32\cr 
Crimea, USSR&    44.5&$-$34.0&*22\cr 
Effelsberg, FRG& 50.3&*$-6.8$&100\cr 
Jodrell Bank, UK&53.1&!*2.3&*25\cr 
Noto, Italy&     36.7&$-$12.8&*32\cr 
Onsala, Sweden&  57.2&$-$11.9&*20\cr 
\noalign{\vskip 3pt\hrule\vskip 4pt} 
}}

In some cases it may also be necessary to define a symbol to stand for the
width of a decimal point or some other symbol used in the table.

\subsection  Adding footnotes to a table

A\&A allows long captions after the title of the table, but in many
cases footnotes are still needed.  Different journals use different formats
for table footnotes.  A\&A sets footnotes the width of the column or
page, and provides |\tablefoot| and |\tablefoottext| commands to achieve their
style.  To use the |\halign| scheme described here, we have to capture the
width of one column or the whole page and use it to typeset the footnote,
depending on whether the table is one or two columns wide.

To do that, define two commands, |\endPlancktable| for one column, and
|\endPlancktablewide| for two-column tables.\footnote{$^{\rm\dag}$}{N.B. ---
In versions of Planck.tex earlier than 15 November 2010, this command was
called ``|\endtable|,''  which conflicted sometimes with a command buried deep
in the |aa.cls| file. The problem is fixed by |\endPlancktable|.}

\medskip
\begingroup\obeylines

|\def\endPlancktable{\tablewidth=\columnwidth|
\hglue 0pt|    $$\hss\copy\tablebox\hss$$|
\hglue 0pt|    \vskip-\lastskip\vskip -2pt}|

\medskip

|\def\endPlancktablewide{\tablewidth=\textwidth| 
\hglue 0pt|    $$\hss\copy\tablebox\hss$$|
\hglue 0pt|    \vskip-\lastskip\vskip -2pt}|

\endgroup


\medskip\noindent Also define a command for a table footnote:

\begingroup\obeylines
|\def\tablenote#1 #2\par{\begingroup \parindent=0.8em|
\hglue 0pt|    \abovedisplayshortskip=0pt\belowdisplayshortskip=0pt|
\hglue 0pt|    \noindent|
\hglue 0pt|    $$\hss\vbox{\hsize\tablewidth \hangindent=\parindent \hangafter=1 \noindent|
\hglue 0pt|    \hbox to \parindent{$^#1$\hss}\strut#2\strut\par}\hss$$|
\hglue 0pt|    \endgroup}|

\endgroup



\bigskip\noindent
In the La\TeX\ input file, it looks like this:

\begingroup\obeylines

|\setbox\tablebox=\vbox{\halign| 
|%|
|% table stuff|
|%|
|} |
|\endPlancktable          % or \endPlancktablewide|
|\tablenote {{\rm a}} Whatever you want in tablenote ``a.''\par |
\endgroup

\bigskip\noindent
That's it.   Here's an example:

\begingroup\obeylines

|\centerline{\csc Assumed EVN Stations$^{\rm a}$}| 
|\vskip -8pt |
|\setbox\tablebox=\vbox{| 
|%| 
|   \newdimen\digitwidth|
|   \setbox0=\hbox{\rm 0}|
|   \digitwidth=\wd0| 
|   \catcode`*=\active|
|   \def*{\kern\digitwidth}| 
|%|
|   \newdimen\signwidth|
|   \setbox0=\hbox{+}| 
|   \signwidth=\wd0|
|   \catcode`!=\active| 
|   \def!{\kern\signwidth}|

% 
|\halign{\hbox to 2.15in{#\leaderfil}\tabskip 2.2em&|
\hglue 0pt|       \hfil#\hfil&|
\hglue 0pt|        \hfil#\hfil&|
\hglue 0pt|        \hfil#\hfil\tabskip 8pt\cr| 
|\noalign{\vskip 3pt\hrule\vskip 1.5pt\hrule\vskip 5pt}|
|\omit\hfil Station\hfil&Latitude&W. Longitude&Diameter [m]\cr| 
|\noalign{\vskip 4pt\hrule\vskip 6pt}|
|\noalign{\doubleline\vskip 2pt}|
|\omit&Latitude&Longitude&Diameter\cr| 
|\omit\hfil Station\hfil&[\deg]&[\deg]&[m]\cr|
|\noalign{\vskip 4pt\hrule\vskip 6pt}|
|Bologna, Italy&  44.5&$-11.3$&*32\cr|
|Crimea, USSR&    44.5&$-$34.0&*22\cr|
|Effelsberg, FRG& 50.3&*$-6.8$&100\cr|
|Jodrell Bank, UK&53.1&!*2.3&*25\cr|
|Noto, Italy&     36.7&$-$12.8&*32\cr| 
|Onsala, Sweden&  57.2&$-$11.9&*20\cr| 
|\noalign{\vskip 3pt\hrule\vskip 4pt}|
|} } |
|\endPlancktable|                               
|\tablenote {{\rm a}} An aperture efficiency of 0.3 was assumed for Effelsberg.
All other station ||parameters were assumed to be identical to those of VLBA
stations.\par|

\endgroup

\page


\bigskip
\noindent
This produces


\centerline{\csc Assumed EVN Stations$^{\rm a}$}
\setbox\tablebox=\vbox{
% 
   \newdimen\digitwidth
   \setbox0=\hbox{\rm 0}
   \digitwidth=\wd0
   \catcode`*=\active
   \def*{\kern\digitwidth} 
%
   \newdimen\signwidth
   \setbox0=\hbox{+}
   \signwidth=\wd0
   \catcode`!=\active 
   \def!{\kern\signwidth}

% 
\halign{\hbox to 2.15in{#\leaderfil}\tabskip 2.2em&
       \hfil#\hfil&
       \hfil#\hfil&
       \hfil#\hfil\tabskip 8pt\cr
\noalign{\doubleline\vskip 2pt}
\omit&Latitude&Longitude&Diameter\cr
\omit\hfil Station\hfil&[\deg]&[\deg]&[m]\cr
\noalign{\vskip 4pt\hrule\vskip 6pt}
Bologna, Italy&  44.5&$-11.3$&*32\cr
Crimea, USSR&    44.5&$-$34.0&*22\cr
Effelsberg, FRG& 50.3&*$-6.8$&100\cr
Jodrell Bank, UK&53.1&!*2.3&*25\cr
Noto, Italy&     36.7&$-$12.8&*32\cr
Onsala, Sweden&  57.2&$-$11.9&*20\cr
\noalign{\vskip 3pt\hrule\vskip 4pt}
} }
\endPlancktable           
\tablenote {{\rm a}} An aperture efficiency of 0.3 was assumed for Effelsberg.
All other station parameters were assumed to be identical to those of VLBA
stations.\par

%\bigskip
Note that A\&A prefers the footnote ``letter'' to be in a roman rather than
italic font.  Study the title line and what follows |\tablenote| in the
example input file to see how to do this.  The extra curly brackets are
necessary.  A\&A also runs all the footnotes together in a single
``{\bf Notes:}'' paragraph at the bottom of the table.  

\vskip 0pt plus 1600fill


\subsection Paragraphs as table entries

Table entries sometimes are more or less like paragraphs.  In such cases, you
want \TeX\ to choose line breaks and justify the margins.  The way to do it is
with a variant of putting things in |\vbox|es, as shown in the example below.
A |\vtop| box is like a |\vbox|, except the top line of the |\vtop| box lines
up with the entries in the other columns.  If a |\vbox| had been used, the
{\it bottom\/} line of the |\vbox| would have lined up with the entries in the
other columns.  Here's an example.  \TeX's ominous
black boxes indicate that there is
no way to break the lines that satisfies the specifications on uniformity of
word spacing in effect at the time.  This is a common problem in typesetting
narrow columns.  Read the entries in Column~2 for two solutions, both of which
should now be in your ``toolkit.''


%\begingroup\obeylines
\setbox\tablebox=\vbox{
\halign{\hbox to 2.0in{#\leaderfil}\tabskip 2em&
        \vtop{\hsize 3.0in\hangafter=1\hangindent=1em\noindent\strut#\strut\par}\tabskip=0pt\cr
\noalign{\doubleline\vskip 2pt}
\omit\hfil Column 1\hfil&\omit\hfil Column 2\hfil\cr
\noalign{\vskip 6pt\hrule\vskip 6pt}
Entry 1&This is a long entry to show how a paragraph can be typeset and used
as a table entry.  A |\vtop| box is used so that the first line of the
paragraph lines up horizontally with the entry in column \#1, and hanging
indentation gives a good appearance.\cr 
\noalign{\vskip 3pt}
Entry 2&Narrow columns can be difficult for line breaks.  Relaxing \TeX's
default and stringent requirement on uniformity of word spacing by increasing
|\tolerance|, possibly all the way to 10000 (in effect, no requirement), will
solve the problem, possibly at the cost of wide spaces between words in some
lines.\cr
\noalign{\vskip 3pt}
Entry 3&It is sometimes easier, however, to play around with the width of the
column.  In this example, |\hsize=3in| sets the width of Column 2.
|\hsize=3.2in| makes the problems go away.\cr
\noalign{\vskip 6pt}
Entry 4&\omit\vbox{\hsize 3.0in\hangafter=1\hangindent=1em\noindent\strut This
shows why a |\vtop| should be used instead of a |\vbox|, which lines up on the
last line of the paragraph instead of the first.\strut\par}\cr
\noalign{\vskip 3pt\hrule\vskip 4pt}
}  }
\endPlancktable 
%\endgroup

\bigskip\noindent
Here's what produced the table above:

\begingroup\obeylines
|\setbox\tablebox=\vbox{|
|\halign{\hbox to 2.0in{#\leaderfil}\tabskip 2em&|
\hglue 0pt|        \vtop{\hsize 3.0in\hangafter=1\hangindent=1em\noindent\strut#\strut\par}|
\hglue 0pt|        \tabskip=0pt\cr|
|\noalign{\doubleline\vskip 2pt}|
|\omit\hfil Column 1\hfil&\omit\hfil Column 2\hfil\cr|
|\noalign{\vskip 6pt\hrule\vskip 6pt}|
|Entry 1&This is a long entry to show how a paragraph can be typeset and used as a |
\hglue 0pt|   table entry.  A \vtop box is used so that the first line of the paragraph lines|
\hglue 0pt|   up horizontally with the entry in column \#1, and hanging indentation gives a |
\hglue 0pt|   good appearance.\cr|
|\noalign{\vskip 3pt}|
|Entry 2&Narrow columns can be difficult for lines breaks.  Relaxing \TeX's |
\hglue 0pt|   default and stringent requirement on uniformity of word spacing by increasing |
\hglue 0pt|   \tolerance, possibly all the way to 10000 (in effect, no requirement), will solve|
\hglue 0pt|   the problem, possibly at the cost of wide spaces between words in |
\hglue 0pt|   some lines.\cr|
|\noalign{\vskip 3pt}|
|Entry 3&It is sometimes easier, however, to play around with the width of |
\hglue 0pt|   the column.  In this example, \hsize=3in sets the width of Column 2. |
\hglue 0pt|   \hsize=3.2in makes the problems go away.\cr|
|\noalign{\vskip 6pt}|
|Entry 4&\omit\vbox{\hsize 3.0in\hangafter=1\hangindent=1em\noindent\strut This |
\hglue 0pt|   shows why a \vtop should be used instead of a \vbox, which lines up on the|
\hglue 0pt|   last line of the paragraph instead of the first.\strut\par}\cr|
|\noalign{\vskip 3pt\hrule\vskip 4pt}|
|}  }|
|\endPlancktable|
\endgroup

\New
\appendix A; Use of commas

\noindent
Inserting commas can be helpful for making sentences readable.
However, there are cases when it is wrong to use commas, for example, the
following two phrases are both correct but mean different things:
\example{My brother, John, is 27.}
\example{My brother John is 27.}
\noindent
The latter would be applicable if I had more than one brother and wanted to
distinguish which one, the first if I had only one brother.  More obviously,
\example{Let's eat Grandma.}
\example{Let's eat, Grandma.}

\noindent mean different things!!

Gowers's ``Complete Plain Words'' distinguishes ``defining'' and ``commenting''
clauses; there are no commas for defining clauses, but commas for commenting
ones.  The following example is given:
\example{Pilots, whose minds are dull, do not usually live long.}
\noindent
This is a defining clause, not a commenting one, and commas are incorrect
in this case.  In our field, we should distinguish, for example,
\example{The parameter $n_{\rm s}$ is poorly constrained}
\noindent
(a defining clause, giving which parameter) from
\example{The scalar power-law index, $n_{\rm s}$, is poorly constrained}
\noindent
(a commenting clause, since ``$n_{\rm s}$'' can be omitted without changing
the meaning).

As a closing remark, Gowers also says ``The use of commas cannot be learned
by rule.''
\black

\bye

%%Remaining issues:
%% End with a short section called something like "Don't forget"
%%  - which includes a reminder to remove commented-out parts from TeX files
%% Check for Planck in present tense in papers, e.g., "Planck observes"
%% Sort out where "repositories" files are on PPM and point there?
%% "allpapers" update?
%% Any more LaTeX fixes or problems with hyperref?
%% Sort out what to say about masks?
%% Use of \begin{equation} vs $$, giving left-justified vs centred
%% Include a comment about VLA, GBT etc. acronyms?
%% Discussing of best-fit vs best-fitting - appendix?  well-fitted vs well-fit?
%% "feedhorn" one word or two?  how about "bandpass"
%% FIGURE ISSUES:
%%   MicroK rather than [microK] for colour bars?
%%   Comment on size of colour bar - leaving space for text?
%%   Use of minus signs in numbers <0 in figure labels?
%%   For difference maps use word "minus" if necessary to make minus sign clear?
%%   en.wikibooks.org—Floats,_Figures_and_Captions ?
